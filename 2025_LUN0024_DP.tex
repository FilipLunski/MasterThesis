
\documentclass[czech,master]{diploma}
% Dalsi doplnujici baliky maker
\usepackage[autostyle=true,czech=quotes]{csquotes} % korektni sazba uvozovek, podpora pro balik biblatex
\usepackage[backend=biber, style=iso-numeric, alldates=iso]{biblatex} % bibliografie
\usepackage{dcolumn} % sloupce tabulky s ciselnymi hodnotami
\usepackage{subfig} % makra pro "podobrazky" a "podtabulky"
\usepackage[cpp]{diplomalst}



\usepackage{multirow}
\usepackage{pgfplots}
\pgfplotsset{compat=newest}
\usepackage{amsmath}

\usepackage{pgffor}
\usepackage{catchfile}
\usepackage{svg}

% Novy druh tabulkoveho sloupce, ve kterem jsou cisla zarovnana podle desetinne carky
\newcolumntype{d}[1]{D{,}{,}{#1}}



% Zadame pozadovane vstupy pro generovani titulnich stran.
\ThesisAuthor{Bc. Filip Łuński}

\ThesisSupervisor{Ing. Tomáš Wiszczor, Ph.D.}

\CzechThesisTitle{Využití kamerového systému pro zajištěni bezpečnosti osob na pracovišti}

\EnglishThesisTitle{Use of Surveillance Cameras to Ensure the Safety of People in the Workplace}

\SubmissionYear{2025}

\ThesisAssignmentFileName{ThesisSpecification_LUN0024_vsboee2404016E.pdf}


% \Acknowledgement{Rád bych na tomto místě poděkoval všem, kteří mi s prací pomohli, protože bez nich by tato práce nevznikla.}


\CzechAbstract{Automatizované sledování videa z bezpečnostních kamer bylo po mnoho let používáno spíše v oblasti zabezpečení, nikoliv pro zajištění bezpečnosti osob. Rozvoj technologie v posledních letech, zejména v oblasti umělé inteligence a neuronových sítí, však umožňuje koplešnější analýzy chování osob. V této práci je tak navrženo řešení pro detekci pádu osob v obrazovém toku v reálném čase. Řešení je postaveno na kombinaci dvou neuronových sítí. První detekuje všechny osoby v obraze a jejich klíčové body, druhá tyto body klasifikuje do tříd normální a upadl. V práci je popsán výběr detektoru pózy, návrh architektury klasifikační sítě a implementace výsledného detektoru pádu. }

\CzechKeywords{python, strojové učení, neuronové sítě, konvoluční neuronové sítě, rekurentní neuronové sítě, GRU, LSTM, PyTorch, detekce pozy, detekce chování, detekce pádu, YOLO  }

\EnglishAbstract{This is English abstract. Lorem ipsum dolor sit amet, consectetuer adipiscing elit. Fusce tellus odio, dapibus id fermentum quis, suscipit id erat. Aenean placerat. Vivamus ac leo pretium faucibus. Duis risus. Fusce consectetuer risus a nunc. Duis ante orci, molestie vitae vehicula venenatis, tincidunt ac pede. Aliquam erat volutpat. Donec vitae arcu. Nullam lectus justo, vulputate eget mollis sed, tempor sed magna. Curabitur ligula sapien, pulvinar a vestibulum quis, facilisis vel sapien. Vestibulum fermentum tortor id mi. Etiam bibendum elit eget erat. Pellentesque pretium lectus id turpis. Nulla quis diam.}

\EnglishKeywords{python, machine learning, neural networks, convolutional neural networks, reccurent neural networks, GRU, LSTM, PyTorch, pose estimation, behaviour detection, fall detection, YOLO }

% Seznam zkratek a symbolů 
\AddAcronym{AF}{Aktivační funkce}

\AddAcronym{NN}{Neural network – neuronová síť}
\AddAcronym{ANN}{Artificial neural network – umělá neuronová síť}
\AddAcronym{FFNN}{Feedforward Neural Network – dopředná neuronová síť}
\AddAcronym{CNN}{Convolutional neural network – konvoluční neuronová síť}
\AddAcronym{RNN}{Recurrent neural network – rekurentní neuronová síť}
\AddAcronym{LSTM}{Long short-term memory – dlouhá krátkodobá paměť}
\AddAcronym{GRU}{Gated recurrent unit}
\AddAcronym{AI}{Artificial intelligence – umělá inteligence}
\AddAcronym{ML}{Machine learning – strojové učení}
\AddAcronym{DL}{Deep learning – hluboké učení}
\AddAcronym{RoI}{Region of interest – oblast zájmu}
\AddAcronym{PAF}{Part affinity field – pole propojení klíčových bodů}
\AddAcronym{ReLU}{Rectified Linear Unit}
\AddAcronym{PReLU}{Parametric Rectified Linear Unit}
\AddAcronym{Adam}{Adaptive Moment Estimation}
\AddAcronym{BP}{Backpropagation}
\AddAcronym{BPTT}{Backpropagation Through Time – zpětné šíření chyby v čase}
\AddAcronym{BCE}{Binary Cross-Entropy}
\AddAcronym{RMSprop}{Root Mean Square Propagation}

\endinput

% Vlozeni bibliografie
\addbibresource{biblatex.bib}


% Zacatek dokumentu
\begin{document}

% Nechame vysazet titulni strany.
\MakeTitlePages

% Jsou v praci obrazky nebo tabulky? Pokud ano vysazime jejich seznam a odstrankujeme.
% Pokud ne smazeme nasledujici dve makra.
\listoffigures
\clearpage

\listoftables
\clearpage

% A nasleduje text zaverecne prace.
\chapter{Úvod}
\label{chap:Introduction}

Kamerové systémy jsou využívány již mnoho let a jejich využití je stále širší.
Již před několika lety se odhadovalo, že celkový počet bezpečnostních kamer ve
světě přesahuje miliardu \cite{surveillance}. Využívány jsou v průmyslu,
dopravě, obchodech, veřejných prostorech, zdravotnictví či domácnostech.

Zpočátku bylo možné video sledovat pouze živě, později, s příchodem videokazet,
bylo možné záznam sledovat rovněž zpětně. Digitální éra a síťové kamery
umožnily přístup ke kamerovým záznamům z libovolného místa na světě. V poslední
době se také začalo nahrazovat živé sledování automatickým zpracováním obrazu a
detekcí událostí s využitím technik umělé inteligence.

Kamerové systémy se používají zejména ve dvou oblastech: zabezpečení (ang.
security), myšleno jako ochrana před úmyslnými hrozbami a protiprávními činy,
jako jsou krádeže, poškozování majetku, či neoprávněný vstup; a bezpečnost
(ang. safety), což zahrnuje ochranu před nehodami a náhodnými hrozbami, jako
jsou pády, požár, úniky nebezpečných látek, či porušování bezpečnostních
předpisů.

Jak již bylo zmíněno, kamery lze využívat jednak pro živé sledování, jednak pro
záznam a jeho analýzu po události. Kamerové záznamy jsou důležité zejména pro
zpětnou analýzu incidentů, důkazní materiál pro soudní spory, zjišťování příčin
nehod, či pro zlepšení bezpečnostních opatření. Živé sledování videa se pak
snaží incidentům přímo předcházet. Bylo však prokázáno, že schopnost lidského
pozorovatele detekovat nebezpečí se velmi snižuje s délkou sledování a s počtem
monitorovaných kamer \cite{soton371614}. Právě proto se s příchodem technik
umělé inteligence začalo využívat automatické zpracování obrazu a detekce
hrozeb, nebezpečí, nebo již probíhajících incidentů v jejich počátcích. Tyto
techniky pak úplně nahrazují lidského pozorovatele, nebo mu pomáhají včas
zpozorovat nebezpečí a zareagovat.

Automatická analýza obrazu je používaná již několik desítek let, většinou ale
spíše pro oblast zabezpečení, než pro bezpečnost. To z toho důvodu, že úlohy,
jako identifikace neoprávněného vstupu, detekce zbraní, rozpoznávání registračních značek vozidel nebo
podezřelých osob dle obličeje jsou pro algoritmy mnohem jednodušší, než
například detekce pádu, nouzové situace či zdravotního problému. Jedná se zde
totiž o vysokou míru abstrakce, kdy i člověk může špatně identifikovat některé
situace. Hlavním problémem těchto komplexnějších analýz je vysoká falešná
pozitivita, kdy je například těžké rozeznat člověka trénujícího běh od člověka
utíkajícího před nebezpečím. Nicméně rozvoj v oblasti hlubokého učení a
konvolučních neuronových sítí, jako i vývoj a dostupnost hardwaru podporujícího
tyto techniky, umožňuje dnes jejich využití i pro složitější úlohy.

Ve firmě Linde jsou kamerové systémy používány v mnoha průmyslových provozech,
nicméně chybí ucelený systém pro automatickou analýzu obrazu a detekci různých
druhů nebezpečí. Úkolem tedy v budoucnu bude navrhnout a implementovat
modulární systém s možností sledování konkrétních nebezpečí na konkrétních
místech. Ty budou zahrnovat například detekci pádu, požáru, zdravotních
problémů, nebo porušování bezpečnostních opatření. Systém pak bude v případě
rozpoznání nějaké hrozby informovat příslušného pracovníka.

Tato práce je zaměřena pouze na jednu z těchto úloh, a to na detekci pádu. Pád
může mít různé příčiny, ať už je to zdravotní problém jako ztráta vědomí, nebo
zakopnutí. Někdy se zdá, že samotné zakopnutí je banální problém, nicméně pokud
se na pracovišti nenachází nikdo, kdo by mohl pomoct, a poškozený není schopen
sám přivolat pomoc, může vést takový incident k vážným následkům.

První část práce bude věnována teoretickým základům, seznámení z neuronovými
sítěmi. Kromě obecného popisu neuronových sítí bude pozornost věnována také jejich pokročilejším architekturám, které budou v řešení použity. Další kapitoly budou zaměřeny na detekci osob a odhad jejich klíčových bodů. Budou představeny různé
přístupy a otestovány různé algoritmy s ohledem na výkon, možnou hardwarovou
akceleraci a preciznost. Předmětem další části bude samotná detekce pádu, tedy
algoritmus, který na základě odhadnutých klíčových bodů určí, zda došlo k pádu,
či nikoliv. Poslední část práce bude zaměřena na kompletaci a otestování výsledného
řešení a zhodnocení jeho výkonu.

\endinput
% \part{Teorie}
\chapter{Neuronové sítě}
\label{sec:NN}

Umělá neuronová síť (ang. Artificial Neural Network - ANN) nebo jen neuronová
síť (ang. Neural Network - NN) je výpočetní model inspirovaný biologickými
nervovými systémy v lidském mozku. Na rozdíl od konvenčních výpočetních modelů,
které zpracovávají informace algoritmicky, a tedy postupují dle předem určeného
postupu, se informace v tomto modelu šíří v síti váh mezi jednotlivými neurony.
Jelikož je výstup ze sítě dané architektury závislý hlavně na numerických
parametrech, zejména váhách jednotlivých spojů mezi neurony, lze funkčnost sítě
měnit bez změny programu pouhou změnou těchto parametrů, a to i automaticky v
procesu trénování modelu.

Nyní krátce projdeme historií vývoje neuronových s síti.

\section{Historie}
\label{sec:NN_History}

\subsection{Prvopočátky}
První matematický model neuronové sítě byl popsán v roce 1943 dvěma
neurofyziology - Warrenem McCullochem a Walterem Pittsem. \cite{McCulloch1943}
Model byl založen na síti jednoduchých logických prvků, které provedou vážený
součet svých vstupů a na výstup odešle signál založený na prahové funkcí.

V roce 1958 pak Frank Rosenblatt představil elektronicky model neuronové sítě.
Základní jednotku, postavenou na McCulloch-Pittsově modelu, nazval perceptron.
\cite{Rosenblatt1958} Jeho architektura byla podobná modelu znázorněnému na
obrázku \ref{fig:neuron}, kde aktivační funkce je prahová funkce. Rosenblattův
stroj - Mark I Perceptron - byl postavený pro rozpoznávání jednoduchých vzorů v
obrazech. Hlavním omezením tohoto modelu bylo, že byl schopen rozlišovat pouze
lineárně separovatelné třídy.

Další systém - ADALINE (Adaptive Linear Neuron) - byl představen Bernardem
Widrowem and Tedym Hoffem v roce 1960. Tento model umělého neuronu byl velmi
podobný perceptronu, na rozdíl od něj ale neobsahoval prahovou ale lineární
funkcí, výstup tedy nebyl binární ale spojitý. Pro učení pak byla využitá
metoda nejmenších čtverců, která minimalizovala chybu mezi skutečným a
očekávaným výstupem. \cite{nn_history}

I když ve svých počátcích přitahoval koncept umělé inteligence mnoho vědců jako
i sponzorů, v následujících létech zájem ochabl, jelikož nebylo dosaženo
předpokládaných výsledků, hlavně s ohledem na tehdejší stav vývoje hardwaru a
obecně výpočetní techniky. Proto se tomuto období někdy říká Ai Winter.
Neznamená to ale, že ti, kteří se oboru nadále věnovali, nedosáhli významných
výsledků. \cite{nn_history}

\subsection{Objev backpropagation}
Významným milníkem v historii neuronových sítí byl objev algoritmu
backpropagation, zvaného taky algoritmus zpětného šíření chyby. Tento
algoritmus byl vyvinut v roce 1974 Paulem Werbosem, popularitu ale dosáhl až po
nezávislém objevení v roce 1986 Davidem Rumelhartem et al.
\cite{backpropagation}

Tento algoritmus umožnil trénovat sítě s více vrstvami, což položilo základ
hlubokému učení. Algoritmus využívá metodu gradientního sestupu v kombinaci s
řetězovým pravidlem derivací k nalezení optimálních vah sítě vedoucích k
minimalizaci chyby.

Vynález backpropagation byl jedním z hlavních důvodů, proč se v 80. letech
obnovil zájem o neuronové sítě a umělou inteligenci obecně. 



\begin{figure}[]
    \centering
    \includegraphics[width=0.5\textwidth]{Figures/neuron.png}
    \caption{Model umělého neuronu \cite{lagan}}
    \label{fig:neuron}
\end{figure}

\endinput
\chapter{Konvoluční neuronové sítě}
\label{sec:CNN}

\endinput
\chapter{Rekurentní neuronové sítě}
\label{chap:RNN}

Rekurentní neuronové sítě (ang. Recurrent Neural Networks - RNN) je kategorie
neuronových sítí, které do své architektury zapojují zpětnou vazbu. Na rozdíl
od dopředných sítí, které zpracovávají jednotlivé vstupy nezávisle, rekurentní
sítě spolu s aktuálním vstupem při evaluaci zohledňují nějakým způsobem i
výsledek předchozí iterace. Jejich využití tedy je ve dvou oblastech: analýza
změn pozorovaného objektu v čase (např. sledování pohybu, analýza chování či
predikce časových řad) a zpracování kontextuálních informací jako je např.
přirozený jazyk.

Jelikož v případě pádu se nejedná o statickou informaci, ale o jistý druh
pohybu, mohly by právě rekurentní sítě stanovit optimální cestu pro naše řešení
\cite{dhruv2020image}. Pojďme se tedy nyní podívat na to, jak rekurentní neuronové sítě fungují a jaké
jsou jejich nejpoužívanější architektury.

\section{Základní principy}

Data, která v aktuální iteraci přebíráme z předchozí iterace, se často označují
jako skrytý stav (ang. hidden state). Je to forma paměti, která se s každou
iterací aktualizuje. Často je reprezentován jako stavová vrstva (ang. state
layer nebo context layer), která přijímá hodnoty z výstupu neuronů, uchovává je
mezi iteracemi a předává je spolu se vstupními daty na vstup neuronů. Na
obrázku \ref{fig:rnn} je tato vrstva reprezentována neurony $c_i$.

Nejjednodušší forma rekurentní neuronové sítě je NN s jednou skrytou vrstvou;
tato vrstva kromě dat ze vstupní vrstvy přijímá také výstup předchozí iterace
buď svých vlastních neuronů, anebo z neuronů výstupní vrstvy, viz obrázek
\ref{fig:rnn}. Obrázek je zjednodušený, v praxi jsou stavová a skrytá vrstva
plně propojeny. Tyto architektury se jmenují Elmanova síť \cite{elman}, resp.
Jordanova síť \cite{jordan}, od jejích tvůrců. Tyto sítě jsou taky známé jako
jednoduché rekurentní sítě (ang. Simple Recurrent Networks - SRN).

I když pojem rekurentních sítí byl známý už od začátků neuronových sítí jako
takových a byly i případy jejích použití, právě tyto sítě patřily k prvním,
které používaly pro trénování algoritmus backpropagation. Jednoduché rekurentní
sítě uchovávají pouze krátkodobé vzory a jsou vhodné spíše pro jednoduché
úlohy, jako je např. predikce časových řad.

\begin{figure}[]
    \centering
    \includegraphics[width=0.8\textwidth]{Figures/rnn.png}
    \caption{Základní architektury RNN \cite{aksoy}}
    \label{fig:rnn}
\end{figure}

\subsection{Hluboké rekurentní sítě}

Stejně jako u dopředných neuronových sítí, kde se od jednoduchého perceptronu
přešlo k hlubokým sítím, se i rekurentní sítě rozšířily na více vrstev. V
hlubokých rekurentních neuronových sítích (ang. deep RNN - DRNN) jsou pak
jednotlivé vrstvy většinou podobné struktuře Elmanovy sítě - zpětná vazba je
předávána pouze v rámci jedné vrstvy, nikoliv mezi vrstvami RNN (například z
výstupní vrstvy do první skryté vrstvy). Má to několik důvodů. Trénování sítě
se zpětnou vazbou mezi vrstvami by bylo velmi složité a obtížné. Taky, u
neuronových sítí obecně platí, že každá vrstva sítě se učí pochopit problém na
jiné úrovni abstrakce, zpětná vazba přes několik vrstev by pak mohla narušit
stabilitu tohoto procesu a omezit kvalitu učení.

\subsection{Trénování rekurentních sítí}

Pro pochopení rekurentních neuronových sítí je třeba si vysvětlit, jak se
trénují. Pro vizualizaci trénování RNN se tyto sítě takzvaně rozbaluje v čase
(ang. unrolling). Znamená to, že jednotlivé iterace vizualizujeme jako sekvenci
stejných sítí (stejné váhy), které v čase $t$ přijímají vstup $x_t$ a vracejí
výstup $y_t$, viz obrázek \ref{fig:bptt}. Zároveň místo smyček znázorňujících
zpětnou vazbu přijímá skrytá vrstva v čase $t$ stav $c_{t-1}$ z předchozí
iterace. Takto je propojená mezi iteracemi každá skrytá vrstva (na obrázku
\ref{fig:bptt} vizualizováno propojení přes stavovou vrstvu).

\begin{figure}[]
    \centering
    \includegraphics[width=0.75\textwidth]{Figures/BPTT.pdf}
    \caption{Unrolling hluboké RNN}
    \label{fig:bptt}
\end{figure}

Při trénování se pak používá algoritmus zpětného šíření chyby v čase (ang.
backpropagation through time - BPTT). Algoritmus funguje stejně jako klasický
backpropagation, šíří se ale nejenom vrstvami, ale i iteracemi. Unrolling nám
pomáhá backpropagation pochopit, jednotlivé iterace totiž jsou naskládány jako
vrstvy a celou síť řešíme jako klasickou dopřednou NN.

\subsection{Problémy mizejícího a explodujícího gradientu}

Výše popsané základní rekurentní neuronové sítě, někdy označovány jako vanilla
RNN, trpí několika zásadními problémy. U dopředných sítí jsme zmiňovali problém
mizejícího gradientu (ang. vanishing gradient), vystupující zejména u hlubších
sítí. Ten se projevuje i u RNN a je zesílený tím, že jsou jednotlivé iterace
naskládané na sebe, podobně jako vrstvy. Zejména pak u delších sekvencí budou
mít dřívější vstupy velmi malý vliv na učení sítě.

U RNN se taky projevuje problém opačný - explodující gradient (ang. exploding
gradient). Ten způsobuje, že v průběhu sekvence se váhy začnou exponenciálně
zvětšovat a dosáhnou tak nepřiměřeně velkých hodnot.

Podívejme se, co přesně tyto problémy způsobuje. Součástí algoritmu
backpropagation je počítání parciální derivace ztrátové funkce podle
jednotlivých vah. V případě BPTT potřebujeme mimo jiné počítat parciální
derivace skrytého stavu mezi jednotlivými iteracemi $\frac{\partial
        h_{t-1}}{\partial h_t}$. Tyto derivace následně opakovaně násobíme při použití
řetězového pravidla. Pokud je tato derivace $\frac{\partial h_{t-1}}{\partial
        h_t}<1$, jeho vynásobení bude mít za následek postupné zmenšování gradientu.
Pokud budeme například mít sekvencí $100$ iteraci, pak i kdyby se gradienty v
každé iteraci zmenšovaly $0,9$ krát, po $100$ iteracích by gradient klesl na
hodnotu $0,9^{100} \approx 2,7 \times 10^{-5}$, což je prakticky nula. Pokud se
naopak bude gradient zvětšovat $1,1$ krát, po $100$ iteracích by gradient
vzrostl na $1,1^{100} \approx 13 780$, což způsobí úplnou destabilizaci sítě a
nedosáhneme žádného výsledku. Vidíme tedy, že v případě, kdy je $\frac{\partial
        h_{t-1}}{\partial h_t}>1$, dochází k explodujícímu gradientu.

Z důvodu těchto problémů byly vyvinuty složitější rekurentní struktury. Jejich
architektura je v podstatě podobná, jednotlivé vrstvy jsou ale zastoupeny
jinými stavebními bloky, které umožňují zejména širší pochopení kontextu a
efektivnější proces trénování. Vanilla RNN se v praxi dnes využívají velmi
zřídka. K nejpoužívanějším architekturám patří LSTM (ang. long short-term
memory) a GRU (ang. gated recurrent unit), které nyní popíšeme.

\section{LSTM}

Dlouhá krátkodobá paměť (ang. long short-term memory - LSTM ), představena
Hochreiterem a Schmidhuberem v roce 1997, je typ rekurentní neuronové sítě,
který byl navržen tak, aby překonal problémy mizejícího a explodujícího
gradientu.

\begin{figure}[]
    \centering
    \includegraphics[width=0.7\textwidth]{Figures/LSTM_unit.pdf}
    \caption{Jednotka LSTM}
    \label{fig:lstm}
\end{figure}

Její základem je jednotka, viz obrázek \ref{fig:lstm}.1, která ve třech
stádiích aktualizuje krátkodobou a dlouhodobou paměť. Dlouhodobá paměť je
reprezentovaná pomocí stavu buňky (ang. cell state, na obrázku \ref{fig:lstm}.1
$c_t$), který je postupně upravován a nakonec předán další iteraci. Dokáže
uchovávat dlouhodobé závislosti. Krátkodobá paměť je reprezentována pomocí
skrytého stavu. Je použita pro úpravu dlouhodobé paměti, v konečném stadiu je
ale vždy v rámci dané iterace vytvořena nová. Je tak vhodná pro uchování
krátkodobých závislostí. Na obrázku \ref{fig:lstm} je znázorněna jako $h_t$.

Jednotka LSTM má tři hlavní komponenty - zapomínací bránu (ang. forget gate),
vstupní bránu (ang. input gate) a výstupní bránu (ang. output gate). Brány
určují, které informace mají být předány dál.

První, zapomínací brána určuje, které informace z dlouhodobé paměti $c_{t-1}$
se dostanou dále - co má jednotka zapomenout, resp. zapamatovat. Ve vstupní
bráně se nejprve vytvoří kandidátní stav buňky. Ten je výsledkem neuronové
vrstvy s tangenciální aktivační funkcí, do které vstupuje aktuální vstup $x_t$
a krátkodobá paměť $h_{t-1}$. Pak se určí, které informace z kandidátního stavu
buňky se přičtou do stavu buňky a vznikne tak aktuální stav buňky $c_t$. Ve
výstupní bráně se pomocí tangenciální aktivační funkce vytvoří na základě stavu
buňky $c_t$ kandidátní skrytý stav. Pak se určí, které z těchto informací budou
tvořit nový skrytý stav $h_t$.

V každé bráně tedy máme informace, pro které určujeme, zda je poslat dále či
nikoliv, nazvěme je propouštěný obsah (předchozí stav buňky, kandidátní stav
buňky či kandidátní skrytý stav). Toto určení se provádí vždy pomocí neuronové
vrstvy se sigmoidní aktivační funkcí. Do těchto vrstev vstupuje vždy předchozí
skrytý stav $h_{t-1}$ a aktuální vstup $x_t$. Výstupem je hodnota mezi $0$ a
$1$ pro každou informaci. Pak se tento výsledek vynásobí propouštěným obsahem.
Pokud je výstup této vrstvy $0$, informace se nepředávají dál, pokud je $1$,
informace se předávají dále. Vstupy do všech neuronových vrstev jsou vždy
vynásobeny váhami, ty ale nejsou pro jednoduchost na obrázku \ref{fig:lstm}.1
zobrazeny. Na obrázku \ref{fig:lstm_deep} je znázorněna rozvinutá hluboká LSTM
síť. Jednotlivé vrstvy sítě jsou naskládány vertikálně, jednotlivé iterace pak
jsou rozvinuty vedle sebe. Jednotlivé vrstvy si předávají skrytý stav -
krátkodobou paměť, mezi iteracemi si pak daná vrstva předává krátkodobou i
dlouhodobou paměť.

\begin{figure}
    \centering
    \includegraphics[width=0.70\textwidth]{Figures/LSTM_deep.pdf}
    \caption{Rozvinutá hluboká LSTM síť}
    \label{fig:lstm_deep}
\end{figure}

LSTM sítě vynikají v udržování dlouhodobých závislostí a složitých struktur.
Jelikož mají tři brány, je síť schopná přesně rozhodnout, které informace chce
dlouhodobě uchovávat, které naopak mají větší vliv na aktuální výstup a které
mají být zapomenuty. Je to ale za cenu většího výpočetního nároku a
složitějšího trénování. Taky je pro tyto sítě vhodné mít větší množství
trénovacích dat, jinak může dojít k přetrénování. Využívá se tak zejména pro
predikci dlouhých a komplexních časových sekvencí či zpracování přirozeného
jazyka. Zejména u přirozeného jazyka se LSTM sítě osvědčily jako velmi
efektivní. Potřebujeme totiž, aby si síť pamatovala dlouhé závislosti, zároveň
máme většinou k dispozici obrovské množství vzorků.

Síť LSTM by mohla být vhodná pro klasifikaci pózy zejména pokud bychom
potřebovali analyzovat pohyb v delších časových úsecích. Událost pádu se naopak
obvykle odehrává v krátkém čase, nicméně můžeme otestovat, jakých výsledků bude
tato architektura dosahovat oproti jiným rekurentním sítím, zejména v případě,
kdy bychom modelu nepředávali pouze pózu z $n$ posledních snímků, ale celou
sekvenci detekovanou pro danou osobu.

\section{GRU}

Gated Recurrent Unit (GRU) je novější typ rekurentní neuronové sítě, který
představil v roce 2014 Cho et al. \cite{gru}. Je postavený na principu brán,
podobném jako LSTM, nepotřebuje ale zvlášť stav pro dlouhodobou paměť. Místo
toho kombinuje krátkodobou a dlouhodobou paměť do skrytého stavu $h_t$.

\begin{figure}
    \centering
    \includegraphics[width=0.6\textwidth]{Figures/GRU_unit.pdf}
    \caption{Jednotka GRU}
    \label{fig:gru_unit}
\end{figure}

GRU obsahuje dvě brány: resetovací bránu (ang. reset gate) a aktualizační bránu
(ang. update gate), viz obrázek \ref{fig:gru_unit}. V resetovací bráně se
určuje, které informace z předchozího skrytého stavu $h_{t-1}$ budou mít vliv
na tvorbu kandidátního skrytého stavu. V aktualizační bráně vzniká nový skrytý
stav $h_t$ kombinací předchozího skrytého stavu a kandidátního skrytého stavu.
Ten je vytvořen pomocí neuronové vrstvy s tangenciální aktivační funkcí, do
které vstupuje výstup resetovací brány a aktuální vstup $x_t$. Pak se na
základě předchozího skrytého stavu $h_{t-1}$ a aktuálního vstupu $x_t$ určí,
které informace v novém skrytém stavu budou převzaty z předchozího skrytého
stavu a které z kandidátního skrytého stavu.

Hlavní výhodou GRU je jednoduchost. Oproti LSTM má méně parametrů a provádí
méně výpočtů. Je tak jednak rychlejší při evaluaci, jednak jednodušší pro
natrénování. Taky, u GRU sítí je menší pravděpodobnost přetrénování, což je
výhodné zejména v situacích, kdy máme omezený počet trénovacích dat. GRU sítě
se často využívají v úlohách, kde je důležité rychlé zpracování a efektivita,
např. v mobilních aplikacích a zpracování v reálném čase. Je to ale za cenu
trošku horšího zpracování komplexních a dlouhodobých závislostí. Oproti LSTM
nemá GRU takovou kontrolu nad tím, které informace dlouhodobě uchovávat a má
sklon k rychlejšímu zapomínání. Proto se až tak nehodí pro složitější úlohy a
situace, kdy je nutné si pamatovat velmi dlouhé časové závislosti. Nicméně jsou
dneska první volbou pro mnoho úloh, po LSTM sítích se pak sahá, až když si GRU
sítě s danou úlohou neporadí.

\endinput
% \part{Detekce osob a jejich pozice}
\chapter{Představení problematiky detekce pádu}
\label{chap:Goal}

V této kapitole je stanoven přesný cíl a navržen postup, jak k problému
přistoupit.

Úkolem bude v reálném čase z videostreamu detekovat pád osoby. Pád osoby
je definován jako náhle, neúmyslné klesnutí těla z výškové pozice (např. stání,
chůze nebo sezení) na zem nebo jinou nižší úroveň, přičemž tato osoba nemá
kontrolu nad tímto pohybem. Samozřejmě není vždy možné úplně dobře
rozeznat, zda se nejedná o úmyslné klesnutí, např. prudké lehnutí.

Dle některých definic (zejména ve zdravotnictví) se o pád nejedná, pokud jde o
důsledek závažné vnitřní příhody (např. mrtvice). Toto zde nebude rozlišováno,
naopak bude cílem detekovat jak pády v důsledku ztráty rovnováhy či vlivem
vnějších faktorů (např. zakopnutí, převrácení těžkým předmětem), tak pády v
důsledku akutních události vlivem zdravotních problémů, jako jsou např.
mrtvice, záchvaty, mdloby či jiné důvody ztráty vědomí.

\section{Návrh řešení}

Cílem této práce je navrhnout algoritmus, který bude detekovat, zda je ve
vstupní sekvenci snímku některá osoba, jejíž pozice je klasifikována jako pád.
Hlavním cílem výsledného programu bude alarmovat příslušného pracovníka, pokud
osoba upadne.

Alarmovat se bude až, pokud osoba zůstane v ležící pozici. To umožní
odfiltrovat falešné alarmy v případě sehnutí či pokud bude osoba špatně
viditelná a algoritmus tak na okamžik špatně vyhodnotí její pohyb. Tímto
postprocesingem se ale práce nebude zabývat, spíše bude pozornost věnována
samotné klasifikaci pozice.

Stejně jako u detekce objektů, viz \ref{sec:obj_det}, by bylo možné i pro
detekci pádu vytvořit vhodnou konvoluční síť, která by přímo z obrázku
definovala, zda se jedná o pád nebo ne. U detekce se už dnes sice s ohledem na
pokrok hardwaru tento přístup používá, nicméně se jedná o velmi náročný úkol,
který vyžaduje rozsáhlou optimalizaci, pokročilou architekturu a velké množství
trénovacích dat. Nicméně, pokud by se podařilo takovouto síť natrénovat, mohla
by lépe detekovat některé situace např. podle výrazu tváře.

V navrženém řešení tedy bude v prvním kroku pomocí vhodné předtrénované
neuronové sítě detekována pozice osoby ve formě klíčových bodů, tato část bude
nazývána \textit{detekční algoritmus}. Na základě těchto bodů pak další
neuronová síť vyhodnotí, zda se jedná o pád, tato část bude nazývána
\textit{klasifikační algoritmus}. To úlohu velice zjednoduší, jelikož místo
analyzování tisíců pixelů, bude algoritmus analyzovat pár desítek klíčových
bodů. Další výhodou je, že u takového postupu je možné použít techniky, kdy je
sledována změna pózy v čase, což by bylo mnohem složitější s jednofázovou
konvoluční síti.

Detekční algoritmus dostane na vstup celý snímek a může detekovat několik osob.
Klasifikační algoritmus ale bude zpracovávat každou osobu, resp. její pózu,
zvlášť.

Další alternativou by mohlo být pouze detekovat osoby jako objekty, a na
základě jejich bounding boxů určit, zda se jedná o pád. Tento postup by byl
jednodušší na dvou úrovních. Jednak je detekce objektů méně náročná úloha než
detekce pózy, jednak by ve druhé fázi bylo analyzováno pouze několik parametrů
bounding boxu (rozměry a velikost) oproti pár desítkám klíčových bodů. Nicméně,
pokud se nad tím zamyslíme, ne vždy vypovídají parametry bounding boxů o pozici
člověka. Tento postup by tak pravděpodobně vedl k mnohem méně přesnému
výsledku, než analýza klíčových bodů, kdy může síť analyzovat takové vzorce
jako je např. délka končetin v pohledu či úhel mezi nimi.

Nyní bude popsáno, jak a z jakými daty se bude pracovat, zejména při trénování,
v další kapitole pak bude rozebrána problematika detekce pózy a bude zvolen
algoritmus pro detekci klíčových bodů. Dále se bude práce věnovat vývoji modelu
detekujícího pád na základě těchto klíčových bodů.

\section{Trénovací datasety}
\label{sec:TrainingData}

Pro trénování vyvíjeného modelu jsme použili necelých 150 krátkých (1 až 15
sekund) videí ze dvou zdrojů. Prvním je dataset CAUCAFall vytvořený právě pro
práci s pády osob \cite{caucafall}. Tento dataset obsahuje 100 nahrávek
simulovaných pádu v různých světelných podmínkách, s různými osobami. Zahrnují
širokou škálu scénářů, jednak pro různé druhy pádů (v různých směrech či z
židle), jednak pro situace podobné pádu, jako je kleknutí či sehnutí se, jednak
běžné činnosti jako chůze či sednutí. Poměr videí s úpadkem a bez je 50/50.

Dalším zdrojem pro trénovací data je YouTube video tvůrce Kevina Parryho
\textit{50 Ways to Fall}. Ve videu autor simuluje pády v různých scénářích,
jako je zakopnutí, omdlení či poražení elektrickým proudem. Vzhledem k zabavné
povaze videa byly některé scénáře vypuštěny, nakonec jsme použili 45 videí. Zde
prakticky všechny videa obsahují pád, v některých se ale postava vrátí do
normálního stavu (např. kotrmelec).

\begin{figure}[]
    \centering
    \includegraphics[width=0.45\textwidth]{Figures/datasets_examples/cauca1.png}
    \includegraphics[width=0.45\textwidth]{Figures/datasets_examples/cauca2.png}
    \includegraphics[width=0.45\textwidth]{Figures/datasets_examples/fifty1.png}
    \includegraphics[width=0.45\textwidth]{Figures/datasets_examples/fifty2.png}
    \caption{Příkladové snímky z datasetů CAUCAFall (nahoře) a 50 Ways to Fall (dole).}
    \label{fig:datasets_examples}
\end{figure}

V obou případech se jedná o videa vždy jedné osoby. To proto, že použijeme
detekční algoritmus již natrénovaný na videích s více osobami, a náš
klasifikační algoritmus bude zpracovávat každou osobu zvlášť. Práce s více
osobami tak bude úlohou výsledného detektoru, nikoliv trénované klasifikační
sítě.

Videa z datasetu CAUCAFall jsou v rozlišení $720\times480$, zatímco videa z
datasetu 50 Ways to Fall jsou v rozlišení $1280\times720$.

Oba datasety byly rozděleny do tří sad: trénovací, validační a testovací.
Rozděleny byly v tomto poměru: trénovací sada 70\%, validační sada 15\% a
testovací sada 15\%. Trénovací a validační sada budou použity v procesu
trénování – trénovací pro výpočet ztráty a úpravu vah, validační pro průběžné
ověřování výkonu, testovací sada bude na konci použitá pro otestování jednak
klasifikačního algoritmu, jednak celkového řešení.

\section{Třídy a jejich anotace}
Pro videa byly vytvořeny anotace aktuální třídy pózy. Tato anotace není pro
každý snímek, ale pouze při změně definuje časovou značku a následující třídu.

V anotacích byly použity 3 třídy, ty odpovídají třem různým třídám pózy
relevantním k problému – \textit{normální}, kdy osoba např. chodí, sedí nebo
stojí, \textit{padá} – přechodný stav padání, definován od započatí pohybu
směrem dolů, a \textit{upadl} – definován od momentu, kdy se osoba dotkla země
trupem nebo všemi končetinami.

Pro řešený problém jsou obecně potřebné jenom dvě třídy – \textit{normální} a
\textit{upadl}. Skript tvořící trénovací data proto považuje třídu
\textit{padá} za třídu \textit{normální}. Nicméně by se mohlo do budoucna pro
pokročilejší optimalizaci zkusit experimentovat i se třídou \textit{padá},
která by mohla síti pomoct hlouběji pochopit problematiku a přesněji rozeznat
některé situace, zejména pak v případě využití rekurentních neuronových sítí.

\section{Příprava trénovacích dat pro klasifikační algoritmus}

Dále byl vytvořen skript, který prošel každé video z trénovacích datasetů a na
základě výše zmíněných anotací vytvořil trénovací data pro klasifikační
síť. Ty obsahují pro každý snímek detekované klíčové body (jako vstup) a
aktuální třídu (jako požadovaný výstup). Pro detekci klíčových bodů byl použit
vybraný model pro detekci pózy. Výběr modelu je popsán v následující kapitole.
Na modelu použitém při tvorbě dat by teoreticky nemuselo záležet (pokud detekuje stejné typy
klíčových bodů), je ale lepší použít ve výsledném programu stejný model jako
pro trénovací data. Modely se totiž můžou v některých situacích chovat trochu
jinak (např. okluze) a klasifikační algoritmus by tak dostával v praxi jiná data, než pro
jaké byl natrénován.

Jelikož pro rekurentní neuronové sítě potřebujeme sekvenci snímků, musí být
trénovací data ještě zpracována. To ale bude již součastí samotného trénovacího
skriptu, jelikož se konečná podoba dat může lišit délkou sekvencí.
\endinput
\chapter{Výběr algoritmu pro detekci klíčových bodů}
\label{chap:Pose}

Jelikož je dnes dostupných mnoho různých algoritmů či natrénovaných modelů pro
detekci pózy osob v obrázku či videu, nemá smysl pro navrhované řešení
implementovat takovýto algoritmus od nuly. Možné by to samozřejmě bylo, i
vzhledem k dostupnosti otevřených trénovacích dat (např. dataset COCO
\cite{coco}), nicméně by pravděpodobně nebylo dosaženo tak kvalitních výsledků,
jako řešení, která jsou výsledkem mnoholetých výzkumů. Hlavně pak by bylo těžko
dosáhnout výkonu těchto řešení, a ten je pro navrhované řešení stěžejní,
jelikož je potřeba video zpracovávat v reálném čase.

V následující kapitole budou popsány obecné principy detekce osob a jejich pózy
v obraze. Následně budou popsány některé populární algoritmy pro detekci pózy
se zaměřením na jejich specifika. Několik z nich pak bude otestováno, výsledky
budou porovnány, a na jejich základě bude zvolen algoritmus použitý v konečném
řešení detekce pádu.

\section{Detekce pózy}

\begin{figure}[]
    \centering
    \begin{minipage}{0.48\textwidth}
        \centering
        \includegraphics[width=0.5\textwidth]{Figures/keypoints.png}
    \end{minipage}
    \hfill
    \begin{minipage}{0.48\textwidth}
        \centering
        \includegraphics[width=0.35\textwidth]{Figures/pose1.png}
    \end{minipage}
    \caption{(Vlevo) Topologie klíčových bodů použitá např. v COCO-pose \cite{2dhpe} (Vpravo) Příklad detekce pózy pomocí YOLO.}
    \label{fig:keypoints}
\end{figure}

Úloha detekce pózy spočívá v nalezení klíčových bodů postavy v obraze.
% Může se
% jednat také o zvíře, v našem případě se ale budeme zabývat pouze klíčovými body
% lidské postavy. 
Klíčové body představují důležité body lidského těla, znalost jejich lokalizace
nám umožňuje analyzovat pózu dané osoby, popřípadě sledovat její pohyb. K
základním klíčovým bodům patří hlava, ramena, lokty, zápěstí, kyčle, kolena a
kotníky, viz obrázek \ref{fig:keypoints}. Některé algoritmy dokážou rozeznat i
orientaci dlaně či stopy, nebo rozpoznat klíčové body na hlavě, jako jsou ústa,
nos, oči a uši \cite{blazepose}.

Klíčové body jsou většinou reprezentovány jako dvojice souřadnic $(x, y)$
vzhledem k celému obrazu, některé algoritmy poskytují i souřadnice
normalizované vzhledem k bounding boxu osoby. Existují také algoritmy pro 3D
souřadnice, těmto ale nebude věnována pozornost, i když by mohly stanovit
zajímavou alternativu, zejména pokud by pro detekci bylo použito více kamer z
různých pohledů.

V oblasti algoritmů pro detekci pózy existují dva základní přístupy: zdola
nahoru a shora dolů. Přístup zdola nahoru se snaží detekovat všechny klíčové
body v obraze, aniž by rozlišoval jednotlivé osoby, pokud je algoritmus schopen
detekce pózy pro více osob, pak v dalším kroku tyto body spojuje do
jednotlivých postav. Naproti tomu přístup shora dolů nejprve detekuje všechny
osoby v obraze, v jejich rámci pak detekuje klíčové body.

%todo 
\section{Detekce klíčových bodů}

\subsubsection*{Heatmapy}

U obou výše zmíněných přístupů se nejčastěji provádí vyhledání všech klíčových
bodů pomocí tzv. heatmap. Je to 2D mapa pravděpodobnosti, že se v daném bodě
vyskytuje nějaký klíčový bod. Maximální hodnoty v této mapě pak představují
lokalizaci klíčových bodů.

Pro vygenerování heatmap se používá konvoluční neuronová síť. Pro každý klíčový
bod, resp. pro každý typ klíčového bodu k (v případě detekce pózy více osob)
vzniká jedna heatmapa. Jako referenční heatmapy pro trénování se používají
mapy, kde je klíčový bod reprezentován 2D Gaussovým rozložením s vrcholem v
místě daného bodu.

V dalším kroku jsou z heatmap vygenerovány, nejčastěji s pomocí algoritmu
argmax, souřadnice klíčových bodů. V případě vícero osob je pak třeba tyto body
spojit do jednotlivých osob.

\subsubsection*{Regrese}

Využití heatmap je velmi přesné, nicméně z důvodu nutnosti provádění dvou
sekvenčních výpočtů je trochu pomalé. Heatmapy také komplikují proces
trénování, jelikož je třeba spolu s trénovacími daty dodat modelu i heatmapy.
Některé algoritmy se proto snaží formulovat úlohu jako regresi vedoucí přímo k
souřadnicím klíčových bodů. Tento přístup je ve své podstatě trochu méně
přesný, nicméně je rychlejší.

Vůbec první algoritmus pro detekci pózy využívající hluboké učení, DeepPose
\cite{deeppose}, který byl vytvořen v roce 2014 společností Google, používal
právě regresi. Také algoritmus YOLO používá regresi pro určení souřadnic
klíčových bodů, nicméně detekce je prováděná pro detekované objekty, nikoliv
nad celým vstupním obrazem \cite{yolo-pose}.

% , a je součástí poupraveného modelu pro detekci objektů – v posledních
% vrstvách sítě je kromě regrese definující bounding box a klasifikace určující
% třídu prováděna regrese pro určení klíčových bodů.

\section{Detekce objektů a osob v obraze}
\label{sec:obj_det}

Detekce osob se v podstatě může generalizovat na detekci objektů v obraze.
Detekci objektů v obraze definujeme jako úlohu, kdy ve vstupním obrázku určíme
lokalizaci a třídu všech hledaných objektů. Lokalizace je většinou
reprezentována jako souřadnice obdélníku ohraničujícího daný objekt, tzv.
bounding box.

V kapitole \ref{chap:CNN} byla popsána základní architektura konvolučních
neuronových sítí, ta se ale většinou v praxi používá pro klasifikaci obrázků,
nikoliv pro detekci objektů – algoritmus tedy pouze určí, o jakou třídu objektu
se jedná, a ideálně potřebuje, aby objekt vyplňoval celý vstupní obraz.
Teoreticky by bylo možné detekci formulovat jako regresní problém a natrénovat
takovou síť, která by pomocí několika konvolučních vrstev následovaných
několika plně propojenými vrstvami byla schopna predikovat lokalizaci a třídu
všech objektů v obraze \cite{szegedy}. Problém detekce je ale velice komplexní
a také by vyžadoval velice komplexní síť – více vrstev s mnoha filtry, resp.
neurony. Jak již ale bylo zmiňováno, komplexnost sítě zvyšuje její nároky na
výpočetní výkon a komplikuje nebo úplně znemožňuje její trénování s ohledem na
pravděpodobnost přetrénování.

Snahou tedy bylo najít metody, které poupraví architekturu sítě tak, aby byla
schopna efektivní detekce objektů. Většina těchto metod se nějakým způsobem
snaží rozdělit vstupní obrázek na menší části, ty následně jednak klasifikovat,
a tedy určit, zda se v dané lokalitě vyskytuje objekt, popřípadě pomocí regrese
určit jeho přesnou lokalizaci. Rozdělení může být provedeno přímo na vstupním
obrázku nebo na mapě příznaků v rámci sítě.

Metoda sliding window (klouzavé okno), která aplikuje hrubou sílu a projde
veškeré možné oblasti, je samozřejmě velice neefektivní, a tak se další metody
snaží buď najít pouze oblasti, ve kterých je pravděpodobné, že se nějaký objekt
nachází – dvoufázový přístup, anebo rozdělí obrázek do mřížky – jednofázový
přístup.

Jelikož jsou tyto metody základem pro většinu detektorů klíčových bodů, bude
nyní několik základních popsáno.

% \subsection{Sliding window}

% Jednou z prvních takových metod byl tzv. sliding window (klouzavé okno), který
% aplikuje hrubou sílu. Vstupní obrázek se postupně projíždí oknem o fixní
% velikosti. Vznikne tak množina pokrývající každou možnou lokaci objektů. Na
% tyto oblasti se pak aplikuje klasifikační algoritmus. Postup se opakuje pro
% několik velikostí okna, aby se detekovaly objekty různé velikosti.

% Tento postup je ale velice pomalý, jelikož je pro každý obrázek zvolený velký
% počet oblastí, pro které je třeba provést klasifikaci popřípadě regresi. Navíc
% je většina těchto oblastí prázdná, a dochází tak k plýtvání výpočetním výkonem.
% Algoritmus se také potýká s překrývajícími se objekty.

% Další metody se tedy snaží redukovat počet oblastí, na které se aplikuje
% klasifikace, tak, že se vybere pouze oblasti, které pravděpodobně budou
% obsahovat nějaký objekt.

\subsection{Dvoufázový přístup}

\subsubsection*{R-CNN}
Prvním algoritmem, který efektivně zredukoval počet oblastí pro klasifikaci,
byl algoritmus R-CNN (Region-based Convolutional Network) \cite{r-cnn}. Tento
algoritmus nejprve použil některou z dostupných metod (autoři použili selective
search) pro vygenerování navržených oblastí (region proposals), které
pravděpodobně obsahují nějaký objekt. Tyto metody jsou nezávislé na třídě
objektů. Algoritmus tedy vygeneruje zhruba 2000 oblastí, vzniklé obrázky jsou
následně upraveny na velikost požadovanou CNN v další fázi. CNN extrahuje z
dané oblasti mapu příznaků, na jejímž základě plně propojené vrstvy predikují
třídu objektu popřípadě jeho bounding box.

Problémem R-CNN je, že výběr oblasti a jejich následná klasifikace jsou
nezávislé úlohy a jsou nezávisle trénovány. Detekce objektu je také poměrně
pomalá, protože je extrakce příznaků prováděna pro všechny oblasti zvlášť. Tyto
problémy se snaží řešit další upravené verze R-CNN.

\subsubsection*{Fast R-CNN}
První z nich je Fast R-CNN \cite{fast-r-cnn}, která je upravená tak, aby bylo
možné provádět trénování v jednom kroku. Také extrahuje příznaky pro celý
vstupní obraz najednou, pomocí selective search pak identifikuje oblasti zájmu
(ang. region of interest – RoI), které následně použije pro klasifikaci a
regresi. Tato metoda je přesnější a asi desetkrát rychlejší než původní R-CNN.

\subsubsection*{Faster R-CNN}
Další algoritmus, Faster R-CNN \cite{faster-r-cnn}, nahrazuje metodu selective
search vlastní, plně konvoluční sítí RPN (region proposal network).
Zefektivňuje tak proces trénování, výsledná síť je také rychlejší a přesnější
než Fast R-CNN.

\subsection{Jednofázový přístup}

Jednofázový přístup se snaží najít řešení, ve kterém není nutné hledat navržené
oblasti, ale provést klasifikaci a regresi na předem dané množině oblastí,
obvykle určené mřížkou.

\subsubsection*{YOLO}
Prvním takovým algoritmem byl YOLO (také YOLOv1, z ang. you only look once)
\cite{yolo}. Ten, v původní verzi, rozdělí vstupní obraz do pevně dané mřížky
velikosti $S \times S$ a v každém z těchto polí určí $B$ bounding boxů a jejich
třídu. V původní verzi bylo zvoleno $S = 7$ a $B = 2$.

Obraz je nejprve zpracován pomocí konvolučních vrstev, které extrahují mapu
znaků o velikosti $S \times S \times K$, kde $K$ je počet kanálů. Každý pixel
této mapy představuje jedno pole mřížky. Dále je mapa zpracována plně
propojenými vrstvami, které provádějí nad každým polem mřížky klasifikaci a
regresi, viz obrázek \ref{fig:yolo}.

\begin{figure}[]
    \centering
    \includegraphics[width=0.8\textwidth]{Figures/yolo}
    \caption{Architektura původní verze YOLO \cite{yolo}}
    \label{fig:yolo}
\end{figure}

Každý bounding box je reprezentován souřadnicemi středu a velikosti (šířka a
výška). Dohromady s informací o jistotě detekce bounding boxu (confidence
score) vrátí model $5$ informací o každém bounding boxu. Pro každé pole mřížky
pak určí společnou informaci o třídě všech objektů v daném poli. Pokud objekt
není detekován, třída indikuje pozadí a souřadnice bounding boxu jsou
ignorovány. Velikost výstupního vektoru je tedy $7 \times 7 \times (2 * 5 +
    C)$, kde $C$ je počet definovaných tříd – v původní verzi pouze 20.

Tento algoritmus, navržený v roce 2015 J. Redmonem et al., byl revolučně
rychlý, zároveň v porovnání s jinými real-time detekčními algoritmy dosahoval i
slušné přesnosti. Nicméně byl velice citlivý na velikost objektu a přesnost
detekce, zejména u menších objektů, byla horší než u dvoufázových algoritmů.

Další verze algoritmu YOLO přinesly postupná vylepšení ve formě optimalizace
trénování a architektury. YOLOv2 \cite{yolo9000} zavedl mj. trénování na
několika měřítkách a byl natrénován s 9000 třídami (proto také nazýván
YOLO9000). YOLOv3 \cite{yolov3} přinesl mj. detekci na několika měřítkách.
Postupně byla také zvětšována mřížka a měnila se použitá architektura CNN sítě
sloužící pro extrakci příznaků pro jednotlivá pole mřížky. Postupně také byly
přidávány další funkce jako je segmentace, detekce pózy či sledování objektů
(ang. tracking).
%todo cite

V 2020 roce firma Ultralytics poprvé implementovala YOLO s využitím populární
knihovny PyTorch (YOLOv5), což umožnilo snadnější využití YOLO v praxi. Firma
Ultralytics také vytvořila framework pro použití různých verzí YOLO (YOLOv3 a
novější). Také pracuje na dalších vylepšeních a optimalizacích. Konkrétně
vytvořila YOLOv5 (2020), YOLOv8 (2023) a YOLOv11 (2024). Tyto verze nicméně
nejsou podloženy odbornými články, někteří je tak považují za neoficiální
verze.

\subsubsection*{SSD}
Dalším populárním algoritmem, který používá jednofázový přístup, je SSD (z ang.
single shot detector) \cite{szegedy:ssd}. Ten rozdělí vstupní obraz do několika
mřížek o různé velikosti. Postup je takový, že nejprve obraz projde konvoluční
sítí, konkrétně sítí VGG16, která extrahuje mapu příznaků. Tu se postupně
dalšími konvolučními vrstvami zmenšuje, výstup každého stádia zmenšení,
reprezentující mřížku dané velikosti, se spolu s původní mapou dále zpracovává
plně propojenou sítí.
\begin{figure}[]
    \centering
    \includegraphics[width=0.8\textwidth]{Figures/ssd.png}
    \caption{Architektura SSD \cite{szegedy:ssd}}
    \label{fig:ssd}
\end{figure}

Výstupní bounding boxy nejsou, jako v případě YOLO, pouze výsledkem regrese,
ale pro každé pole dané mřížky je definováno několik výchozích oblastí (ang.
default box), ze kterých jsou vybrány ty, které obsahují objekt. K nim je
predikována třída objektu a posun i změna velikosti výchozí oblasti,
upřesňující výsledný bounding box.

V době svého vzniku byl SSD rychlejší a přesnější než YOLO, ale novější verze
YOLO jej už předběhly. Nicméně některé principy SSD, jako výchozí oblasti či
použití různých měřítek, byly převzaty do novějších verzí YOLO.
%todo cite

\section{Charakteristiky vybraných implementací pro detekci pózy}

V této sekci bude popsáno několik populárních algoritmů (a jejich implementací)
pro detekci pózy, zejména se zaměřením na jejich rychlost, přesnost a specifika
architektury.

Velká část algoritmů byla zamítnuta a neproběhlo ani jejich testování.
Nejčastějším důvodem zamítnutí některého z algoritmů je, že schází jeho volně
dostupná, aktualizovaná implementace. Tvůrci algoritmů většinou investují čas a
zdroje pro vývoj algoritmu a natrénování modelu (často pro akademické účely),
pak ale neinvestují do jeho údržby. Zejména v prostředí Pythonu, kde se
neustále vyvíjejí nové knihovny a zpětná kompatibilita starších verzí není
zaručena, je pak obtížné použít takovéto řešení ve svém projektu, aniž by bylo
nutné investovat další čas do pochopení zdrojového kódu a jeho úpravy. Jak již
bylo zmíněno na počátku kapitoly, implementace algoritmu od nuly by vyžadovala
velké množství času a zdrojů (zejména výpočetních), hlavně by také byla
potřebná hlubší znalost problematiky. Někdy je možné řešení použít za cenu
kompromisu ve formě použití starších verzí knihoven či Pythonu, může to ale
představovat bezpečnostní rizika. Také některé algoritmy jsou sice volně
dostupné, ale jejich implementace jsou součástí komerčních produktů, např.
OpenPose, jež je mj. součástí produktu Viso Suite od viso.ai
\cite{visoai_openpose}.

\subsection{DeepPose}

DeepPose je historicky první algoritmus pro detekci pózy využívající hluboké
učení. Vyvinuli jej Alexander Toshev a Christian Szegedy ze společnosti Google
v roce 2014 \cite{deeppose}. Algoritmus předpokládá, že se ve vstupním obraze
nachází pouze jedna osoba. Síť se snaží v jednom kroku pomocí regrese jak
detekovat osobu, tak i její klíčové body. Jelikož je těžké takto dosáhnout
velmi přesných výsledků, algoritmus používá další fázi, která pomocí regrese
provádí posun bodů k přesnějším výsledkům. Tato fáze je aplikována opakovaně,
kaskádně se tak zvyšuje přesnost detekce.

Při svém vzniku byl DeepPose revoluční, nicméně v porovnání s dnešními řešeními
je poměrně pomalý a nepřesný. Nicméně položil základ pro využití hlubokého
učení v oblasti detekce pózy.

\subsection{OpenPose}

OpenPose \cite{openpose} je typicky příklad přístupu zdola nahoru. Jeho výhodou
je ale možnost vyhledání více osob v jednom snímku. Tento algoritmus, který
vyvinuli v roce 2019 Zhe Cao et al., nejprve pomocí CNN vytvoří heatmapu pro
každý typ klíčového bodu. Pro spojení bodů do jednotlivých osob využije pole
propojení klíčových bodů (ang. part affinity field – PAF). PAF je mapa
vytvořená pro každou končetinu (myšleno obecně spojení dvou klíčových bodů),
která v oblasti dané končetiny obsahuje hodnoty určující směr z jednoho bodu do
druhého. Pokud jsou pak spojené informace z heatmap a z PAF, je možné poměrně
jednoznačně zkompletovat jednotlivé klíčové body do celých postav. Stejně jako
heatmapy, jsou i PAF součástí trénovacích dat.

\subsection{OpenPifPaf}

Algoritmus OpenPifPaf \cite{openpifpaf}, vyvinutý v roce 2021 Svenem Kreissem
et al., je v podstatě vylepšenou verzí OpenPose. Jeho název je odvozen od dvou
stavebních kamenů: PIF (Part Intensity Field) – pole intenzity klíčových bodů,
a PAF (Part Affinity Field) – pole propojení klíčových bodů. PIF je rozšířením
heatmap, kdy kromě intenzity pravděpodobnosti klíčového bodu obsahuje i jeho
posun, zaručující přesnější lokalizaci bodu, a odhadovanou velikost dané části
těla. PAF v OpenPifPaf je také podobný tomu v OpenPose, navíc ale indikuje
kromě směru i velikost dané končetiny, což umožňuje lepší prostorové zachycení
pózy.

Dalším rozšířením oproti OpenPose je možnost sledování osob ve videu. Mapa
příznaků, která je výsledkem vstupní CNN, je udržována v mezipaměti, do další
části sítě pak vždy vstupují mapy pro aktuální a předchozí snímek. Výstupem pak
kromě klíčových bodů v každém snímku a jejich propojení tvořící kostru, jsou i
propojení mezi klíčovými body z jednotlivých snímků, viz
\ref{fig:pipaf-tracking}. Algoritmus si pak udržuje ID sledovaných osob, pokud
je k dříve nalezené osobě nalezena nová pozice, je jí přiřazeno stejné ID.
Pokud je nalezena nová osoba, je jí přiřazeno nové ID.

\begin{figure}[]
    \centering
    \includegraphics[height=0.2\textheight]{Figures/skeleton_forward2.png}
    \caption{Vizualizace  sledování osoby mezi dvěma snímky v OpenPifPaf \cite{openpifpaf}}
    \label{fig:pipaf-tracking}
\end{figure}

U OpenPifPaf je k dispozici výběr několika páteřních modelů, jako je ResNet50
či ShuffleNet v různých variantách a velikostech. Je tak možné zvolit model,
který je kompromisem mezi výkonem a přesností, v závislosti na konkrétních
požadavcích aplikace.

\subsection{MediaPipe – BlazePose}

MediaPipe je framework vyvinutý společností Google, umožňující jednoduchou
integraci různých technik strojového učení. Obsahuje různé algoritmy pro řešení
úloh jako detekce objektů, segmentace či detekce klíčových bodů (tváře či
pózy). MediaPipe je optimalizován pro mobilní zařízení a webové aplikace.
Detekce pózy v tomto frameworku je postavená na algoritmu BlazePose.

BlazePose implementuje přístup shora dolů, detekuje tedy nejprve RoI, ve
kterých detekuje osobu a její pózu. Nativně podporuje pouze jednu osobu ve
snímku. Ve videu ale v rámci optimalizace neprovádí detekci RoI pro každý
snímek, pouze pokud v aktuální RoI již není detekována osoba. Výhodou tohoto
algoritmu je, že detekuje 33 bodů v postavě, což je podstatně více, než většina
ostatních algoritmů, umožňuje tak přesnější analýzu některých situací, např.
podle natočení tváře, dlaní či stop.

Framework MediaPipe implementuje BlazePose spolu s detekcí více osob (v
první fázi používá detektor objektů). Výhodou tohoto frameworku je jeho kontinuální
vývoj a jednoduchost integrace. Nevýhodou ale je, že pro systémy Windows není
implementována podpora GPU. Jelikož výsledný produkt bude spouštěn primárně na
Windows zařízeních, je tato vlastnost rozhodující. Model je dostupný ve třech
velikostních variantách: $Lite$, $Full$ a $Heavy$.

\subsection{YOLO}

Od vydání YOLOv7 v roce 2022 integruje framework YOLO i detekci pózy. Oficiální
článek Chien-Yao Wanga et al. \cite{yolov7} sice neobsahoval tuto funkčnost,
ale oficiální implementace zahrnula i implementaci YOLO-Pose \cite{yolo-pose}.
Obecně detekce pózy v YOLO kombinuje přístup shora dolů a zdola nahoru.
Algoritmus sice vyhledává klíčové body spolu s bounding boxy osob, nicméně vše
v jednom kroku. Samotná detekce klíčových bodů využívá regresi, což
zjednodušuje proces trénování, jelikož není třeba tvořit heatmapy.

Architektura použitá v YOLO-Pose se ale liší od architektury používané v
pozdějších verzích. V YOLO-Pose jsou na konci řetězce umístěny hlavy pro různá
měřítka, jejich výstupem jsou bounding boxy a klíčové body, oba tvořené spolu.
V pozdějších verzích je architektura YOLO koncipována univerzálněji pro různé
úlohy. Obsahuje tak tři fáze\cite{yolov11}: páteř (ang. backbone), která
extrahuje mapu příznaků, krk (ang. neck), který přizpůsobuje mapu příznaků pro
různá měřítka, a hlavy (ang. head), které paralelně zpracovávají výstupy pro
různé úlohy, viz obrázek \ref{fig:yolov11} . Bounding box a klíčové body jsou
tedy sice generovány paralelně a teoreticky nezávisle, nicméně s ohledem na
proces trénování a postprocessing se v praxi navzájem výrazně ovlivňují.

\begin{figure}[]
    \centering
    \includegraphics[height=0.2\textheight]{Figures/yolo_v11.png}
    \caption{Architektura YOLOv11 \cite{yolov11}}
    \label{fig:yolov11}
\end{figure}

Nejnovější verze YOLO také podporují kombinaci detekce klíčových bodů a
sledování osob. Model tedy kromě klíčových bodů vrací ID dané osoby, pomocí
kterého lze spojit danou postavu s předchozími snímky. Je tak možné efektivně
analyzovat pohyby jednotlivých osob.

Jednou z výhod framevorku YOLO, zejména verzí vyvíjených firmou Ultralytics, je
široká škála velikostí modelů. Každý model je dostupný v pěti variantách:
$Nano$, $Small$, $Medium$, $Large$, $Xlarge$.

\subsection{Torchvision}

Torchvision je knihovna, která je součástí frameworku PyTorch. Obsahuje různé
nástroje pro strojové vidění, jako je detekce objektů či segmentace. Její
součástí je i předtrénovaný model pro detekci pózy, který implementuje
algoritmus Keypoint R-CNN \cite{keypoint-rcnn}.

Algoritmus Keypoint R-CNN je založen na stejné myšlence jako Mask R-CNN
\cite{mask-r-cnn}, a tedy rozšíření Faster R-CNN o další hlavu, která v případě
Mask R-CNN provádí segmentaci, v případě Keypoint R-CNN detekuje klíčové body
\cite{keypoint-rcnn}. V algoritmu je také upravená pooling vrstva, která
zajišťuje, že se výstupy algoritmu shodují se vstupy s přesností pixelu.

Tato implementace bude testována zejména z důvodu její jednoduchosti použití a
integrace do frameworku PyTorch, který bude použitý i v další části řešení. Na
druhou stranu je oproti jiným frameworkům, jako je YOLO či MediaPipe, k
dispozici pouze jeden model, nikoliv více velikostních variant.

\section{Testování a porovnání vybraných algoritmů pro detekci pózy}

Pro testování byly vybrány čtyři algoritmy pro detekci pózy, zejména na základě
jednoduchosti jejich implementace, aktualizované podpory a požadovaných funkcí.
Testovány tedy budou algoritmy Torchvision, OpenPifPaf, MediaPipe BlazePose a
YOLO v nejnovější verzi 11. Testy byly provedeny na 25 videích ze stejného
datasetu jako byl použit později pro trénování algoritmu pro analýzu klíčových
bodů. Testování probíhalo na počítači s procesorem Intel Core i7, 32 GB RAM a
grafickou kartou NVIDIA GeForce RTX 3080 s 10 GB VRAM.

Algoritmy Torchvision a OpenPifPaf byly testovány s použitím GPU, MediaPipe
pouze s využitím CPU, jelikož v prostředí Windows nemá podporu GPU. Algoritmus
YOLO byl testován na GPU a CPU, pro porovnání jeho výkonu v různých podmínkách.
Algoritmus YOLO byl také vyzkoušen na GPU s využitím funkce sledování.

Výstupem testování pro daný algoritmus a jeho variantu je průměrná doba
zpracování jednoho snímku a videa s vykreslením klíčových bodů. Tato videa
dovolují ověřit schopnost detekce klíčových bodů v různých situacích a její
přesnost.

Následující část se věnuje výsledkům testování jednotlivých algoritmů. Pro
každý algoritmus budou porovnány jednotlivé varianty s ohledem na rychlost a
přesnost a budou vyhodnoceny jeho výhody či nevýhody oproti ostatním
algoritmům. Nakonec bude zvolen model, který bude použitý v další části vývoje.

\subsection{Výkonové požadavky}
\label{sec:performance_requirements}

Při výběru algoritmu je třeba s ohledem na práci v reálném čase brát v úvahu
zejména výkon detekčního algoritmu. Bezpečnostní kamery mají obvykle snímkovou
frekvenci od 15 do 30 snímků za sekundu (FPS), ideální by tedy bylo, aby
konečný program byl schopen pracovat s frekvencí alespoň 30 FPS. Zároveň se
předpokládá, že v prostředí, kde bude program nasazen, bude dostupná grafická
karta.

V této fázi již bylo zkoušeno trénování neuronové sítě pro klasifikaci pózy, a
bylo ověřeno, že i v případě hlubších a komplexnějších sítí je dosaženo doby
inference v řádu nižších jednotek milisekund. Hlavní vliv na výslednou rychlost
programu tedy bude mít hlavně detekční algoritmus, který musí zpracovávat
mnohem větší objem dat – stovky tisíc až miliony pixelů oproti např. 17
klíčovým bodům v případě klasifikačního algoritmu.

\subsection{OpenPifPaf}

Algoritmus OpenPifPaf byl zkoušen v několika variantách, postavených na síti
\textit{ResNet 50} a \textit{ShuffleNet V2} \cite{shufflenetv2}. V tabulce
\ref{tab:openpifpaf_performance} je vidět, že většina variant ani zdaleka
nedosahuje požadovaného výkonu.

Tento algoritmus je poměrně robustní z pohledu světelných podmínek a rozlišení
či rozmazaní obrazu, jinak ale dosahuje nejhorší přesnosti ze všech testovaných
algoritmů. Kromě nedetekování části těla, které nejsou vidět (jsou např.
schovány za jinou částí těla), totiž často nedetekuje člověka vůbec, nejčastěji
pak když člověk padá nebo leží, což jsou situace pro nás stěžejní. Hlavně tento
problém vystupuje ve variantě $resnet50$ a $shufflenetv2k16$, jsou tak pro
řešení nepoužitelné. Varianty $shufflenetv2k30$ a $tshufflenetv2k30$ by sice s
ohledem na kvalitu výsledků použitelné byly, nicméně je jejich výkon příliš
nízký.

\begin{table}[htbp]
    \centering
    \caption{Porovnání výkonu modelu OpenPifPaf}
    \label{tab:openpifpaf_performance}
    \begin{tabular}{|l|l|l|l|}
        \hline
        \textbf{Verze}   & \textbf{inference [ms]} & \textbf{Frekvence [FPS]} \\
        resnet50         & 49,2                    & 20.324                   \\ \hline
        shufflenetv2k16  & 31,3                    & 31.910                   \\ \hline
        shufflenetv2k30  & 58,6                    & 17.070                   \\ \hline
        tshufflenetv2k30 & 70,0                    & 14.281                   \\ \hline
    \end{tabular}
\end{table}

\subsection{MediaPipe BlazePose}

Algoritmus BlazePose z knihovny MediaPipe byl testován ve třech variantách:
$Lite$, $Full$ a $Heavy$. Ve všech variantách tento algoritmus dosahoval velmi
přesných výsledků, asi nejlepších ze všech testovaných algoritmů. Na rozdíl od
jiných algoritmů totiž, pokud detekoval osobu, vždy velmi přesně označil
všechny její klíčové body, v rámci možností i ty, které byly hůře viditelné
(např. schovány za jinou částí těla). Na obrázku \ref{fig:ym_comparison} je
vidět rozdíl v přesnosti detekce klíčových bodů mezi nejmenší variantou
BlazePose a druhou největší variantou YOLO, kdy BlazePose dosahuje mnohem větší
přesnosti, v tomto příkladě zejména co se týče detekce nohou.

\begin{figure}
    \centering
    \includegraphics[width=0.4\textwidth]{Figures/pose_tests/mh1.png}
    \includegraphics[width=0.4\textwidth]{Figures/pose_tests/yl1.png}
    \caption{Porovnání přesnosti \textit{MediaPipe Lite} (vlevo) a \textit{YOLO Large} (vpravo)}
    \label{fig:ym_comparison}
\end{figure}

S ohledem na to, že nemáme k dispozici grafickou akceleraci, je jeho výkon
velmi dobrý, viz tabulka \ref{tab:mediapipe_performance}. Verze $Lite$ a $Full$
by tak mohla být v našem řešení použitelná, což by nám také dávalo možnost
nasazovat výsledný program v mobilních zařízeních či jiných systémech bez
grafické karty.

\begin{table}[htbp]
    \centering
    \caption{Porovnání výkonu modelu MediaPipe BlazePose}
    \label{tab:mediapipe_performance}
    \begin{tabular}{|l|l|l|l|}
        \hline
        \textbf{Verze} & \textbf{inference [ms]} & \textbf{Frekvence [FPS]} \\
        \hline
        lite           & 25.5                    & 39.239                   \\ \hline
        full           & 30.9                    & 32.405                   \\ \hline
        heavy          & 68.2                    & 14.655                   \\ \hline
    \end{tabular}
\end{table}

Algoritmus si ale velice špatně radí s horšími světelnými podmínkami či menším
rozlišením obrazu. Ve většině případů sice detekuje klíčové body velmi přesně,
pokud je ale osoba hůř viditelná, nedetekuje ji vůbec. Tento problém se
projevuje ve všech variantách podobně. Jelikož bude výsledný program nasazován
spíše právě v podmínkách s horším osvětlením a ve větší vzdálenosti od osob,
pravděpodobně se pro aplikaci nebude hodit.

\subsection{Torchvision Keypoint R-CNN}

Torchvision Keypoint R-CNN nedosáhla ani dostatečného výkonu, viz tabulka
\ref{tab:torchvision_performance}, ani kvalitních výsledků. Podobně jako
\textit{OpenPifPaf} má totiž problém, když osoba padá anebo leží. V tomto
případě osobu často detekuje, ale naprosto ztrácí přesnost detekovaných
klíčových bodů, nejčastěji záměnou jednotlivých bodů, viz obrázek
\ref{fig:torchvision_bad}. Zároveň oproti BlazePose nemá takový problém s
horšími světelnými podmínkami a menším rozlišením obrazu.

\begin{figure}[]
    \centering
    \includegraphics[width=0.2\textwidth]{Figures/pose_tests/torchvision_bad.png}
    \caption{Příklad špatné detekce bodů v modelu Torchvision Keypoint R-CNN}
    \label{fig:torchvision_bad}
\end{figure}

\begin{table}[htbp]
    \centering
    \caption{Výkon modelu Torchvision Keypoint R-CNN}
    \label{tab:torchvision_performance}
    \begin{tabular}{|l|l|}
        \hline
        \textbf{inference [ms]} & \textbf{Frekvence [FPS]} \\
        \hline
        50.3                    & 19.897                   \\ \hline
    \end{tabular}
\end{table}

\subsection{YOLO}

Algoritmus YOLO ve verzi 11 byl testován v pěti variantách: $Nano$, $Small$,
$Medium$, $Large$ a $Xlarge$. Všechny varianty byly testovány na GPU i CPU.

V tabulce \ref{tab:yolo_performance} je vidět, že na CPU dosahuje tento
algoritmus poměrně špatných výsledků. Jediný použitelný by mohl být v takové
situaci model $Nano$. YOLO je ale velice kvalitně optimalizováno pro grafické
karty, lze tak pozorovat, že na GPU je výkon výrazně vyšší. Pro potřeby řešení
by tak byly použitelné prakticky všechny varianty.

Jelikož pro analýzu více osob v jednom snímku je potřeba, zejména v případě
použití rekurentní neuronové sítě, jednotlivé osoby od sebe oddělit a
identifikovat i mezi snímky, bude velmi užitečná funkce sledování objektu.
Proto byl otestován algoritmus YOLO i s touto funkcí. Jak je vidět v tabulce
\ref{tab:yolo_performance}, je výkon sice horší než bez sledování, pořád ale
tři menší varianty dosahují frekvence větší než $30$ FPS.

\begin{table}[htbp]
    \centering
    \caption{Porovnání výkonu modelu YOLO}
    \label{tab:yolo_performance}
    \begin{tabular}{|c|l|l|l|}
        \hline
                                           & \textbf{Verze} & \textbf{inference [ms]} & \textbf{Frekvence [FPS]} \\
        \hline\hline
        \multirow{3}{*}{CPU}               & nano           & 32.5                    & 30.749                   \\ \cline{2-4}
                                           & small          & 53.9                    & 18.563                   \\ \cline{2-4}
                                           & medium         & 114.1                   & 8.763                    \\ \cline{2-4}
                                           & large          & 143.4                   & 6.973                    \\ \cline{2-4}
                                           & xlarge         & 833.2                   & 1.200                    \\ \hline\hline
        \multirow{3}{*}{GPU}               & nano           & 15.1                    & 66.323                   \\ \cline{2-4}
                                           & small          & 15.2                    & 65.972                   \\ \cline{2-4}
                                           & medium         & 17.4                    & 57.500                   \\ \cline{2-4}
                                           & large          & 24.4                    & 41.026                   \\ \cline{2-4}
                                           & xlarge         & 24.4                    & 41.005                   \\ \hline\hline
        \multirow{3}{*}{GPU se sledováním} & nano           & 27.4                    & 36.500                   \\ \cline{2-4}
                                           & small          & 27.7                    & 36.148                   \\ \cline{2-4}
                                           & medium         & 29.5                    & 33.882                   \\ \cline{2-4}
                                           & large          & 37.5                    & 26.664                   \\ \cline{2-4}
                                           & xlarge         & 40.5                    & 24.695                   \\ \hline
    \end{tabular}
\end{table}

Všechny varianty algoritmu YOLO dosahují poměrně kvalitních výsledků. I v
horších světelných podmínkách vždy detekují osobu, a víceméně přesně určí její
klíčové body. Obecně je ale vidět, že je model trochu méně robustní (než např.
MediaPipe) v situacích, kdy není dobře vidět některá končetina – je např.
schovaná za jinou částí těla, anebo ve specifických pózách – např. když je
osoba v dřepu nebo je v obraze natočená vzhůru nohama. V takovýchto případech
dosahuje menší přesnosti pro jednotlivé body – detekuje jiné natočení končetiny
nebo v extrémních případech špatně vyhodnotí natočení celé postavy. Špatně
viditelné části těla pak často vůbec nedetekuje.

Z pohledu přesnosti je zde výrazně vidět vliv velikosti modelu na přesnost
detekce. Varianta $Nano$ ve výše zmíněných situacích někdy detekuje body zcela
špatně, a je tak prakticky nepoužitelná. Varianta $Small$ je znatelně lepší,
pořád ale v horších podmínkách vyhodnocuje mnoho části těla špatně – např.
zamění nohy. Varianta $Medium$ je už výrazně lepší. Není sice ideální, ve valné
většině ale vyhodnotí všechny části těla správně i když ne z přesností několika
pixelů. Varianty $Large$ a $Xlarge$ jsou pak už velmi přesné, projevuje se to
ale znatelně menší rychlostí.

\begin{figure}
    \centering
    \includegraphics[width=0.18\textwidth]{Figures/pose_tests/y_n.png}
    \includegraphics[width=0.18\textwidth]{Figures/pose_tests/y_s.png}
    \includegraphics[width=0.18\textwidth]{Figures/pose_tests/y_m.png}
    \includegraphics[width=0.18\textwidth]{Figures/pose_tests/y_l.png}
    \includegraphics[width=0.18\textwidth]{Figures/pose_tests/y_x.png}
    \caption{Porovnání přesnosti variant modelu YOLO v situaci s netypickým natočením postavy. Zleva: $Nano$, $Small$, $Medium$, $Large$, $Xlarge$}
    \label{fig:y_comparison}
\end{figure}

Je zajímavé pozorovat, že i když je algoritmus BlazePose v mnoha situacích
mnohém přesnější, než i větší varianty YOLO, viz obrázek
\ref{fig:ym_comparison}, při horší viditelnosti, kdy i nejmenší varianty YOLO
detekují osobu, BlazePose zcela selhává. Ve snímku na obrázku
\ref{fig:y_comparison} například nedetekoval BlazePose osobu vůbec.

\subsection{Shrnutí a výběr modelu}

Z výše popsaného testování bylo zjištěno, že modely OpenPifPaf a Torchvision
jsou natrénované spíše pro detekci postavy, když je osoba v běžnějších pózách,
jako je ve stoje či chůze. Stejně i u menších variant YOLO přesnost prudce
klesá v méně typických pózách či natočeních v obraze. Jelikož ale je navrhované řešení zaměřeno právě na detekování spíše netypické postavy, je pro nás důležité
detekovat pozici ve všech situacích.

Naopak algoritmus BlazePose je velmi přesný, strádá ale při horší viditelnosti
osoby. Nehodí se tak pro naše využití s kamerami s horším rozlišením a vysokou
vzdáleností od osob. Ostatní algoritmy jsou v tomto ohledu mnohem robustnější,
nejlépe si s horšími podmínkami poradí algoritmus YOLO.

Optimální cesta se tedy zdá být algoritmus YOLO ve verzi $Medium$, který dosahuje
dostatečné rychlosti i při sledování osob, zároveň dostatečně přesně detekuje
pózy ve všech pozicích i podmínkách. Pokud by nebyla pro analýzu pózy použitá
rekurentní neuronová síť, nemuselo by být nutné používat funkci sledování a
bylo by tak možné použít i větší variantu YOLO, což by mohlo zlepšit přesnost
detekce.

\endinput

\chapter{Implementace klasifikační neuronové sítě}
\label{chap:ClassificationImplementation}

Problematika vyhodnocování je velmi široká a přináší mnoho problémů. Ostatně i
člověk někdy může špatně interpretovat chování druhé osoby. Například pokud
někdo skáče do postele či jinak prudce lehá, může to vypadat jako nebezpečná
situace. Stejně se v počítačovém vidění nevyhneme falešným poplachům, nicméně
se budeme snažit zapojit různé techniky pro zlepšení přesnosti našeho
detektoru.

Naším úkolem je nyní vytvořit algoritmus, který pro danou pózu (reprezentovanou
klíčovými body), resp. sekvenci takových póz (získané pro jednu osobu ze
sekvence snímků), určí, zda se jedná o situaci pádu, či nikoliv. Tento
algoritmus bude přijímat vždy pózu jedné osoby, funkcionalita pro více osob
bude řešená později.

V kapitole \ref{chap:Pose} jsme pro detekci klíčových bodů zvolili model
\textit{YOLO pose}. Ten je předtrénovaný na datasetu \textit{COCO}, který pro
lidské pózy definuje 17 klíčových bodů v dvourozměrném prostoru. To udává
velikost vstupu do našeho klasifikačního algoritmu. Navrhneme tedy a
natrénujeme neuronovou síť, jejíž vstupem bude 17 2D klíčových bodů - tedy 34
čísel - a výstupem bude klasifikace třídy pózy - \textit{normální} a
\textit{upadl}.

V této kapitole se zaměříme na implementaci klasifikačního algoritmu pro
analýzu pózy. Nejdříve se podíváme na použité technologie a knihovny, které nám
pomohou s vývojem. Poté rozebereme možné architektury sítě a ukážeme, jak byly
implementovány. Dále se zaměříme na návrh vnitřní struktury u techto
architektur, zkusíme je natrénovat v různých konfiguracích a nakonec vybereme
nejoptimálnější řešení.

\section{Použité technologie}

\subsection{PyTorch}

Celý projekt byl vyvíjen v prostředí skriptovacího jazyka Python. Samotný vývoj
neuronových sítí probíhal ve frameworku PyTorch. PyTorch je open-source
knihovna pro strojové učení, široce používaná například pro počítačové vidění
či zpracování přirozeného jazyka. Jeho hlavními funkcemi je práce s tenzory
podobnými jako v NumPy, ale s podporou silné akcelerace s využitím GPU, a vývoj
neuronových sítí postavený na vysokoúrovňových stavebních blocích s podporou
automatické derivace pro počítaní gradientů.

Implementace neuronových sítí v PyTorch je velice jednoduchá a díky vysoké míře
abstrakce umožňuje se při vývoji soustředit na samotnou architekturu a design
sítě, nikoliv na detaily implementace. Pro vytvoření nové neuronové sítě stačí
vytvořit třídu, která bude dědit od základní třídy $nn.Module$, v konstruktoru
definovat jednotlivé vrstvy včetně různých regularizačních hyperparametrů. V
metodě $forward$ pak definujeme dopředný průchod sítě. Dále máme možnost
kontrolovat např. výpočet ztrátové funkce či metriky přesnosti.

PyTorch obsahuje také předpřipravené moduly pro GRU či LSTM sítě, včetně
vícevrstvých architektur. Pro tyto moduly definujeme velikost vstupu a velikost
skrytého stavu, dále můžeme definovat počet vrstev či dropout. Výstup z těchto
modulů je pak skrytý stav, který můžeme dále zpracovávat pomocí jednoduché
dopředné sítě pro predikci třídy.

\subsection{Lightning}
\label{sec:Lightning}

PyTorch Lightning je wrapper pro PyTorch, který dále usnadňuje vývoj
neuronových sítí. Stará se za nás o detaily procesu trénování, jako je správa
epoch, logování či optimalizace kroku učení (ang. learning rate). Podporuje
taky nativně práci s TensorBoard, což je logovací nástroj umožňující přehledné
sledování metrik během trénování, včetně grafického zobrazení ve webovém
prostředí.

Pro implementaci modelu vytvoříme třídu, která dědí z
$lightning.LightningModule$, a kromě konstruktoru, ve kterém inicializujeme
jednotlivé vrstvy, definujeme metody $training\_step$ (krok trénování),
$validation\_step$ - krok validace, $configure\_optimizers$ - definování
optimalizační techniky a $forward$, která definuje, jak signál prochází
jednotlivými vrstvami.

\section{Implementace vybraných architektur}
\label{sec:SelectedArchitectures}

Podíváme se nyní, jak jsme jednotlivé architektury implementovali, nezávisle na
jejich vnitřní struktuře a konfiguraci.

\subsection{Dopředná neuronová síť}

Nejjednodušší architekturou, kterou můžeme pro náš model použít, je dopředná
neuronová síť. Tento model pak bude klasifikovat jednotlivé pózy, aniž by znal
jejich kontext. Síť bude klasifikovat klíčové body pouze podle aktuální
lokalizace ve snímku, nikoliv podle pohybu.

Výhodou této architektury je jednoduchost, potažmo rychlost. Síť nepotřebuje
mnoho parametrů a oproti rekurentním sítím potřebuje pro evaluaci pouze jeden
dopředný průchod vrstvami sítě.

Další výhodou je jednoduchost trénování a používání. V případě více osob pro
samotnou klasifikaci pádu není nutné sledování osob. Můžeme jednoduše
klasifikovat všechny detekované pózy, aniž bychom řešili, které osobě patří.

Tato síť ale ve výsledku bude klasifikovat pózy pouze dle vzájemného umístění
jednotlivých klíčových bodů, potažmo délky končetin, nebude ale brát v úvahu
natočení postavy. To proto, že postavy ve snímku vystupují pod různým úhlem
natočení v závislosti na natočení kamery. Naopak síť, která je schopná sledovat
pohyb, bude schopna sledovat mj. i změnu natočení postavy a to bez ohledu na
natočení kamery.

Při použití této architektury jsou jako vstupní data použity klíčové body jedné
osoby z daného jednoho snímku. Trénovací data pak pouze přečteme z trénovacího
souboru a překonvertujeme je do formy tenzoru, konkrétně je zabalíme do
instance třídy $DataLoader$. Při použití ve výsledném programu předáme modelu
klíčové body každé detekované osoby v daném snímku.

Výstupem sítě je hodnota od $0$ do $1$, čísla od $0$ do $0.5$ jsou považována
za třídu \textit{normální}, zatímco čísla od $0.5$ do $1$ za třídu
\textit{upadl}.

Pro implementaci dopředné sítě v Pytorch Lightning jsme použili modul
$nn.sequential$, ve kterém definujeme postupné kroky průchodů sítě. $nn.Linear$
definuje vrstvu sítě včetně počtu neuronů předcházející a aktuální vrstvy. Dále
můžeme definovat aktivační funkci po dané vrstvě a regularizační techniky jako
je dropout či normalizace. Příklad implementace dopředné sítě je uveden v kódu
\ref{src:ffnn}. V tomto příkladě jsou definovány dvě vnitřní vrstvy o velikosti
$128$ a $32$ a výstupní vrstva s jedním neuronem. Mezi vrstvami je aplikována
normalizace a dropout $0.3$, jako aktivační funkce je použita ReLU.

\begin{lstlisting}[language=Python, label=src:ffnn, caption={Ukázka implementace Dopředné sítě v PyTorch Lightning}]
class KeypointClassifierFFNN(L.LightningModule):
    def __init__(self):
        super(KeypointClassifierFFNN, self).__init__()
        self.classifier = nn.Sequential(
            nn.Linear(34, 128),
            nn.BatchNorm1d(128),
            nn.ReLU(),
            nn.Dropout(0.3),
            nn.Linear(128, 32),
            nn.BatchNorm1d(32),
            nn.ReLU(),
            nn.Dropout(0.3),
            nn.Linear(32, 1),
        )
        self.criterion = nn.BCEWithLogitsLoss()
        self.accuracy = BinaryAccuracy(threshold=0.5)
        self.sigmoid = nn.Sigmoid()    
    def forward(self, x):
        x = self.classifier(x)
        return x
    def configure_optimizers(self):
        optimizer = optim.Adam(self.parameters(), lr=1e-4)
        return optimizer
    def training_step(self, batch, batch_idx):
        input, target = batch
        output = self(input)
        loss = self.criterion(output, target.float())    
        self.log("train_loss", loss, on_epoch=True, on_step=False)
        return loss
    def validation_step(self, batch, batch_idx):
        data, target = batch
        output = self(data)
        loss = self.criterion(output, target.float())
        self.log('val_loss', loss, on_epoch=True, on_step=False)
        output = self.sigmoid(output)
        accuracy = self.accuracy(output, target.int())    
        self.log('val_accuracy', accuracy, on_epoch=True, on_step=False)
    return loss
\end{lstlisting}

Abychom mohli tuto síť efektivně trénovat, musíme umožnit jednoduchou změnu
konfigurace sítě. Třída $KeypointClassifierFFNN$ proto v konstruktoru přijímá
parametry \textit{layers} - pole čísel reprezentujících velikosti jednotlivých
vrstev, \textit{activation} - identifikátor vybrané aktivační funkce a
\textit{dropout} definující velikost dropoutu, viz kód \ref{src:ffnn_params}.
Jako ztrátovou funkci jsme použili křížovou entropii, tedy
$nn.BCEWithLogitsLoss()$, která je vhodná pro binární klasifikaci.

\begin{lstlisting}[language=Python, label=src:ffnn_params, caption={Parametrizace konfigurace dopředné sítě}] 
class KeypointClassifierFFNN(L.LightningModule):
    def __init__(self, layers, activation, dropout=0.3, device=None):
        super(KeypointClassifierFFNN, self).__init__()        
        classifier = nn.Sequential()
        for i in range(len(layers)-1):
            classifier.add_module(f'layer_{i}', nn.Linear(layers[i], layers[i+1]))
            classifier.add_module(f'batch_norm_{i}', nn.BatchNorm1d(layers[i+1]))
            classifier.add_module(f'activation_{i}', activations[activation])
            classifier.add_module(f'dropout_{i}', nn.Dropout(dropout))
        classifier.add_module(f'layer_{len(layers)-1}', nn.Linear(layers[-1], 1))
        self.classifier = classifier
        self.criterion = nn.BCEWithLogitsLoss()
        self.accuracy = BinaryAccuracy(threshold=0.5)
        self.layers = layers
        self.sigmoid = nn.Sigmoid()   
\end{lstlisting}

\subsection{GRU síť}

Jelikož pád je událost, nikoliv statická póza, mohlo by být optimálnější použít
algoritmus, který bude analyzovat nejenom aktuální pózu, ale sekvenci
posledních $n$ póz. Pro tento účel se nabízí rekurentní neuronové sítě. V
dnešní době je nejpoužívanější rekurentní architekturou GRU (Gated Recurrent
Unit).

Příklad implementace GRU sítě je uveden v kódu \ref{src:gru}, kde je definována
GRU jednotka s jednou vrstvou, velikosti vstupního vektoru $34$ a skrytým
stavem velikosti $128$, následována dvouvrstvou dopřednou plně propojenou sítí
ze $128$ neurony v první vrstvě a s jedním neuronem ve druhé vrstvě.

\begin{lstlisting}[language=Python, label=src:gru, caption={Ukázka implementace GRU sítě v PyTorch Lightning}]
    class KeypointClassifierGRU(lightning.LightningModule):
        def __init__(self):
            super(KeypointClassifierGRU, self).__init__()
            self.gru = nn.GRU(34, 128)              # GRU vrstvy
            self.classifier = nn.Sequential(        # Plně propojené vrstvy
                nn.Linear(128, 128),
                nn.ReLU(),
                nn.Dropout(0.3),
                nn.Linear(128, 1) )
            self.criterion = nn.BCEWithLogitsLoss ()    # Ztrátová funkce
            self.accuracy = BinaryAccuracy()            # Metrika přesnosti
        def forward(self, x):
            _, x = self.gru(x)
            x = x[-1]                               # Poslední skrytý stav
            x = self.classifier(x)
            return x
    
\end{lstlisting}

Část těchto hyperparametrů jsme opět parametrizovali, abychom mohli postupně
otestovat různé jejich kombinace a vyhodnotit jejich vliv na výkon modelu, viz
kód \ref{src:params}. Jedná se o velikost vstupního vektoru, velikost skrytého
stavu sítě GRU, počet vrstev GRU, velikost první plně propojené vrstvy,
velikost výstupní vrstvy (pro binární klasifikaci vždy $1$, potřebné pro
pozdější optimalizace) a dropout pro GRU i plně propojené vrstvy. Všechny tyto
hyperparametry budeme ladit v další části.

\begin{lstlisting}[language=Python, label=src:params, caption={Parametry konstruktoru třídy $KeypointClassifierGRU$ definující hyperparametry sítě}]
class KeypointClassifierGRU(L.LightningModule):
    def __init__(self, input_size=34, rnn_hidden_size=128, rnn_layers_count=2, fc_size=128, output_size=1, rnn_dropout=0.3, fc_dropout=0.3, device=None):
\end{lstlisting}

Stejně jako u předchozí architektury, jako ztrátová funkce byla použita binární
křížová entropie, tedy $nn.BCEWithLogitsLoss()$, která je vhodná pro binární
klasifikaci.

\subsection{LSTM síť}

Další populární rekurentní architekturou je LSTM (Long Short-Term Memory). Ta
je lepší pro delší sekvence, je to ale za cenu větší komplexity. Jak již bylo
vysvětleno, s nárůstem komplexity se zvyšuje riziko přetrénování, zejména při
nedostatku trénovacích dat. Vzhledem k množství trénovacích dat, které máme k
dispozici, se tak dá předpokládat, že trénování této sítě bude spíše méně
stabilní než v případě sítě GRU.

Implementace sítě LSTM se od GRU liší pouze použitím modulu $nn.LSTM$ místo
$nn.GRU$. Použití těchto modulů je v PyTorch velice podobné, pouze LSTM vrací
kromě skrytého stavu také stav buňky (dlouhodobou paměť), ten ale stejně
nevyužijeme. Pro síť LSTM budeme ladit stejnou množinu hyperparametrů, jako pro
GRU síť, abychom mohli porovnat jejich výkonnost.

\section{Návrh architektury a konfigurace sítě}

Nyní přistoupíme do samotného návrhu architektury a konfigurace sítě. Vybrané
architektury jsme natrénovali v různých konfiguracích a celý postup trénování
jsme zapisovali pomocí logovacího nástroje TensorBoard. Nyní na základě grafů
základních metrik, jako jsou ztrátová funkce či přesnost, zhodnotíme výkon
vybraných architektur, podíváme se, jaký mají jednotlivé hyperparametry vliv na
proces trénování a výsledný výkon. Nakonec vybereme nejoptimálnější variantu.

K hyperparametrům, které jsme zkoušeli, patří počet vrstev a jejich velikost a
velikost dropoutu. U dopředných sítí jsme navíc zkoušeli i různé aktivační
funkce, u rekurentních sítí pak velikost skrytého stavu. Pro optimalizaci váh
během trénování byl použit optimalizační algoritmus Adam \cite{adam}, jehož
výhodou je

\subsection{Dopředná neuronová síť}

Nejprve jsme zkoušeli natrénovat dopřednou neuronovou síť. Vyzkoušeli jsme od
dvou do čtyř vnitřních vrstev, ve velikostech od $32$ do $512$ neuronů, vždy
mocniny dvou. Pojďme postupně rozebrat jednotlivé hyperparametry a jejich vliv
na výkon modelu.

U dopředných sítí se nám nejlépe ověřila velikost dávky $4096$. Síť se lépe
trénovala s použitím normalizace (modul $nn.BatchNorm1d$) a dropoutu. Pro
většinu konfigurací byla velikost dropoutu $0.4$ nejoptimálnější.

Síť jsme zkoušeli trénovat s těmito vybranými aktivačními funkcemi: $ReLU$,
$Tanh$, $PReLU$ a $Mish$. $ReLu$ a $PReLu$ často dosahovaly podobných výsledků,
většinou ale nejlepších výsledků dosahovala $ReLU$. Příkladem je graf přesnosti
pro trénování sítě se třemi vnitřními vrstvami velikosti $128$, $64$, a $32$ na
obrázku \ref{graph:fnnactivations}.

\begin{figure}[] % 'htbp' controls figure placement (here, top, bottom, page)
    \centering
    \caption{Graf přesnosti na validačních datech v průbehu trénování pro různé použité aktivační funkce}
    \label{graph:fnnactivations}
    \begin{tikzpicture}
        \begin{axis}[
                xlabel={Epochy},
                ylabel={Validační ztráta},
                ymin=0.7,
                grid=both,
                width=0.9\textwidth,
                height=0.4\textheight,
                legend pos=south east,
            ]
            \addplot[
                color=blue,
                mark=*,
                no markers
            ]
            table {Plots/relu.dat};
            \addlegendentry{ReLu}
            \addplot[
                color=green,
                mark=*,
                no markers
            ]
            table {Plots/prelu.dat};
            \addlegendentry{PReLu}
            \addplot[
                color=orange,
                mark=*,
                no markers
            ]
            table {Plots/tanh.dat};
            \addlegendentry{Tanh}
            \addplot[
                color=purple,
                mark=*,
                no markers
            ]
            table {Plots/mish.dat};
            \addlegendentry{Mish}
        \end{axis}
    \end{tikzpicture}
\end{figure}

Obecně větší sítě dosahovaly lepších výsledků - menší ztrátové funkce a vyšší
přesnosti na validačních datech, v jistém momentě jsme již ale začali narážet
na problém se stabilitou trénování. Nejoptimálnějších výsledků dosáhla síť se
třemi vrstvami o velikostech $128$, $64$ a $32$. Jak můžeme vidět na grafu
\ref{graph:deepffnn}, sítě s větší šířkou nebo hloubkou sice dosahují lepších
výsledků, začínají ale brzy výrazně kolísat. Zkoušeli jsme v těchto případech i
L2 regularizaci (parametr $weight\_decay$), to ale neřešilo problém. Pro
komplexnější sítě bychom pravděpodobně potřebovali větší množství dat.

Pojďme si tedy shrnout, jaká je pro nás nejoptimálnější konfigurace: použili
jsme tři vrstvy o velikostech $128$, $64$ a $32$, dropout $0.3$, normalizaci
mezi vrstvami, velikost dávky 4096 a aktivační funkci $ReLu$. Dosáhli jsme tak
validační ztráty $0.18$ a validační přesnosti 93.8%

\begin{figure}[] % 'htbp' controls figure placement (here, top, bottom, page)
    \centering
    \caption{Graf validační ztráty v průběhu trénování hlubší sítě }
    \label{graph:deepffnn}
    \begin{tikzpicture}
        \begin{axis}[
                xlabel={Epochy},
                ylabel={Validační ztráta},
                grid=both,
                width=0.9\textwidth,
                height=0.5\textheight,
                legend pos=north east,
                ymax=0.25,
            ]

            \addplot[
                color=blue,
                mark=*,
                no markers
            ]
            table {Plots/fnn_[34,128,64,32].dat};
            \addlegendentry{Velikosti vrstev: 128, 64, 32}
            \addplot[
                color=green,
                mark=*,
                no markers
            ]
            table {Plots/fnn_[34,256,128,32].dat};
            \addlegendentry{Velikosti vrstev: 256, 128, 32}
            \addplot[
                color=orange,
                mark=*,
                no markers
            ]
            table {Plots/fnn_[34,256,128,64,32].dat};
            \addlegendentry{Velikosti vrstev: 256, 128, 64, 32}
            \addplot[
                color=purple,
                mark=*,
                no markers
            ]
            table {Plots/fnn_[34,512,128,64,32].dat};
            \addlegendentry{Velikosti vrstev: 512, 128, 64, 32}
        \end{axis}
    \end{tikzpicture}
\end{figure}

\subsection{GRU síť}

Jak již bylo zmíněno, v oblasti jednodušších RNN jsou dnešním standardem GRU
sítě, sítě LSTM se používají zejména, pokud si GRU s problémem neradí, anebo je
vzhledem k problému důležité uchování dlouhodobých závislostí.

Stejnou taktiku zvolíme i my. Nejdříve otestujeme GRU sítě, a to pro několik
možností délky analyzované sekvence. To znamená, že pro každý snímek předáme
síti očekávanou třídu a sekvenci póz dané osoby z $n$ posledních snímků. Pak
vyzkoušíme síti předávat celou sekvenci póz dané osoby.

Naše GRU síť se skládá z několika vrstev GRU, konkrétně jsme zkoušeli trénovat
1 až 3 vrstvy, které následuje jedná plně propojená vrstva a výstupní vrstva.
Velikosti GRU vrstev jsme volili v rozsahu od 64 do 256, u plně propojené
vrstvy jsme volili mezi 32 a 128 neurony. Většinou jsme používali dávky o
velikosti $4096$, menší dávky velice destabilizovaly trénování.

Zjistili jsme, že na rozdíl od dopředných sítí, v případě GRU nejlepších
výsledků dosahovaly jednodušší sítě. Většina sítí s jednou GRU vrstvou tak
dosahovala poměrně stabilních výsledků. Nejvyšší přesnosti jsme pak dosáhli
sítí s jednou GRU vrstvou o velikosti $64$ a plně propojenou vrstvou o
velikosti $64$, s dropoutem 0.15 v GRU vrstvě a 0.4 v plně propojené vrstvě. I
když většinou bylo dosaženo stabilnějších výsledků s větším dropoutem jako je
0.4, některé jednodušší sítě se lépe trénovaly s menším dropoutem.

Sítě s více GRU vrstvami většinou než dosáhly optimálního výkonu, začaly
kolísat a ve validační ztrátě a přesnosti se objevovaly obrovské výkyvy.
Zkoušeli jsme větší dávky - $8162$, které sice stabilizovaly trénování při
stejném počtu epoch, jelikož se ale vetší dávka loučí s pomalejším učením,
nedosahovaly takové přesnosti jako při dávkách velikosti $4096$. Pro dosažení
podobné přesnosti jsme tedy zkoušeli trénovat ve více epochách, tehdy ale brzy
docházelo k přetrénování.

RNN sítě obecně můžou přijímat sekvence libovolné délky, je ale efektivnější,
pokud je síť trénovaná na jednotné délce, tato délka sekvence je pak použitá i
v případě inferencí. Zkoušeli jsme tedy trénovat sítě na sekvencích o délce
$50$ a $100$ snímků, vyzkoušeli jsme i neomezenou délku sekvence, tedy modelu
předáváme klíčové body z celé sekvence snímků, na kterých byla daná osoba
detekována. Obecně jsme ale zjistili, že vždycky dosahují stabilnějších
výsledků sítě spíše s kratšími sekvencemi. Zůstali jsme tedy u sekvencí délky
$50$.

\subsection{LSTM síť}

LSTM sítě jsme trénovali se stejnými hyperparametry a konfiguracemi jako sítě
GRU. Obecně se taky dá říct, že měly tyto parametry u obou architektur velice
podobný vliv na výsldky, a tedy to co jsme napsali o GRU sítích, by se většinou
dalo napsát i o LSTM sítích. Taky měly nejlepší výsledky s jednou vrstvou,
podobně se projevovala déelka sekvence, a obdobně byl většinou optimální
dropout velikosti $0.4$.

Obecně ale LSTM sítě byly mnohem méně stabilní, a prakticky v žádné konfiguraci
se nepodařilo natrénovat síť do požadované přesnosti, aniž by došlo k
přetrénování anebo by začala validační ztráta extrémně kolísat. Pravděpodobně
tedy nemáme dostatečný vzorek trénovacích dat pro trénování LSTM sítě.


%
%
%
% TODO ::(
%
%
%
%
%
%
%
%
%
%
%
%
%

% \endinput
% %
% %
% %
% %
% %
% %
% %
% %
% %
% %
% %
% %
% %
% %
% %
% %
% %
% %
% %
% %
% %
% %
% %
% %
% %
% %
% %
% %
% %
% %
% %
% %
% %
% %
% %
% %
% %
% %
% %
% %
% %
% %
% %
% %
% %
% %
% \subsection{Vícetřídní klasifikace}

% Jak již bylo zmíněno (\ref{sec:TrainingData}), pro naše řešení potřebujeme
% detekovat pouze třídy \textit{normální} a \textit{upadl}, vyzkoušíme ale
% natrénovat i modely se třemi třídami - přidáme třídu \textit{padá}. Nemá smysl
% v našem případě implementovat takovou dopřednou neuronovou síť, jelikož
% klasifikuje snímky bez kontextu předchozích snímků.

% Budeme tedy navíc implementovat sítě GRU a LSTM, s velice podobnou topologií
% jako sítě s binární klasifikací. Hlavním rozdílem bude velikost výstupní
% vrstvy, jež bude rovna počtu tříd a použitá ztrátová funkce - nyní použijeme
% křížovou entropii. Výstup vícetřídní klasifikační sítě pak představuje pole
% pravděpodobností, že se jedná o danou třídu. Při inferenci pak použijeme
% jednotku softmax pro převod výstupu na číslo třídy.
% \chapter{Experimenty s topologiemi klasifikační sítě}
\label{chap:ClassificationExperiments}

% TODO


\chapter{Implementace detekčního algoritmu}
\label{chap:detectionAlgorithm}

V této kapitole se zaměříme na implementaci samotného detekčního algoritmu,
který na základě postupně předáváné sekvence snímku bude detekovat, zda došlo k
pádu.

\section{Sledování s YOLO}

V YOLO verze 11 máme kromě již zmíněné detekce objektů i klíčových bodů k
dispozici také další funkce. To, které funkce chceme využít, definujeme
vybraným modelem. V závislosti na něm pak pak při interferenci model vrací
patřičné hodnoty. Dostupné jsou tyto modely:
\begin{itemize}
    \item \textit{YOLO11<v>-seg }- detekce objektů - bounding boxů
    \item \textit{YOLO11<v>-cls }- detekce objektů, klíčových bodů a segmentace
    \item \textit{YOLO11<v>-pose} - detekce klíčových bodů
    \item \textit{YOLO11<v>-obb }- orientovaná detekce objektů - bounding boxy natočené dle natočení objektu
\end{itemize}

kde $<v>$ označuje velikost modelu - můžeme vybrat menší modely pro větší
výkon ale horší přesnost, anebo větší modely, které jsou sice velmi přesné,
musíme ale počítat s vysokými nároky na výkon. V našem případě jsme zvolili model $yolo11m-pose$

\endinput
% \part{Detekce pádu}
\chapter{Závěr}
\label{chap:Conclusion}

Cílem práce bylo navrhnout a implementovat řešení pro detekci pádu osoby v toku
obrázku v reálném čase. V první části byly popsány teoretické základy použitých
technologií, se zaměřením na neuronové sítě.

V praktické části byl popsán postup návrhu a vývoje výsledného programu. Řešení
jsme dosáhli spojením volně dostupného modelu pro detekci pózy ve formě
klíčových bodů a neuronové sítě pro klasifikaci těchto bodů do třídy
\textit{normální} nebo \textit{upadl}.

Prozkoumali jsme několik detekčních algoritmů – vysvětlili jsme si zásady
funkčnosti detekce klíčových bodů a otestovali jsme jednotlivé algoritmy na
testovacích datech. Pro detekci pózy jsme nakonec zvolili model YOLO11-pose pro
jeho vysokou rychlost, zejména při využití grafické akcelerace, a přesnost i
při horší viditelnosti. Tento model také v sobě integruje funkci sledování
osob, která nám umožňuje provádět následnou analýzu pro každou osobu zvlášť i v
kontextu více snímků.

Pro klasifikaci pózy jsme zkoušeli natrénovat modely postavené na třech  
architekturách: dopředná neuronová síť a dvě rekurentní architektury – LSTM a
GRU. LSTM síť se nám nepodařilo efektivně natrénovat, pravděpodobně pro
nedostatek dat a její nevhodnost pro náš problém.

Dopřednou síť a GRU síť jsme po nalezení optimální konfigurace a natrénování
otestovali na testovacích datech. Následně jsme je použili v implementaci
detektoru pádu. Ten jsme dále v obou verzích otestovali na videích z testovací
sady. Dopředná síť se ukázala jako znatelně méně přesná, zejména při méně
kvalitní detekci pózy v situacích s horší viditelností. Také projevovala
falešné pozitivní detekce pádu.

Pro klasifikaci pózy jsme tedy zvolili rekurentní architekturu GRU s jednou
rekurentní a jednou plně propojenou vrstvou, obě velikosti $64$. Tuto síť se
nám podařilo natrénovat na přesnost přes 96\% na testovacích datech, zároveň se
ukázala jako velice stabilní a přesná i při detekci pádu v testovacích videích.

Pokud bychom chtěli pokračovat v optimalizaci našeho řešení, nejvíce
prostoru vidíme v množství trénovacích dat. Zejména by bylo vhodné použít videa s
více úhly natočení vůči zemi a s různými vzdálenostmi osoby od kamery. Větší a
rozmanitější dataset by nám umožnil natrénovat i komplexnější architektury,
které by byly přesnější a robustnější. Pro získání více dat by se mohlo nasadit
náš detektor na podnikové kamery, nebo by se pomocí něj dalo zanalyzovat starší
záznamy. V těchto případech by bylo vhodné nastavit nižší prah detekce, abychom
získali i data s vyšší pravděpodobností falešné pozitivity.

Dalším vylepšením by mohlo být zavedení více tříd do analýzy pózy. V textu byla
zmíněna třída \textit{padá}, nicméně bychom mohli také přidat třídy např. pro
pozici v sedě, ve dřepu, či na kolenou. Model by tak mohl lépe rozeznávat
některé situace, které se zdají být pádu podobné, a tak bychom odstranili
potenciální falešné pozitivní detekce pádu.

Zajímavým rozšířením by také mohlo být propojení více kamer z jedné místnosti
pro sledování osob z více úhlů. Dávalo by to možnost zasazení postav do
trojrozměrného prostoru a umožňovalo dosáhnout mnohem kvalitnější detekce
pózy. Nicméně by bylo obtížnější získat pro takové řešení trénovací data.

Pro nasazení detektoru pádu v reálném provoze, tedy na obrazovém toku z
podnikových bezpečnostních kamer, bude ještě třeba vytvořit vhodné rozhraní pro
správné čtení různých protokolů a formátů videa. Pro integraci do komplexnějšího
programu, ve kterém by probíhalo více analýz videa, by bylo vhodné zvážit
jednak jejich paralelní zpracování, jednak sdílení určitých dat mezi
jednotlivými analyzátory.

\endinput

% Seznam literatury
\printbibliography[title={Literatura}]

% Prilohy
\appendix
%\input{Chapters/Appendix1.tex}
%\input{Chapters/Appendix2.tex}

% Priloha vlozena primo do hlavniho LaTeX souboru. Ne vsechny prilohy je nutne mit ve zvlastnich souborech.
%\chapter{Dlouhý zdrojový kód}
%\lstinputlisting[label=src:CppExternal,caption={Dlouhý zdrojový kód v jazyce C++ načtený s externího souboru}]{SourceCodes/ArraySortingAlgorithms.cpp}

\end{document}

