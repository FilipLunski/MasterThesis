
\documentclass[czech,master]{diploma}
% Dalsi doplnujici baliky maker
\usepackage[autostyle=true,czech=quotes]{csquotes} % korektni sazba uvozovek, podpora pro balik biblatex
\usepackage[backend=biber, style=iso-numeric, alldates=iso]{biblatex} % bibliografie
\usepackage{dcolumn} % sloupce tabulky s ciselnymi hodnotami
\usepackage{subfig} % makra pro "podobrazky" a "podtabulky"
\usepackage[cpp]{diplomalst}


\usepackage{pgfplots}
\usepackage{amsmath}

\usepackage{pgffor}
\usepackage{catchfile}

% Novy druh tabulkoveho sloupce, ve kterem jsou cisla zarovnana podle desetinne carky
\newcolumntype{d}[1]{D{,}{,}{#1}}



% Zadame pozadovane vstupy pro generovani titulnich stran.
\ThesisAuthor{Filip Łuński}

\ThesisSupervisor{Ing. Tomáš Wiszczor, Ph.D.}

\CzechThesisTitle{Využití kamerového systému pro zajištěni bezpečnosti osob na pracovišti}

\EnglishThesisTitle{Use of Surveillance Cameras to Ensure the Safety of People in the Workplace}

\SubmissionYear{2025}

\ThesisAssignmentFileName{ThesisSpecification_LUN0024_vsboee2404016E.pdf}


\Acknowledgement{Rád bych na tomto místě poděkoval všem, kteří mi s prací pomohli, protože bez nich by tato práce nevznikla.}


\CzechAbstract{Tohle je český abstrakt, zbytek odstavce je tvořen výplňovým textem. Naší si rozmachu potřebami s posílat v poskytnout ty má plot. Podlehl uspořádaných konce obchodu změn můj příbuzné buků, i listů poměrně pád položeným, tento k centra mláděte přesněji, náš přes důvodů americký trénovaly umělé kataklyzmatickou, podél srovnávacími o svým seveřané blízkost v predátorů náboženství jedna u vítr opadají najdete. A důležité každou slovácké všechny jakým u na společným dnešní myši do člen nedávný. Zjistí hází vymíráním výborná.}

\CzechKeywords{python, strojové učení, neuronové sítě, konvoluční neuronové sítě, detekce pozy, detekce chování, detekce pádu, }

\EnglishAbstract{This is English abstract. Lorem ipsum dolor sit amet, consectetuer adipiscing elit. Fusce tellus odio, dapibus id fermentum quis, suscipit id erat. Aenean placerat. Vivamus ac leo pretium faucibus. Duis risus. Fusce consectetuer risus a nunc. Duis ante orci, molestie vitae vehicula venenatis, tincidunt ac pede. Aliquam erat volutpat. Donec vitae arcu. Nullam lectus justo, vulputate eget mollis sed, tempor sed magna. Curabitur ligula sapien, pulvinar a vestibulum quis, facilisis vel sapien. Vestibulum fermentum tortor id mi. Etiam bibendum elit eget erat. Pellentesque pretium lectus id turpis. Nulla quis diam.}

\EnglishKeywords{python, machine learning, neural networks, convolutional neural networks, reccurent neural networks, pose estimation, behaviour detection, fall detection, YOLO }

% Seznam zkratek a symbolů 
\AddAcronym{AF}{Aktivační funkce}

\AddAcronym{NN}{Neural network – neuronová síť}
\AddAcronym{ANN}{Artificial neural network – umělá neuronová síť}
\AddAcronym{FFNN}{Feedforward Neural Network – dopředná neuronová síť}
\AddAcronym{CNN}{Convolutional neural network – konvoluční neuronová síť}
\AddAcronym{RNN}{Recurrent neural network – rekurentní neuronová síť}
\AddAcronym{LSTM}{Long short-term memory – dlouhá krátkodobá paměť}
\AddAcronym{GRU}{Gated recurrent unit}
\AddAcronym{AI}{Artificial intelligence – umělá inteligence}
\AddAcronym{ML}{Machine learning – strojové učení}
\AddAcronym{DL}{Deep learning – hluboké učení}
\AddAcronym{RoI}{Region of interest – oblast zájmu}
\AddAcronym{PAF}{Part affinity field – pole propojení klíčových bodů}
\AddAcronym{ReLU}{Rectified Linear Unit}
\AddAcronym{PReLU}{Parametric Rectified Linear Unit}
\AddAcronym{Adam}{Adaptive Moment Estimation}
\AddAcronym{BP}{Backpropagation}
\AddAcronym{BPTT}{Backpropagation Through Time – zpětné šíření chyby v čase}
\AddAcronym{BCE}{Binary Cross-Entropy}
\AddAcronym{RMSprop}{Root Mean Square Propagation}

\endinput

% Vlozeni bibliografie
\addbibresource{biblatex.bib}


% Zacatek dokumentu
\begin{document}

% Nechame vysazet titulni strany.
\MakeTitlePages

% Jsou v praci obrazky nebo tabulky? Pokud ano vysazime jejich seznam a odstrankujeme.
% Pokud ne smazeme nasledujici dve makra.
\listoffigures
\clearpage

\listoftables
\clearpage

% A nasleduje text zaverecne prace.
\chapter{Úvod}
\label{chap:Introduction}

Kamerové systémy jsou využívány již mnoho let a jejich využití je stále širší.
Již před několika lety se odhadovalo, že celkový počet bezpečnostních kamer ve
světě přesahuje miliardu \cite{surveillance}. Využívány jsou v průmyslu,
dopravě, obchodech, veřejných prostorech, zdravotnictví či domácnostech.

Zpočátku bylo možné video sledovat pouze živě, později, s příchodem videokazet,
bylo možné záznam sledovat rovněž zpětně. Digitální éra a síťové kamery
umožnily přístup ke kamerovým záznamům z libovolného místa na světě. V poslední
době se také začalo nahrazovat živé sledování automatickým zpracováním obrazu a
detekcí událostí s využitím technik umělé inteligence.

Kamerové systémy se používají zejména ve dvou oblastech: zabezpečení (ang.
security), myšleno jako ochrana před úmyslnými hrozbami a protiprávními činy,
jako jsou krádeže, poškozování majetku, či neoprávněný vstup; a bezpečnost
(ang. safety), což zahrnuje ochranu před nehodami a náhodnými hrozbami, jako
jsou pády, požár, úniky nebezpečných látek, či porušování bezpečnostních
předpisů.

Jak již bylo zmíněno, kamery lze využívat jednak pro živé sledování, jednak pro
záznam a jeho analýzu po události. Kamerové záznamy jsou důležité zejména pro
zpětnou analýzu incidentů, důkazní materiál pro soudní spory, zjišťování příčin
nehod, či pro zlepšení bezpečnostních opatření. Živé sledování videa se pak
snaží incidentům přímo předcházet. Bylo však prokázáno, že schopnost lidského
pozorovatele detekovat nebezpečí se velmi snižuje s délkou sledování a s počtem
monitorovaných kamer \cite{soton371614}. Právě proto se s příchodem technik
umělé inteligence začalo využívat automatické zpracování obrazu a detekce
hrozeb, nebezpečí, nebo již probíhajících incidentů v jejich počátcích. Tyto
techniky pak úplně nahrazují lidského pozorovatele, nebo mu pomáhají včas
zpozorovat nebezpečí a zareagovat.

Automatická analýza obrazu je používaná již několik desítek let, většinou ale
spíše pro oblast zabezpečení, než pro bezpečnost. To z toho důvodu, že úlohy,
jako identifikace neoprávněného vstupu, detekce zbraní, rozpoznávání registračních značek vozidel nebo
podezřelých osob dle obličeje jsou pro algoritmy mnohem jednodušší, než
například detekce pádu, nouzové situace či zdravotního problému. Jedná se zde
totiž o vysokou míru abstrakce, kdy i člověk může špatně identifikovat některé
situace. Hlavním problémem těchto komplexnějších analýz je vysoká falešná
pozitivita, kdy je například těžké rozeznat člověka trénujícího běh od člověka
utíkajícího před nebezpečím. Nicméně rozvoj v oblasti hlubokého učení a
konvolučních neuronových sítí, jako i vývoj a dostupnost hardwaru podporujícího
tyto techniky, umožňuje dnes jejich využití i pro složitější úlohy.

Ve firmě Linde jsou kamerové systémy používány v mnoha průmyslových provozech,
nicméně chybí ucelený systém pro automatickou analýzu obrazu a detekci různých
druhů nebezpečí. Úkolem tedy v budoucnu bude navrhnout a implementovat
modulární systém s možností sledování konkrétních nebezpečí na konkrétních
místech. Ty budou zahrnovat například detekci pádu, požáru, zdravotních
problémů, nebo porušování bezpečnostních opatření. Systém pak bude v případě
rozpoznání nějaké hrozby informovat příslušného pracovníka.

Tato práce je zaměřena pouze na jednu z těchto úloh, a to na detekci pádu. Pád
může mít různé příčiny, ať už je to zdravotní problém jako ztráta vědomí, nebo
zakopnutí. Někdy se zdá, že samotné zakopnutí je banální problém, nicméně pokud
se na pracovišti nenachází nikdo, kdo by mohl pomoct, a poškozený není schopen
sám přivolat pomoc, může vést takový incident k vážným následkům.

První část práce bude věnována teoretickým základům, seznámení z neuronovými
sítěmi. Kromě obecného popisu neuronových sítí bude pozornost věnována také jejich pokročilejším architekturám, které budou v řešení použity. Další kapitoly budou zaměřeny na detekci osob a odhad jejich klíčových bodů. Budou představeny různé
přístupy a otestovány různé algoritmy s ohledem na výkon, možnou hardwarovou
akceleraci a preciznost. Předmětem další části bude samotná detekce pádu, tedy
algoritmus, který na základě odhadnutých klíčových bodů určí, zda došlo k pádu,
či nikoliv. Poslední část práce bude zaměřena na kompletaci a otestování výsledného
řešení a zhodnocení jeho výkonu.

\endinput
% \part{Teorie}
\chapter{Neuronové sítě}
\label{sec:NN}

Umělá neuronová síť (ang. Artificial Neural Network - ANN) nebo jen neuronová
síť (ang. Neural Network - NN) je výpočetní model inspirovaný biologickými
nervovými systémy v lidském mozku. Na rozdíl od konvenčních výpočetních modelů,
které zpracovávají informace algoritmicky, a tedy postupují dle předem určeného
postupu, se informace v tomto modelu šíří v síti váh mezi jednotlivými neurony.
Jelikož je výstup ze sítě dané architektury závislý hlavně na numerických
parametrech, zejména váhách jednotlivých spojů mezi neurony, lze funkčnost sítě
měnit bez změny programu pouhou změnou těchto parametrů, a to i automaticky v
procesu trénování modelu.

Nyní krátce projdeme historií vývoje neuronových s síti.

\section{Historie}
\label{sec:NN_History}

\subsection{Prvopočátky}
První matematický model neuronové sítě byl popsán v roce 1943 dvěma
neurofyziology - Warrenem McCullochem a Walterem Pittsem. \cite{McCulloch1943}
Model byl založen na síti jednoduchých logických prvků, které provedou vážený
součet svých vstupů a na výstup odešle signál založený na prahové funkcí.

V roce 1958 pak Frank Rosenblatt představil elektronicky model neuronové sítě.
Základní jednotku, postavenou na McCulloch-Pittsově modelu, nazval perceptron.
\cite{Rosenblatt1958} Jeho architektura byla podobná modelu znázorněnému na
obrázku \ref{fig:neuron}, kde aktivační funkce je prahová funkce. Rosenblattův
stroj - Mark I Perceptron - byl postavený pro rozpoznávání jednoduchých vzorů v
obrazech. Hlavním omezením tohoto modelu bylo, že byl schopen rozlišovat pouze
lineárně separovatelné třídy.

Další systém - ADALINE (Adaptive Linear Neuron) - byl představen Bernardem
Widrowem and Tedym Hoffem v roce 1960. Tento model umělého neuronu byl velmi
podobný perceptronu, na rozdíl od něj ale neobsahoval prahovou ale lineární
funkcí, výstup tedy nebyl binární ale spojitý. Pro učení pak byla využitá
metoda nejmenších čtverců, která minimalizovala chybu mezi skutečným a
očekávaným výstupem. \cite{nn_history}

I když ve svých počátcích přitahoval koncept umělé inteligence mnoho vědců jako
i sponzorů, v následujících létech zájem ochabl, jelikož nebylo dosaženo
předpokládaných výsledků, hlavně s ohledem na tehdejší stav vývoje hardwaru a
obecně výpočetní techniky. Proto se tomuto období někdy říká Ai Winter.
Neznamená to ale, že ti, kteří se oboru nadále věnovali, nedosáhli významných
výsledků. \cite{nn_history}

\subsection{Objev backpropagation}
Významným milníkem v historii neuronových sítí byl objev algoritmu
backpropagation, zvaného taky algoritmus zpětného šíření chyby. Tento
algoritmus byl vyvinut v roce 1974 Paulem Werbosem, popularitu ale dosáhl až po
nezávislém objevení v roce 1986 Davidem Rumelhartem et al.
\cite{backpropagation}

Tento algoritmus umožnil trénovat sítě s více vrstvami, což položilo základ
hlubokému učení. Algoritmus využívá metodu gradientního sestupu v kombinaci s
řetězovým pravidlem derivací k nalezení optimálních vah sítě vedoucích k
minimalizaci chyby.

Vynález backpropagation byl jedním z hlavních důvodů, proč se v 80. letech
obnovil zájem o neuronové sítě a umělou inteligenci obecně. 



\begin{figure}[]
    \centering
    \includegraphics[width=0.5\textwidth]{Figures/neuron.png}
    \caption{Model umělého neuronu \cite{lagan}}
    \label{fig:neuron}
\end{figure}

\endinput
\chapter{Konvoluční neuronové sítě}
\label{sec:CNN}

\endinput
\chapter{Rekurentní neuronové sítě}
\label{sec:RNN}



\endinput
% \part{Detekce osob a jejich pozice}
\chapter{Představení problematiky detekce pádu}
\label{chap:Goal}

V této kapitole je stanoven přesný cíl a navržen postup, jak k problému
přistoupit.

Úkolem bude v reálném čase z videostreamu detekovat pád osoby. Pád osoby
je definován jako náhle, neúmyslné klesnutí těla z výškové pozice (např. stání,
chůze nebo sezení) na zem nebo jinou nižší úroveň, přičemž tato osoba nemá
kontrolu nad tímto pohybem. Samozřejmě není vždy možné úplně dobře
rozeznat, zda se nejedná o úmyslné klesnutí, např. prudké lehnutí.

Dle některých definic (zejména ve zdravotnictví) se o pád nejedná, pokud jde o
důsledek závažné vnitřní příhody (např. mrtvice). Toto zde nebude rozlišováno,
naopak bude cílem detekovat jak pády v důsledku ztráty rovnováhy či vlivem
vnějších faktorů (např. zakopnutí, převrácení těžkým předmětem), tak pády v
důsledku akutních události vlivem zdravotních problémů, jako jsou např.
mrtvice, záchvaty, mdloby či jiné důvody ztráty vědomí.

\section{Návrh řešení}

Cílem této práce je navrhnout algoritmus, který bude detekovat, zda je ve
vstupní sekvenci snímku některá osoba, jejíž pozice je klasifikována jako pád.
Hlavním cílem výsledného programu bude alarmovat příslušného pracovníka, pokud
osoba upadne.

Alarmovat se bude až, pokud osoba zůstane v ležící pozici. To umožní
odfiltrovat falešné alarmy v případě sehnutí či pokud bude osoba špatně
viditelná a algoritmus tak na okamžik špatně vyhodnotí její pohyb. Tímto
postprocesingem se ale práce nebude zabývat, spíše bude pozornost věnována
samotné klasifikaci pozice.

Stejně jako u detekce objektů, viz \ref{sec:obj_det}, by bylo možné i pro
detekci pádu vytvořit vhodnou konvoluční síť, která by přímo z obrázku
definovala, zda se jedná o pád nebo ne. U detekce se už dnes sice s ohledem na
pokrok hardwaru tento přístup používá, nicméně se jedná o velmi náročný úkol,
který vyžaduje rozsáhlou optimalizaci, pokročilou architekturu a velké množství
trénovacích dat. Nicméně, pokud by se podařilo takovouto síť natrénovat, mohla
by lépe detekovat některé situace např. podle výrazu tváře.

V navrženém řešení tedy bude v prvním kroku pomocí vhodné předtrénované
neuronové sítě detekována pozice osoby ve formě klíčových bodů, tato část bude
nazývána \textit{detekční algoritmus}. Na základě těchto bodů pak další
neuronová síť vyhodnotí, zda se jedná o pád, tato část bude nazývána
\textit{klasifikační algoritmus}. To úlohu velice zjednoduší, jelikož místo
analyzování tisíců pixelů, bude algoritmus analyzovat pár desítek klíčových
bodů. Další výhodou je, že u takového postupu je možné použít techniky, kdy je
sledována změna pózy v čase, což by bylo mnohem složitější s jednofázovou
konvoluční síti.

Detekční algoritmus dostane na vstup celý snímek a může detekovat několik osob.
Klasifikační algoritmus ale bude zpracovávat každou osobu, resp. její pózu,
zvlášť.

Další alternativou by mohlo být pouze detekovat osoby jako objekty, a na
základě jejich bounding boxů určit, zda se jedná o pád. Tento postup by byl
jednodušší na dvou úrovních. Jednak je detekce objektů méně náročná úloha než
detekce pózy, jednak by ve druhé fázi bylo analyzováno pouze několik parametrů
bounding boxu (rozměry a velikost) oproti pár desítkám klíčových bodů. Nicméně,
pokud se nad tím zamyslíme, ne vždy vypovídají parametry bounding boxů o pozici
člověka. Tento postup by tak pravděpodobně vedl k mnohem méně přesnému
výsledku, než analýza klíčových bodů, kdy může síť analyzovat takové vzorce
jako je např. délka končetin v pohledu či úhel mezi nimi.

Nyní bude popsáno, jak a z jakými daty se bude pracovat, zejména při trénování,
v další kapitole pak bude rozebrána problematika detekce pózy a bude zvolen
algoritmus pro detekci klíčových bodů. Dále se bude práce věnovat vývoji modelu
detekujícího pád na základě těchto klíčových bodů.

\section{Trénovací datasety}
\label{sec:TrainingData}

Pro trénování vyvíjeného modelu jsme použili necelých 150 krátkých (1 až 15
sekund) videí ze dvou zdrojů. Prvním je dataset CAUCAFall vytvořený právě pro
práci s pády osob \cite{caucafall}. Tento dataset obsahuje 100 nahrávek
simulovaných pádu v různých světelných podmínkách, s různými osobami. Zahrnují
širokou škálu scénářů, jednak pro různé druhy pádů (v různých směrech či z
židle), jednak pro situace podobné pádu, jako je kleknutí či sehnutí se, jednak
běžné činnosti jako chůze či sednutí. Poměr videí s úpadkem a bez je 50/50.

Dalším zdrojem pro trénovací data je YouTube video tvůrce Kevina Parryho
\textit{50 Ways to Fall}. Ve videu autor simuluje pády v různých scénářích,
jako je zakopnutí, omdlení či poražení elektrickým proudem. Vzhledem k zabavné
povaze videa byly některé scénáře vypuštěny, nakonec jsme použili 45 videí. Zde
prakticky všechny videa obsahují pád, v některých se ale postava vrátí do
normálního stavu (např. kotrmelec).

\begin{figure}[]
    \centering
    \includegraphics[width=0.45\textwidth]{Figures/datasets_examples/cauca1.png}
    \includegraphics[width=0.45\textwidth]{Figures/datasets_examples/cauca2.png}
    \includegraphics[width=0.45\textwidth]{Figures/datasets_examples/fifty1.png}
    \includegraphics[width=0.45\textwidth]{Figures/datasets_examples/fifty2.png}
    \caption{Příkladové snímky z datasetů CAUCAFall (nahoře) a 50 Ways to Fall (dole).}
    \label{fig:datasets_examples}
\end{figure}

V obou případech se jedná o videa vždy jedné osoby. To proto, že použijeme
detekční algoritmus již natrénovaný na videích s více osobami, a náš
klasifikační algoritmus bude zpracovávat každou osobu zvlášť. Práce s více
osobami tak bude úlohou výsledného detektoru, nikoliv trénované klasifikační
sítě.

Videa z datasetu CAUCAFall jsou v rozlišení $720\times480$, zatímco videa z
datasetu 50 Ways to Fall jsou v rozlišení $1280\times720$.

Oba datasety byly rozděleny do tří sad: trénovací, validační a testovací.
Rozděleny byly v tomto poměru: trénovací sada 70\%, validační sada 15\% a
testovací sada 15\%. Trénovací a validační sada budou použity v procesu
trénování – trénovací pro výpočet ztráty a úpravu vah, validační pro průběžné
ověřování výkonu, testovací sada bude na konci použitá pro otestování jednak
klasifikačního algoritmu, jednak celkového řešení.

\section{Třídy a jejich anotace}
Pro videa byly vytvořeny anotace aktuální třídy pózy. Tato anotace není pro
každý snímek, ale pouze při změně definuje časovou značku a následující třídu.

V anotacích byly použity 3 třídy, ty odpovídají třem různým třídám pózy
relevantním k problému – \textit{normální}, kdy osoba např. chodí, sedí nebo
stojí, \textit{padá} – přechodný stav padání, definován od započatí pohybu
směrem dolů, a \textit{upadl} – definován od momentu, kdy se osoba dotkla země
trupem nebo všemi končetinami.

Pro řešený problém jsou obecně potřebné jenom dvě třídy – \textit{normální} a
\textit{upadl}. Skript tvořící trénovací data proto považuje třídu
\textit{padá} za třídu \textit{normální}. Nicméně by se mohlo do budoucna pro
pokročilejší optimalizaci zkusit experimentovat i se třídou \textit{padá},
která by mohla síti pomoct hlouběji pochopit problematiku a přesněji rozeznat
některé situace, zejména pak v případě využití rekurentních neuronových sítí.

\section{Příprava trénovacích dat pro klasifikační algoritmus}

Dále byl vytvořen skript, který prošel každé video z trénovacích datasetů a na
základě výše zmíněných anotací vytvořil trénovací data pro klasifikační
síť. Ty obsahují pro každý snímek detekované klíčové body (jako vstup) a
aktuální třídu (jako požadovaný výstup). Pro detekci klíčových bodů byl použit
vybraný model pro detekci pózy. Výběr modelu je popsán v následující kapitole.
Na modelu použitém při tvorbě dat by teoreticky nemuselo záležet (pokud detekuje stejné typy
klíčových bodů), je ale lepší použít ve výsledném programu stejný model jako
pro trénovací data. Modely se totiž můžou v některých situacích chovat trochu
jinak (např. okluze) a klasifikační algoritmus by tak dostával v praxi jiná data, než pro
jaké byl natrénován.

Jelikož pro rekurentní neuronové sítě potřebujeme sekvenci snímků, musí být
trénovací data ještě zpracována. To ale bude již součastí samotného trénovacího
skriptu, jelikož se konečná podoba dat může lišit délkou sekvencí.
\endinput
\chapter{Výběr algoritmu pro detekci klíčových bodů}
\label{chap:Pose}

Jelikož je dnes dostupných mnoho různých algoritmů či natrénovaných modelů pro
detekci pózy osob v obrázku či videu, nemá smysl pro navrhované řešení
implementovat takovýto algoritmus od nuly. Možné by to samozřejmě bylo, i
vzhledem k dostupnosti otevřených trénovacích dat (např. dataset COCO
\cite{coco}), nicméně by pravděpodobně nebylo dosaženo tak kvalitních výsledků,
jako řešení, která jsou výsledkem mnoholetých výzkumů. Hlavně pak by bylo těžko
dosáhnout výkonu těchto řešení, a ten je pro navrhované řešení stěžejní,
jelikož je potřeba video zpracovávat v reálném čase.

V následující kapitole budou popsány obecné principy detekce osob a jejich pózy
v obraze. Následně budou popsány některé populární algoritmy pro detekci pózy
se zaměřením na jejich specifika. Několik z nich pak bude otestováno, výsledky
budou porovnány, a na jejich základě bude zvolen algoritmus použitý v konečném
řešení detekce pádu.

\section{Detekce pózy}

\begin{figure}[]
    \centering
    \begin{minipage}{0.48\textwidth}
        \centering
        \includegraphics[width=0.5\textwidth]{Figures/keypoints.png}
    \end{minipage}
    \hfill
    \begin{minipage}{0.48\textwidth}
        \centering
        \includegraphics[width=0.35\textwidth]{Figures/pose1.png}
    \end{minipage}
    \caption{(Vlevo) Topologie klíčových bodů použitá např. v COCO-pose \cite{2dhpe} (Vpravo) Příklad detekce pózy pomocí YOLO.}
    \label{fig:keypoints}
\end{figure}

Úloha detekce pózy spočívá v nalezení klíčových bodů postavy v obraze.
% Může se
% jednat také o zvíře, v našem případě se ale budeme zabývat pouze klíčovými body
% lidské postavy. 
Klíčové body představují důležité body lidského těla, znalost jejich lokalizace
nám umožňuje analyzovat pózu dané osoby, popřípadě sledovat její pohyb. K
základním klíčovým bodům patří hlava, ramena, lokty, zápěstí, kyčle, kolena a
kotníky, viz obrázek \ref{fig:keypoints}. Některé algoritmy dokážou rozeznat i
orientaci dlaně či stopy, nebo rozpoznat klíčové body na hlavě, jako jsou ústa,
nos, oči a uši \cite{blazepose}.

Klíčové body jsou většinou reprezentovány jako dvojice souřadnic $(x, y)$
vzhledem k celému obrazu, některé algoritmy poskytují i souřadnice
normalizované vzhledem k bounding boxu osoby. Existují také algoritmy pro 3D
souřadnice, těmto ale nebude věnována pozornost, i když by mohly stanovit
zajímavou alternativu, zejména pokud by pro detekci bylo použito více kamer z
různých pohledů.

V oblasti algoritmů pro detekci pózy existují dva základní přístupy: zdola
nahoru a shora dolů. Přístup zdola nahoru se snaží detekovat všechny klíčové
body v obraze, aniž by rozlišoval jednotlivé osoby, pokud je algoritmus schopen
detekce pózy pro více osob, pak v dalším kroku tyto body spojuje do
jednotlivých postav. Naproti tomu přístup shora dolů nejprve detekuje všechny
osoby v obraze, v jejich rámci pak detekuje klíčové body.

%todo 
\section{Detekce klíčových bodů}

\subsubsection*{Heatmapy}

U obou výše zmíněných přístupů se nejčastěji provádí vyhledání všech klíčových
bodů pomocí tzv. heatmap. Je to 2D mapa pravděpodobnosti, že se v daném bodě
vyskytuje nějaký klíčový bod. Maximální hodnoty v této mapě pak představují
lokalizaci klíčových bodů.

Pro vygenerování heatmap se používá konvoluční neuronová síť. Pro každý klíčový
bod, resp. pro každý typ klíčového bodu k (v případě detekce pózy více osob)
vzniká jedna heatmapa. Jako referenční heatmapy pro trénování se používají
mapy, kde je klíčový bod reprezentován 2D Gaussovým rozložením s vrcholem v
místě daného bodu.

V dalším kroku jsou z heatmap vygenerovány, nejčastěji s pomocí algoritmu
argmax, souřadnice klíčových bodů. V případě vícero osob je pak třeba tyto body
spojit do jednotlivých osob.

\subsubsection*{Regrese}

Využití heatmap je velmi přesné, nicméně z důvodu nutnosti provádění dvou
sekvenčních výpočtů je trochu pomalé. Heatmapy také komplikují proces
trénování, jelikož je třeba spolu s trénovacími daty dodat modelu i heatmapy.
Některé algoritmy se proto snaží formulovat úlohu jako regresi vedoucí přímo k
souřadnicím klíčových bodů. Tento přístup je ve své podstatě trochu méně
přesný, nicméně je rychlejší.

Vůbec první algoritmus pro detekci pózy využívající hluboké učení, DeepPose
\cite{deeppose}, který byl vytvořen v roce 2014 společností Google, používal
právě regresi. Také algoritmus YOLO používá regresi pro určení souřadnic
klíčových bodů, nicméně detekce je prováděná pro detekované objekty, nikoliv
nad celým vstupním obrazem \cite{yolo-pose}.

% , a je součástí poupraveného modelu pro detekci objektů – v posledních
% vrstvách sítě je kromě regrese definující bounding box a klasifikace určující
% třídu prováděna regrese pro určení klíčových bodů.

\section{Detekce objektů a osob v obraze}
\label{sec:obj_det}

Detekce osob se v podstatě může generalizovat na detekci objektů v obraze.
Detekci objektů v obraze definujeme jako úlohu, kdy ve vstupním obrázku určíme
lokalizaci a třídu všech hledaných objektů. Lokalizace je většinou
reprezentována jako souřadnice obdélníku ohraničujícího daný objekt, tzv.
bounding box.

V kapitole \ref{chap:CNN} byla popsána základní architektura konvolučních
neuronových sítí, ta se ale většinou v praxi používá pro klasifikaci obrázků,
nikoliv pro detekci objektů – algoritmus tedy pouze určí, o jakou třídu objektu
se jedná, a ideálně potřebuje, aby objekt vyplňoval celý vstupní obraz.
Teoreticky by bylo možné detekci formulovat jako regresní problém a natrénovat
takovou síť, která by pomocí několika konvolučních vrstev následovaných
několika plně propojenými vrstvami byla schopna predikovat lokalizaci a třídu
všech objektů v obraze \cite{szegedy}. Problém detekce je ale velice komplexní
a také by vyžadoval velice komplexní síť – více vrstev s mnoha filtry, resp.
neurony. Jak již ale bylo zmiňováno, komplexnost sítě zvyšuje její nároky na
výpočetní výkon a komplikuje nebo úplně znemožňuje její trénování s ohledem na
pravděpodobnost přetrénování.

Snahou tedy bylo najít metody, které poupraví architekturu sítě tak, aby byla
schopna efektivní detekce objektů. Většina těchto metod se nějakým způsobem
snaží rozdělit vstupní obrázek na menší části, ty následně jednak klasifikovat,
a tedy určit, zda se v dané lokalitě vyskytuje objekt, popřípadě pomocí regrese
určit jeho přesnou lokalizaci. Rozdělení může být provedeno přímo na vstupním
obrázku nebo na mapě příznaků v rámci sítě.

Metoda sliding window (klouzavé okno), která aplikuje hrubou sílu a projde
veškeré možné oblasti, je samozřejmě velice neefektivní, a tak se další metody
snaží buď najít pouze oblasti, ve kterých je pravděpodobné, že se nějaký objekt
nachází – dvoufázový přístup, anebo rozdělí obrázek do mřížky – jednofázový
přístup.

Jelikož jsou tyto metody základem pro většinu detektorů klíčových bodů, bude
nyní několik základních popsáno.

% \subsection{Sliding window}

% Jednou z prvních takových metod byl tzv. sliding window (klouzavé okno), který
% aplikuje hrubou sílu. Vstupní obrázek se postupně projíždí oknem o fixní
% velikosti. Vznikne tak množina pokrývající každou možnou lokaci objektů. Na
% tyto oblasti se pak aplikuje klasifikační algoritmus. Postup se opakuje pro
% několik velikostí okna, aby se detekovaly objekty různé velikosti.

% Tento postup je ale velice pomalý, jelikož je pro každý obrázek zvolený velký
% počet oblastí, pro které je třeba provést klasifikaci popřípadě regresi. Navíc
% je většina těchto oblastí prázdná, a dochází tak k plýtvání výpočetním výkonem.
% Algoritmus se také potýká s překrývajícími se objekty.

% Další metody se tedy snaží redukovat počet oblastí, na které se aplikuje
% klasifikace, tak, že se vybere pouze oblasti, které pravděpodobně budou
% obsahovat nějaký objekt.

\subsection{Dvoufázový přístup}

\subsubsection*{R-CNN}
Prvním algoritmem, který efektivně zredukoval počet oblastí pro klasifikaci,
byl algoritmus R-CNN (Region-based Convolutional Network) \cite{r-cnn}. Tento
algoritmus nejprve použil některou z dostupných metod (autoři použili selective
search) pro vygenerování navržených oblastí (region proposals), které
pravděpodobně obsahují nějaký objekt. Tyto metody jsou nezávislé na třídě
objektů. Algoritmus tedy vygeneruje zhruba 2000 oblastí, vzniklé obrázky jsou
následně upraveny na velikost požadovanou CNN v další fázi. CNN extrahuje z
dané oblasti mapu příznaků, na jejímž základě plně propojené vrstvy predikují
třídu objektu popřípadě jeho bounding box.

Problémem R-CNN je, že výběr oblasti a jejich následná klasifikace jsou
nezávislé úlohy a jsou nezávisle trénovány. Detekce objektu je také poměrně
pomalá, protože je extrakce příznaků prováděna pro všechny oblasti zvlášť. Tyto
problémy se snaží řešit další upravené verze R-CNN.

\subsubsection*{Fast R-CNN}
První z nich je Fast R-CNN \cite{fast-r-cnn}, která je upravená tak, aby bylo
možné provádět trénování v jednom kroku. Také extrahuje příznaky pro celý
vstupní obraz najednou, pomocí selective search pak identifikuje oblasti zájmu
(ang. region of interest – RoI), které následně použije pro klasifikaci a
regresi. Tato metoda je přesnější a asi desetkrát rychlejší než původní R-CNN.

\subsubsection*{Faster R-CNN}
Další algoritmus, Faster R-CNN \cite{faster-r-cnn}, nahrazuje metodu selective
search vlastní, plně konvoluční sítí RPN (region proposal network).
Zefektivňuje tak proces trénování, výsledná síť je také rychlejší a přesnější
než Fast R-CNN.

\subsection{Jednofázový přístup}

Jednofázový přístup se snaží najít řešení, ve kterém není nutné hledat navržené
oblasti, ale provést klasifikaci a regresi na předem dané množině oblastí,
obvykle určené mřížkou.

\subsubsection*{YOLO}
Prvním takovým algoritmem byl YOLO (také YOLOv1, z ang. you only look once)
\cite{yolo}. Ten, v původní verzi, rozdělí vstupní obraz do pevně dané mřížky
velikosti $S \times S$ a v každém z těchto polí určí $B$ bounding boxů a jejich
třídu. V původní verzi bylo zvoleno $S = 7$ a $B = 2$.

Obraz je nejprve zpracován pomocí konvolučních vrstev, které extrahují mapu
znaků o velikosti $S \times S \times K$, kde $K$ je počet kanálů. Každý pixel
této mapy představuje jedno pole mřížky. Dále je mapa zpracována plně
propojenými vrstvami, které provádějí nad každým polem mřížky klasifikaci a
regresi, viz obrázek \ref{fig:yolo}.

\begin{figure}[]
    \centering
    \includegraphics[width=0.8\textwidth]{Figures/yolo}
    \caption{Architektura původní verze YOLO \cite{yolo}}
    \label{fig:yolo}
\end{figure}

Každý bounding box je reprezentován souřadnicemi středu a velikosti (šířka a
výška). Dohromady s informací o jistotě detekce bounding boxu (confidence
score) vrátí model $5$ informací o každém bounding boxu. Pro každé pole mřížky
pak určí společnou informaci o třídě všech objektů v daném poli. Pokud objekt
není detekován, třída indikuje pozadí a souřadnice bounding boxu jsou
ignorovány. Velikost výstupního vektoru je tedy $7 \times 7 \times (2 * 5 +
    C)$, kde $C$ je počet definovaných tříd – v původní verzi pouze 20.

Tento algoritmus, navržený v roce 2015 J. Redmonem et al., byl revolučně
rychlý, zároveň v porovnání s jinými real-time detekčními algoritmy dosahoval i
slušné přesnosti. Nicméně byl velice citlivý na velikost objektu a přesnost
detekce, zejména u menších objektů, byla horší než u dvoufázových algoritmů.

Další verze algoritmu YOLO přinesly postupná vylepšení ve formě optimalizace
trénování a architektury. YOLOv2 \cite{yolo9000} zavedl mj. trénování na
několika měřítkách a byl natrénován s 9000 třídami (proto také nazýván
YOLO9000). YOLOv3 \cite{yolov3} přinesl mj. detekci na několika měřítkách.
Postupně byla také zvětšována mřížka a měnila se použitá architektura CNN sítě
sloužící pro extrakci příznaků pro jednotlivá pole mřížky. Postupně také byly
přidávány další funkce jako je segmentace, detekce pózy či sledování objektů
(ang. tracking).
%todo cite

V 2020 roce firma Ultralytics poprvé implementovala YOLO s využitím populární
knihovny PyTorch (YOLOv5), což umožnilo snadnější využití YOLO v praxi. Firma
Ultralytics také vytvořila framework pro použití různých verzí YOLO (YOLOv3 a
novější). Také pracuje na dalších vylepšeních a optimalizacích. Konkrétně
vytvořila YOLOv5 (2020), YOLOv8 (2023) a YOLOv11 (2024). Tyto verze nicméně
nejsou podloženy odbornými články, někteří je tak považují za neoficiální
verze.

\subsubsection*{SSD}
Dalším populárním algoritmem, který používá jednofázový přístup, je SSD (z ang.
single shot detector) \cite{szegedy:ssd}. Ten rozdělí vstupní obraz do několika
mřížek o různé velikosti. Postup je takový, že nejprve obraz projde konvoluční
sítí, konkrétně sítí VGG16, která extrahuje mapu příznaků. Tu se postupně
dalšími konvolučními vrstvami zmenšuje, výstup každého stádia zmenšení,
reprezentující mřížku dané velikosti, se spolu s původní mapou dále zpracovává
plně propojenou sítí.
\begin{figure}[]
    \centering
    \includegraphics[width=0.8\textwidth]{Figures/ssd.png}
    \caption{Architektura SSD \cite{szegedy:ssd}}
    \label{fig:ssd}
\end{figure}

Výstupní bounding boxy nejsou, jako v případě YOLO, pouze výsledkem regrese,
ale pro každé pole dané mřížky je definováno několik výchozích oblastí (ang.
default box), ze kterých jsou vybrány ty, které obsahují objekt. K nim je
predikována třída objektu a posun i změna velikosti výchozí oblasti,
upřesňující výsledný bounding box.

V době svého vzniku byl SSD rychlejší a přesnější než YOLO, ale novější verze
YOLO jej už předběhly. Nicméně některé principy SSD, jako výchozí oblasti či
použití různých měřítek, byly převzaty do novějších verzí YOLO.
%todo cite

\section{Charakteristiky vybraných implementací pro detekci pózy}

V této sekci bude popsáno několik populárních algoritmů (a jejich implementací)
pro detekci pózy, zejména se zaměřením na jejich rychlost, přesnost a specifika
architektury.

Velká část algoritmů byla zamítnuta a neproběhlo ani jejich testování.
Nejčastějším důvodem zamítnutí některého z algoritmů je, že schází jeho volně
dostupná, aktualizovaná implementace. Tvůrci algoritmů většinou investují čas a
zdroje pro vývoj algoritmu a natrénování modelu (často pro akademické účely),
pak ale neinvestují do jeho údržby. Zejména v prostředí Pythonu, kde se
neustále vyvíjejí nové knihovny a zpětná kompatibilita starších verzí není
zaručena, je pak obtížné použít takovéto řešení ve svém projektu, aniž by bylo
nutné investovat další čas do pochopení zdrojového kódu a jeho úpravy. Jak již
bylo zmíněno na počátku kapitoly, implementace algoritmu od nuly by vyžadovala
velké množství času a zdrojů (zejména výpočetních), hlavně by také byla
potřebná hlubší znalost problematiky. Někdy je možné řešení použít za cenu
kompromisu ve formě použití starších verzí knihoven či Pythonu, může to ale
představovat bezpečnostní rizika. Také některé algoritmy jsou sice volně
dostupné, ale jejich implementace jsou součástí komerčních produktů, např.
OpenPose, jež je mj. součástí produktu Viso Suite od viso.ai
\cite{visoai_openpose}.

\subsection{DeepPose}

DeepPose je historicky první algoritmus pro detekci pózy využívající hluboké
učení. Vyvinuli jej Alexander Toshev a Christian Szegedy ze společnosti Google
v roce 2014 \cite{deeppose}. Algoritmus předpokládá, že se ve vstupním obraze
nachází pouze jedna osoba. Síť se snaží v jednom kroku pomocí regrese jak
detekovat osobu, tak i její klíčové body. Jelikož je těžké takto dosáhnout
velmi přesných výsledků, algoritmus používá další fázi, která pomocí regrese
provádí posun bodů k přesnějším výsledkům. Tato fáze je aplikována opakovaně,
kaskádně se tak zvyšuje přesnost detekce.

Při svém vzniku byl DeepPose revoluční, nicméně v porovnání s dnešními řešeními
je poměrně pomalý a nepřesný. Nicméně položil základ pro využití hlubokého
učení v oblasti detekce pózy.

\subsection{OpenPose}

OpenPose \cite{openpose} je typicky příklad přístupu zdola nahoru. Jeho výhodou
je ale možnost vyhledání více osob v jednom snímku. Tento algoritmus, který
vyvinuli v roce 2019 Zhe Cao et al., nejprve pomocí CNN vytvoří heatmapu pro
každý typ klíčového bodu. Pro spojení bodů do jednotlivých osob využije pole
propojení klíčových bodů (ang. part affinity field – PAF). PAF je mapa
vytvořená pro každou končetinu (myšleno obecně spojení dvou klíčových bodů),
která v oblasti dané končetiny obsahuje hodnoty určující směr z jednoho bodu do
druhého. Pokud jsou pak spojené informace z heatmap a z PAF, je možné poměrně
jednoznačně zkompletovat jednotlivé klíčové body do celých postav. Stejně jako
heatmapy, jsou i PAF součástí trénovacích dat.

\subsection{OpenPifPaf}

Algoritmus OpenPifPaf \cite{openpifpaf}, vyvinutý v roce 2021 Svenem Kreissem
et al., je v podstatě vylepšenou verzí OpenPose. Jeho název je odvozen od dvou
stavebních kamenů: PIF (Part Intensity Field) – pole intenzity klíčových bodů,
a PAF (Part Affinity Field) – pole propojení klíčových bodů. PIF je rozšířením
heatmap, kdy kromě intenzity pravděpodobnosti klíčového bodu obsahuje i jeho
posun, zaručující přesnější lokalizaci bodu, a odhadovanou velikost dané části
těla. PAF v OpenPifPaf je také podobný tomu v OpenPose, navíc ale indikuje
kromě směru i velikost dané končetiny, což umožňuje lepší prostorové zachycení
pózy.

Dalším rozšířením oproti OpenPose je možnost sledování osob ve videu. Mapa
příznaků, která je výsledkem vstupní CNN, je udržována v mezipaměti, do další
části sítě pak vždy vstupují mapy pro aktuální a předchozí snímek. Výstupem pak
kromě klíčových bodů v každém snímku a jejich propojení tvořící kostru, jsou i
propojení mezi klíčovými body z jednotlivých snímků, viz
\ref{fig:pipaf-tracking}. Algoritmus si pak udržuje ID sledovaných osob, pokud
je k dříve nalezené osobě nalezena nová pozice, je jí přiřazeno stejné ID.
Pokud je nalezena nová osoba, je jí přiřazeno nové ID.

\begin{figure}[]
    \centering
    \includegraphics[height=0.2\textheight]{Figures/skeleton_forward2.png}
    \caption{Vizualizace  sledování osoby mezi dvěma snímky v OpenPifPaf \cite{openpifpaf}}
    \label{fig:pipaf-tracking}
\end{figure}

U OpenPifPaf je k dispozici výběr několika páteřních modelů, jako je ResNet50
či ShuffleNet v různých variantách a velikostech. Je tak možné zvolit model,
který je kompromisem mezi výkonem a přesností, v závislosti na konkrétních
požadavcích aplikace.

\subsection{MediaPipe – BlazePose}

MediaPipe je framework vyvinutý společností Google, umožňující jednoduchou
integraci různých technik strojového učení. Obsahuje různé algoritmy pro řešení
úloh jako detekce objektů, segmentace či detekce klíčových bodů (tváře či
pózy). MediaPipe je optimalizován pro mobilní zařízení a webové aplikace.
Detekce pózy v tomto frameworku je postavená na algoritmu BlazePose.

BlazePose implementuje přístup shora dolů, detekuje tedy nejprve RoI, ve
kterých detekuje osobu a její pózu. Nativně podporuje pouze jednu osobu ve
snímku. Ve videu ale v rámci optimalizace neprovádí detekci RoI pro každý
snímek, pouze pokud v aktuální RoI již není detekována osoba. Výhodou tohoto
algoritmu je, že detekuje 33 bodů v postavě, což je podstatně více, než většina
ostatních algoritmů, umožňuje tak přesnější analýzu některých situací, např.
podle natočení tváře, dlaní či stop.

Framework MediaPipe implementuje BlazePose spolu s detekcí více osob (v
první fázi používá detektor objektů). Výhodou tohoto frameworku je jeho kontinuální
vývoj a jednoduchost integrace. Nevýhodou ale je, že pro systémy Windows není
implementována podpora GPU. Jelikož výsledný produkt bude spouštěn primárně na
Windows zařízeních, je tato vlastnost rozhodující. Model je dostupný ve třech
velikostních variantách: $Lite$, $Full$ a $Heavy$.

\subsection{YOLO}

Od vydání YOLOv7 v roce 2022 integruje framework YOLO i detekci pózy. Oficiální
článek Chien-Yao Wanga et al. \cite{yolov7} sice neobsahoval tuto funkčnost,
ale oficiální implementace zahrnula i implementaci YOLO-Pose \cite{yolo-pose}.
Obecně detekce pózy v YOLO kombinuje přístup shora dolů a zdola nahoru.
Algoritmus sice vyhledává klíčové body spolu s bounding boxy osob, nicméně vše
v jednom kroku. Samotná detekce klíčových bodů využívá regresi, což
zjednodušuje proces trénování, jelikož není třeba tvořit heatmapy.

Architektura použitá v YOLO-Pose se ale liší od architektury používané v
pozdějších verzích. V YOLO-Pose jsou na konci řetězce umístěny hlavy pro různá
měřítka, jejich výstupem jsou bounding boxy a klíčové body, oba tvořené spolu.
V pozdějších verzích je architektura YOLO koncipována univerzálněji pro různé
úlohy. Obsahuje tak tři fáze\cite{yolov11}: páteř (ang. backbone), která
extrahuje mapu příznaků, krk (ang. neck), který přizpůsobuje mapu příznaků pro
různá měřítka, a hlavy (ang. head), které paralelně zpracovávají výstupy pro
různé úlohy, viz obrázek \ref{fig:yolov11} . Bounding box a klíčové body jsou
tedy sice generovány paralelně a teoreticky nezávisle, nicméně s ohledem na
proces trénování a postprocessing se v praxi navzájem výrazně ovlivňují.

\begin{figure}[]
    \centering
    \includegraphics[height=0.2\textheight]{Figures/yolo_v11.png}
    \caption{Architektura YOLOv11 \cite{yolov11}}
    \label{fig:yolov11}
\end{figure}

Nejnovější verze YOLO také podporují kombinaci detekce klíčových bodů a
sledování osob. Model tedy kromě klíčových bodů vrací ID dané osoby, pomocí
kterého lze spojit danou postavu s předchozími snímky. Je tak možné efektivně
analyzovat pohyby jednotlivých osob.

Jednou z výhod framevorku YOLO, zejména verzí vyvíjených firmou Ultralytics, je
široká škála velikostí modelů. Každý model je dostupný v pěti variantách:
$Nano$, $Small$, $Medium$, $Large$, $Xlarge$.

\subsection{Torchvision}

Torchvision je knihovna, která je součástí frameworku PyTorch. Obsahuje různé
nástroje pro strojové vidění, jako je detekce objektů či segmentace. Její
součástí je i předtrénovaný model pro detekci pózy, který implementuje
algoritmus Keypoint R-CNN \cite{keypoint-rcnn}.

Algoritmus Keypoint R-CNN je založen na stejné myšlence jako Mask R-CNN
\cite{mask-r-cnn}, a tedy rozšíření Faster R-CNN o další hlavu, která v případě
Mask R-CNN provádí segmentaci, v případě Keypoint R-CNN detekuje klíčové body
\cite{keypoint-rcnn}. V algoritmu je také upravená pooling vrstva, která
zajišťuje, že se výstupy algoritmu shodují se vstupy s přesností pixelu.

Tato implementace bude testována zejména z důvodu její jednoduchosti použití a
integrace do frameworku PyTorch, který bude použitý i v další části řešení. Na
druhou stranu je oproti jiným frameworkům, jako je YOLO či MediaPipe, k
dispozici pouze jeden model, nikoliv více velikostních variant.

\section{Testování a porovnání vybraných algoritmů pro detekci pózy}

Pro testování byly vybrány čtyři algoritmy pro detekci pózy, zejména na základě
jednoduchosti jejich implementace, aktualizované podpory a požadovaných funkcí.
Testovány tedy budou algoritmy Torchvision, OpenPifPaf, MediaPipe BlazePose a
YOLO v nejnovější verzi 11. Testy byly provedeny na 25 videích ze stejného
datasetu jako byl použit později pro trénování algoritmu pro analýzu klíčových
bodů. Testování probíhalo na počítači s procesorem Intel Core i7, 32 GB RAM a
grafickou kartou NVIDIA GeForce RTX 3080 s 10 GB VRAM.

Algoritmy Torchvision a OpenPifPaf byly testovány s použitím GPU, MediaPipe
pouze s využitím CPU, jelikož v prostředí Windows nemá podporu GPU. Algoritmus
YOLO byl testován na GPU a CPU, pro porovnání jeho výkonu v různých podmínkách.
Algoritmus YOLO byl také vyzkoušen na GPU s využitím funkce sledování.

Výstupem testování pro daný algoritmus a jeho variantu je průměrná doba
zpracování jednoho snímku a videa s vykreslením klíčových bodů. Tato videa
dovolují ověřit schopnost detekce klíčových bodů v různých situacích a její
přesnost.

Následující část se věnuje výsledkům testování jednotlivých algoritmů. Pro
každý algoritmus budou porovnány jednotlivé varianty s ohledem na rychlost a
přesnost a budou vyhodnoceny jeho výhody či nevýhody oproti ostatním
algoritmům. Nakonec bude zvolen model, který bude použitý v další části vývoje.

\subsection{Výkonové požadavky}
\label{sec:performance_requirements}

Při výběru algoritmu je třeba s ohledem na práci v reálném čase brát v úvahu
zejména výkon detekčního algoritmu. Bezpečnostní kamery mají obvykle snímkovou
frekvenci od 15 do 30 snímků za sekundu (FPS), ideální by tedy bylo, aby
konečný program byl schopen pracovat s frekvencí alespoň 30 FPS. Zároveň se
předpokládá, že v prostředí, kde bude program nasazen, bude dostupná grafická
karta.

V této fázi již bylo zkoušeno trénování neuronové sítě pro klasifikaci pózy, a
bylo ověřeno, že i v případě hlubších a komplexnějších sítí je dosaženo doby
inference v řádu nižších jednotek milisekund. Hlavní vliv na výslednou rychlost
programu tedy bude mít hlavně detekční algoritmus, který musí zpracovávat
mnohem větší objem dat – stovky tisíc až miliony pixelů oproti např. 17
klíčovým bodům v případě klasifikačního algoritmu.

\subsection{OpenPifPaf}

Algoritmus OpenPifPaf byl zkoušen v několika variantách, postavených na síti
\textit{ResNet 50} a \textit{ShuffleNet V2} \cite{shufflenetv2}. V tabulce
\ref{tab:openpifpaf_performance} je vidět, že většina variant ani zdaleka
nedosahuje požadovaného výkonu.

Tento algoritmus je poměrně robustní z pohledu světelných podmínek a rozlišení
či rozmazaní obrazu, jinak ale dosahuje nejhorší přesnosti ze všech testovaných
algoritmů. Kromě nedetekování části těla, které nejsou vidět (jsou např.
schovány za jinou částí těla), totiž často nedetekuje člověka vůbec, nejčastěji
pak když člověk padá nebo leží, což jsou situace pro nás stěžejní. Hlavně tento
problém vystupuje ve variantě $resnet50$ a $shufflenetv2k16$, jsou tak pro
řešení nepoužitelné. Varianty $shufflenetv2k30$ a $tshufflenetv2k30$ by sice s
ohledem na kvalitu výsledků použitelné byly, nicméně je jejich výkon příliš
nízký.

\begin{table}[htbp]
    \centering
    \caption{Porovnání výkonu modelu OpenPifPaf}
    \label{tab:openpifpaf_performance}
    \begin{tabular}{|l|l|l|l|}
        \hline
        \textbf{Verze}   & \textbf{inference [ms]} & \textbf{Frekvence [FPS]} \\
        resnet50         & 49,2                    & 20.324                   \\ \hline
        shufflenetv2k16  & 31,3                    & 31.910                   \\ \hline
        shufflenetv2k30  & 58,6                    & 17.070                   \\ \hline
        tshufflenetv2k30 & 70,0                    & 14.281                   \\ \hline
    \end{tabular}
\end{table}

\subsection{MediaPipe BlazePose}

Algoritmus BlazePose z knihovny MediaPipe byl testován ve třech variantách:
$Lite$, $Full$ a $Heavy$. Ve všech variantách tento algoritmus dosahoval velmi
přesných výsledků, asi nejlepších ze všech testovaných algoritmů. Na rozdíl od
jiných algoritmů totiž, pokud detekoval osobu, vždy velmi přesně označil
všechny její klíčové body, v rámci možností i ty, které byly hůře viditelné
(např. schovány za jinou částí těla). Na obrázku \ref{fig:ym_comparison} je
vidět rozdíl v přesnosti detekce klíčových bodů mezi nejmenší variantou
BlazePose a druhou největší variantou YOLO, kdy BlazePose dosahuje mnohem větší
přesnosti, v tomto příkladě zejména co se týče detekce nohou.

\begin{figure}
    \centering
    \includegraphics[width=0.4\textwidth]{Figures/pose_tests/mh1.png}
    \includegraphics[width=0.4\textwidth]{Figures/pose_tests/yl1.png}
    \caption{Porovnání přesnosti \textit{MediaPipe Lite} (vlevo) a \textit{YOLO Large} (vpravo)}
    \label{fig:ym_comparison}
\end{figure}

S ohledem na to, že nemáme k dispozici grafickou akceleraci, je jeho výkon
velmi dobrý, viz tabulka \ref{tab:mediapipe_performance}. Verze $Lite$ a $Full$
by tak mohla být v našem řešení použitelná, což by nám také dávalo možnost
nasazovat výsledný program v mobilních zařízeních či jiných systémech bez
grafické karty.

\begin{table}[htbp]
    \centering
    \caption{Porovnání výkonu modelu MediaPipe BlazePose}
    \label{tab:mediapipe_performance}
    \begin{tabular}{|l|l|l|l|}
        \hline
        \textbf{Verze} & \textbf{inference [ms]} & \textbf{Frekvence [FPS]} \\
        \hline
        lite           & 25.5                    & 39.239                   \\ \hline
        full           & 30.9                    & 32.405                   \\ \hline
        heavy          & 68.2                    & 14.655                   \\ \hline
    \end{tabular}
\end{table}

Algoritmus si ale velice špatně radí s horšími světelnými podmínkami či menším
rozlišením obrazu. Ve většině případů sice detekuje klíčové body velmi přesně,
pokud je ale osoba hůř viditelná, nedetekuje ji vůbec. Tento problém se
projevuje ve všech variantách podobně. Jelikož bude výsledný program nasazován
spíše právě v podmínkách s horším osvětlením a ve větší vzdálenosti od osob,
pravděpodobně se pro aplikaci nebude hodit.

\subsection{Torchvision Keypoint R-CNN}

Torchvision Keypoint R-CNN nedosáhla ani dostatečného výkonu, viz tabulka
\ref{tab:torchvision_performance}, ani kvalitních výsledků. Podobně jako
\textit{OpenPifPaf} má totiž problém, když osoba padá anebo leží. V tomto
případě osobu často detekuje, ale naprosto ztrácí přesnost detekovaných
klíčových bodů, nejčastěji záměnou jednotlivých bodů, viz obrázek
\ref{fig:torchvision_bad}. Zároveň oproti BlazePose nemá takový problém s
horšími světelnými podmínkami a menším rozlišením obrazu.

\begin{figure}[]
    \centering
    \includegraphics[width=0.2\textwidth]{Figures/pose_tests/torchvision_bad.png}
    \caption{Příklad špatné detekce bodů v modelu Torchvision Keypoint R-CNN}
    \label{fig:torchvision_bad}
\end{figure}

\begin{table}[htbp]
    \centering
    \caption{Výkon modelu Torchvision Keypoint R-CNN}
    \label{tab:torchvision_performance}
    \begin{tabular}{|l|l|}
        \hline
        \textbf{inference [ms]} & \textbf{Frekvence [FPS]} \\
        \hline
        50.3                    & 19.897                   \\ \hline
    \end{tabular}
\end{table}

\subsection{YOLO}

Algoritmus YOLO ve verzi 11 byl testován v pěti variantách: $Nano$, $Small$,
$Medium$, $Large$ a $Xlarge$. Všechny varianty byly testovány na GPU i CPU.

V tabulce \ref{tab:yolo_performance} je vidět, že na CPU dosahuje tento
algoritmus poměrně špatných výsledků. Jediný použitelný by mohl být v takové
situaci model $Nano$. YOLO je ale velice kvalitně optimalizováno pro grafické
karty, lze tak pozorovat, že na GPU je výkon výrazně vyšší. Pro potřeby řešení
by tak byly použitelné prakticky všechny varianty.

Jelikož pro analýzu více osob v jednom snímku je potřeba, zejména v případě
použití rekurentní neuronové sítě, jednotlivé osoby od sebe oddělit a
identifikovat i mezi snímky, bude velmi užitečná funkce sledování objektu.
Proto byl otestován algoritmus YOLO i s touto funkcí. Jak je vidět v tabulce
\ref{tab:yolo_performance}, je výkon sice horší než bez sledování, pořád ale
tři menší varianty dosahují frekvence větší než $30$ FPS.

\begin{table}[htbp]
    \centering
    \caption{Porovnání výkonu modelu YOLO}
    \label{tab:yolo_performance}
    \begin{tabular}{|c|l|l|l|}
        \hline
                                           & \textbf{Verze} & \textbf{inference [ms]} & \textbf{Frekvence [FPS]} \\
        \hline\hline
        \multirow{3}{*}{CPU}               & nano           & 32.5                    & 30.749                   \\ \cline{2-4}
                                           & small          & 53.9                    & 18.563                   \\ \cline{2-4}
                                           & medium         & 114.1                   & 8.763                    \\ \cline{2-4}
                                           & large          & 143.4                   & 6.973                    \\ \cline{2-4}
                                           & xlarge         & 833.2                   & 1.200                    \\ \hline\hline
        \multirow{3}{*}{GPU}               & nano           & 15.1                    & 66.323                   \\ \cline{2-4}
                                           & small          & 15.2                    & 65.972                   \\ \cline{2-4}
                                           & medium         & 17.4                    & 57.500                   \\ \cline{2-4}
                                           & large          & 24.4                    & 41.026                   \\ \cline{2-4}
                                           & xlarge         & 24.4                    & 41.005                   \\ \hline\hline
        \multirow{3}{*}{GPU se sledováním} & nano           & 27.4                    & 36.500                   \\ \cline{2-4}
                                           & small          & 27.7                    & 36.148                   \\ \cline{2-4}
                                           & medium         & 29.5                    & 33.882                   \\ \cline{2-4}
                                           & large          & 37.5                    & 26.664                   \\ \cline{2-4}
                                           & xlarge         & 40.5                    & 24.695                   \\ \hline
    \end{tabular}
\end{table}

Všechny varianty algoritmu YOLO dosahují poměrně kvalitních výsledků. I v
horších světelných podmínkách vždy detekují osobu, a víceméně přesně určí její
klíčové body. Obecně je ale vidět, že je model trochu méně robustní (než např.
MediaPipe) v situacích, kdy není dobře vidět některá končetina – je např.
schovaná za jinou částí těla, anebo ve specifických pózách – např. když je
osoba v dřepu nebo je v obraze natočená vzhůru nohama. V takovýchto případech
dosahuje menší přesnosti pro jednotlivé body – detekuje jiné natočení končetiny
nebo v extrémních případech špatně vyhodnotí natočení celé postavy. Špatně
viditelné části těla pak často vůbec nedetekuje.

Z pohledu přesnosti je zde výrazně vidět vliv velikosti modelu na přesnost
detekce. Varianta $Nano$ ve výše zmíněných situacích někdy detekuje body zcela
špatně, a je tak prakticky nepoužitelná. Varianta $Small$ je znatelně lepší,
pořád ale v horších podmínkách vyhodnocuje mnoho části těla špatně – např.
zamění nohy. Varianta $Medium$ je už výrazně lepší. Není sice ideální, ve valné
většině ale vyhodnotí všechny části těla správně i když ne z přesností několika
pixelů. Varianty $Large$ a $Xlarge$ jsou pak už velmi přesné, projevuje se to
ale znatelně menší rychlostí.

\begin{figure}
    \centering
    \includegraphics[width=0.18\textwidth]{Figures/pose_tests/y_n.png}
    \includegraphics[width=0.18\textwidth]{Figures/pose_tests/y_s.png}
    \includegraphics[width=0.18\textwidth]{Figures/pose_tests/y_m.png}
    \includegraphics[width=0.18\textwidth]{Figures/pose_tests/y_l.png}
    \includegraphics[width=0.18\textwidth]{Figures/pose_tests/y_x.png}
    \caption{Porovnání přesnosti variant modelu YOLO v situaci s netypickým natočením postavy. Zleva: $Nano$, $Small$, $Medium$, $Large$, $Xlarge$}
    \label{fig:y_comparison}
\end{figure}

Je zajímavé pozorovat, že i když je algoritmus BlazePose v mnoha situacích
mnohém přesnější, než i větší varianty YOLO, viz obrázek
\ref{fig:ym_comparison}, při horší viditelnosti, kdy i nejmenší varianty YOLO
detekují osobu, BlazePose zcela selhává. Ve snímku na obrázku
\ref{fig:y_comparison} například nedetekoval BlazePose osobu vůbec.

\subsection{Shrnutí a výběr modelu}

Z výše popsaného testování bylo zjištěno, že modely OpenPifPaf a Torchvision
jsou natrénované spíše pro detekci postavy, když je osoba v běžnějších pózách,
jako je ve stoje či chůze. Stejně i u menších variant YOLO přesnost prudce
klesá v méně typických pózách či natočeních v obraze. Jelikož ale je navrhované řešení zaměřeno právě na detekování spíše netypické postavy, je pro nás důležité
detekovat pozici ve všech situacích.

Naopak algoritmus BlazePose je velmi přesný, strádá ale při horší viditelnosti
osoby. Nehodí se tak pro naše využití s kamerami s horším rozlišením a vysokou
vzdáleností od osob. Ostatní algoritmy jsou v tomto ohledu mnohem robustnější,
nejlépe si s horšími podmínkami poradí algoritmus YOLO.

Optimální cesta se tedy zdá být algoritmus YOLO ve verzi $Medium$, který dosahuje
dostatečné rychlosti i při sledování osob, zároveň dostatečně přesně detekuje
pózy ve všech pozicích i podmínkách. Pokud by nebyla pro analýzu pózy použitá
rekurentní neuronová síť, nemuselo by být nutné používat funkci sledování a
bylo by tak možné použít i větší variantu YOLO, což by mohlo zlepšit přesnost
detekce.

\endinput

\chapter{Závěr}
\label{chap:Conclusion}

Cílem práce bylo navrhnout a implementovat řešení pro detekci pádu osoby v toku
obrázku v reálném čase. V první části byly popsány teoretické základy použitých
technologií, se zaměřením na neuronové sítě.

V praktické části byl popsán postup návrhu a vývoje výsledného programu. Řešení
jsme dosáhli spojením volně dostupného modelu pro detekci pózy ve formě
klíčových bodů a neuronové sítě pro klasifikaci těchto bodů do třídy
\textit{normální} nebo \textit{upadl}.

Prozkoumali jsme několik detekčních algoritmů – vysvětlili jsme si zásady
funkčnosti detekce klíčových bodů a otestovali jsme jednotlivé algoritmy na
testovacích datech. Pro detekci pózy jsme nakonec zvolili model YOLO11-pose pro
jeho vysokou rychlost, zejména při využití grafické akcelerace, a přesnost i
při horší viditelnosti. Tento model také v sobě integruje funkci sledování
osob, která nám umožňuje provádět následnou analýzu pro každou osobu zvlášť i v
kontextu více snímků.

Pro klasifikaci pózy jsme zkoušeli natrénovat modely postavené na třech  
architekturách: dopředná neuronová síť a dvě rekurentní architektury – LSTM a
GRU. LSTM síť se nám nepodařilo efektivně natrénovat, pravděpodobně pro
nedostatek dat a její nevhodnost pro náš problém.

Dopřednou síť a GRU síť jsme po nalezení optimální konfigurace a natrénování
otestovali na testovacích datech. Následně jsme je použili v implementaci
detektoru pádu. Ten jsme dále v obou verzích otestovali na videích z testovací
sady. Dopředná síť se ukázala jako znatelně méně přesná, zejména při méně
kvalitní detekci pózy v situacích s horší viditelností. Také projevovala
falešné pozitivní detekce pádu.

Pro klasifikaci pózy jsme tedy zvolili rekurentní architekturu GRU s jednou
rekurentní a jednou plně propojenou vrstvou, obě velikosti $64$. Tuto síť se
nám podařilo natrénovat na přesnost přes 96\% na testovacích datech, zároveň se
ukázala jako velice stabilní a přesná i při detekci pádu v testovacích videích.

Pokud bychom chtěli pokračovat v optimalizaci našeho řešení, nejvíce
prostoru vidíme v množství trénovacích dat. Zejména by bylo vhodné použít videa s
více úhly natočení vůči zemi a s různými vzdálenostmi osoby od kamery. Větší a
rozmanitější dataset by nám umožnil natrénovat i komplexnější architektury,
které by byly přesnější a robustnější. Pro získání více dat by se mohlo nasadit
náš detektor na podnikové kamery, nebo by se pomocí něj dalo zanalyzovat starší
záznamy. V těchto případech by bylo vhodné nastavit nižší prah detekce, abychom
získali i data s vyšší pravděpodobností falešné pozitivity.

Dalším vylepšením by mohlo být zavedení více tříd do analýzy pózy. V textu byla
zmíněna třída \textit{padá}, nicméně bychom mohli také přidat třídy např. pro
pozici v sedě, ve dřepu, či na kolenou. Model by tak mohl lépe rozeznávat
některé situace, které se zdají být pádu podobné, a tak bychom odstranili
potenciální falešné pozitivní detekce pádu.

Zajímavým rozšířením by také mohlo být propojení více kamer z jedné místnosti
pro sledování osob z více úhlů. Dávalo by to možnost zasazení postav do
trojrozměrného prostoru a umožňovalo dosáhnout mnohem kvalitnější detekce
pózy. Nicméně by bylo obtížnější získat pro takové řešení trénovací data.

Pro nasazení detektoru pádu v reálném provoze, tedy na obrazovém toku z
podnikových bezpečnostních kamer, bude ještě třeba vytvořit vhodné rozhraní pro
správné čtení různých protokolů a formátů videa. Pro integraci do komplexnějšího
programu, ve kterém by probíhalo více analýz videa, by bylo vhodné zvážit
jednak jejich paralelní zpracování, jednak sdílení určitých dat mezi
jednotlivými analyzátory.

\endinput

% Seznam literatury
\printbibliography[title={Literatura}, heading=bibintoc]

% Prilohy
\appendix
%\input{Chapters/Appendix1.tex}
%\input{Chapters/Appendix2.tex}

% Priloha vlozena primo do hlavniho LaTeX souboru. Ne vsechny prilohy je nutne mit ve zvlastnich souborech.
%\chapter{Dlouhý zdrojový kód}
%\lstinputlisting[label=src:CppExternal,caption={Dlouhý zdrojový kód v jazyce C++ načtený s externího souboru}]{SourceCodes/ArraySortingAlgorithms.cpp}

\end{document}
