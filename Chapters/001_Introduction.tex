\chapter{Úvod}
\label{sec:Introduction}

Kamerové systémy jsou využívány již mnoho let a jejich využití je stále širší.
Dnes se odhaduje, že celkový počet bezpečnostních kamer ve světě přesahuje
miliardu. Využívány jsou v průmyslu, dopravě, obchodě, veřejných prostorech,
zdravotnictví či domácnostech.

Zpočátku bylo možné video sledovat pouze živě, později, s příchodem videokazet,
bylo možné záznam sledovat až po události. Digitální era a síťové kamery
umožnily přístup ke kamerovým záznamům z libovolného místa na světě. V poslední
době se také začalo nahrazovat živé sledování automatickým zpracováním obrazu a
detekcí událostí s využitím technik umělé inteligence.

Kamerové systémy se používají zejména ve dvou oblastech: zabezpečení (ang.
security), myšleno jako ochrana před úmyslnými hrozbami a protiprávními činy,
jako jsou krádeže, poškozování majetku, či neoprávněný vstup; a bezpečnost
(ang. safety), což zahrnuje ochranu před nehodami a náhodnými hrozbami, jako
jsou pády, požár, úniky nebezpečných látek, či porušování bezpečnostních
předpisů.

Jak již bylo zmíněno, lze kamery využívat jednak pro živé sledování, jednak pro
záznam a jeho analýzu po události. Kamerové záznamy jsou zejména důležité pro
zpětnou analýzu incidentů, důkazní materiál pro soudní spory, zjišťování příčin
nehod, či pro zlepšení bezpečnostních opatření. Živé sledování videa se pak
snaží incidentům přímo předcházet. Bylo však prokázáno, že schopnost lidského
pozorovatele detekovat nebezpečí se velmi snižuje s délkou sledování a s počtem
monitorovaných kamer. Právě proto se s příchodem technik umělé inteligence
začalo využívat automatické zpracování obrazu a detekce hrozeb, nebezpečí, nebo
již probíhajících incidentů v jejích počátcích. Tyto techniky pak úplně
nahrazují lidského pozorovatele, nebo mu pomáhají včas zpozorovat nebezpečí a
zareagovat.

Automatická analýza obrazu je používaná již několik desítek let, většinou ale
spíše pro oblast zabezpečení, než pro bezpečnost. To z toho důvodu, že úlohy,
jako identifikace neoprávněného vstupu, detekce zbraní, rozpoznávání SPZ nebo
podezřelých osob jsou pro algoritmy mnohem jednodušší, než například detekce
pádu, nouzové situace či zdravotního problému. Hlavním problémem těchto
komplexnějších analýz je vysoká falešná pozitivita, kdy je například těžké
rozeznat člověka trénujícího běh od člověka utíkajícího před nebezpečím.
Nicméně rozvoj v oblasti hlubokého učení a konvolučních neuronových sítí, jako
i vývoj a dostupnost hardwaru podporujícího tyto techniky, umožňuje dneska využít
je i pro složitější úlohy.

Ve firmě Linde jsou kamerové systémy používány v mnoha průmyslových provozech,
nicméně chybí ucelený systém pro automatickou analýzu obrazu a detekci různých
druhů nebezpečí. Naším úkolem tedy v budoucnu bude navrhnout a implementovat
modulární systém s možnosti sledování konkrétních nebezpečí na konkrétních
místech. Ty budou zahrnovat například detekci pádu, požáru, zdravotních problémů,
nebo porušování bezpečnostních opatření. Systém pak bude v případě rozpoznání
nějaké hrozby informovat příslušného pracovníka.

V této práci se zaměříme pouze na jednu z těchto úloh, a to na detekci pádu.
Pád může mít různé příčiny, ať už je to zdravotní problém jako ztráta vědomí,
nebo zakopnutí. Někdy se zdá, že samotné zakopnutí je banální problém, nicméně
pokud se na pracovišti nenachází nikdo, kdo by mohl pomoct, a poškozený není
schopen sám přivolat pomoc, může vést takový incident k vážným následkům.

V první části práce se budeme zabývat teoretickými základy, jako jsou obecné
neuronové sítě, konvoluční neuronové sítě a neuronové sítě zaměřené na časové
řady. V dalších kapitolách se zaměříme na detekci osob a odhad jejich klíčových
bodů. Projdeme si různé přístupy a otestujeme různé algoritmy s ohledem na
výkon, možnou hardwarovou akcelerací a preciznost. V další části se budeme
zabývat samotnou detekcí pádu, tedy algoritmem, který na základě odhadnutých
klíčových bodů určí, zda došlo k pádu, či nikoliv. V závěru práce se zaměříme
na otestování výsledného řešení a zhodnocení jeho výkonu.

\endinput