\chapter{Úvod}
\label{sec:Introduction}

Kamerové systémy jsou využívány již mnoho let a jejich využití je stále širší.
Zpočátku bylo možné video sledovat pouze živě, později, s příchodem videokazet,
bylo možné záznam sledovat až po události. Digitální era a síťové kamery
umožnily přístup ke kamerovým záznamům z libovolného místa na světě. V poslední
době se taky začalo nahrazovat živé sledování automatickým zpracováním obrazu a
detekcí událostí s využitím technik umělé inteligence.

Tyto systémy se používají zejména pro dvě oblasti: zabezpečení (ang. security),
myšleno jako ochrana před úmyslnými hrozbami a protiprávními činy, jako jsou
krádeže, poškozování majetku, či neoprávněný vstup; a bezpečnost (ang. safety),
myšleno jako ochrana před nehodami a náhodnými hrozbami, jako jsou pády, požár,
úniky nebezpečných látek, či porušování bezpečnostních předpisů.

Jak již bylo zmíněno, lze kamery využívat jednak pro živé sledování, jednak pro
záznam a jeho analýzu po události. Pro zabezpečení je živé sledování často
dostačující, protože umožňuje rychlou reakci na události, zatímco pro bezpečnost
je důležitější záznam a jeho analýza po události, protože umožňuje zjištění


Většinou se živé sledování či automatizovaná analýza používá pro zabezpečení,
zatímco pro bezpečnost se častěji využívají záznamy a jejich analýza po
události, a to jednak pro zjištění příčin incidentu, jednak pro zlepšení
bezpečnostních opatření a prevence budoucích incidentů. Existují ale i situace, kdy je 

\endinput