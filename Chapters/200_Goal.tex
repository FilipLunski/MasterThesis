\chapter{Analýza problematiky detekce pádu}
\label{sec:Goal}

V této kapitole stanovíme co je přesně naším cílem a zamyslíme se, jak k našemu
problému přistoupit.

Naším úkolem bude v reálném čase z videostreamu detekovat pád osoby. Pád osoby
definujeme jako náhle, neúmyslné klesnutí těla z výškové pozice (např. stání,
chůze nebo sezení) na zem nebo jinou nižší úroveň, přičemž tato osoba nemá
kontrolu nad tímto pohybem. Samozřejmě nejsme vždy schopni úplně dobře
rozeznat, zda se nejedná o úmyslné klesnutí, např. prudké lehnutí.

Dle některých definic (zejména ve zdravotnictví) se o pád nejedná, pokud jde o
důsledek závažné vnitřní příhody (např. mrtvice). V našem případě toto
nerozlišujeme, naopak chceme detekovat jak pády v důsledku ztráty rovnováhy či
vlivem vnějších faktorů (např. zakopnutí, převrácení těžkým předmětem), tak
pády v důsledku akutních události vlivem zdravotních problémů, jako jsou např.
mrtvice, záchvaty, mdloby či jiné důvody ztráty vědomí.

Cílem této práce je navrhnout algoritmus, který bude detekovat, zda je ve
vstupní sekvenci snímku některá osoba, jejíž pozice je klasifikována jako pád.
Hlavním cílem výsledného programu bude alarmovat příslušného pracovníka, pokud
osoba upadne a zůstane v ležící pozici. To nám dá možnost odfiltrovat falešné
alarmy v případě sehnutí čí pokud bude osoba špatně viditelná a algoritmus tak
špatně vyhodnotí její pohyb. Tímto postprocesingem se ale teď nebudeme zabývat,
spíše se zaměříme na samotnou klasifikaci pozice.

Stejně jako u detekce objektů, viz \ref{sec:obj_det}, bychom mohli i pro
detekci pádu vytvořit vhodnou konvoluční síť, která by přímo z obrázku
definovala, zda se jedná o pád nebo ne. U detekce se už dnes sice s ohledem na
pokrok hardwaru tento přístup používá, nicméně se jedná o velmi náročný úkol,
který vyžaduje rozsáhlou optimalizaci, pokročilou architekturu a velké množství
trénovacích dat. Nicméně, pokud by se podařilo takovouto síť natrénovat, mohla
by lépe detekovat některé situace např. podle výrazu tváře.

V našem případě tedy budeme v prvním kroku pomocí vhodné neuronové sítě
detekovat pozici osoby ve formě klíčových bodů, na jejich základě pak další
neuronová síť vyhodnotí, zda se jedná o pád. To úlohu velice zjednoduší,
jelikož místo analyzování tisíců pixelů, budeme analyzovat pár desítek
klíčových bodů. Další výhodou je, že u takového postupu jsme schopní použít
techniky, kdy sledujeme změny pózy v čase, což by bylo mnohem složitější s
jednofázovou konvoluční síti.

Další alternativou by mohlo být pouze detekovat osoby jako objekty, a na základě
jejich bounding boxů určit, zda se jedná o pád. Tento postup by byl jednodušší
na dvou úrovních. Jednak je detekce objektů méně náročná úloha než detekce
pózy, jednak bychom ve druhé fázi analyzovali pouze několik parametrů bounding
boxu (rozměry a velikost) oproti pár desítkám klíčových bodů. Nicméně, pokud se
nad tím zamyslíme, ne vždy vypovídají parametry bounding boxů o pozici člověka.
Tento postup by tak pravděpodobně vedl k mnohem méně přesnému výsledku, než
analýza klíčových bodů, kdy může síť analyzovat takové vzorce jako je např.
délka končetin v pohledu či úhel mezi nimi. 

V další kapitole rozebraná problematika detekce pózy a bude zvolen algoritmus
pro detekci klíčových bodů. Dále se pak budeme zabývat vývojem modelu
detekujícího pád na základě těchto klíčových bodů.

