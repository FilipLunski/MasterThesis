\chapter{Implementace klasifikační neuronové sítě}
\label{chap:ClassificationImplementation}

V této kapitole se zaměříme na implementaci klasifikačního algoritmu pro
analýzu pózy. Nejdříve se podíváme na použité technologie a knihovny, které nám
pomohou s vývojem. Poté se zaměříme na návrh různých architektur a možné
hyperparametry, jejichž vliv na výkon modelu bychom mohli otestovat. V další
kapitole se pak zaměříme na experimenty s těmito architekturami a
hyperparametry a zhodnotíme získané výsledky.

\section{Použité technologie}

\subsection{PyTorch}

Celý projekt byl vyvíjen v prostředí skriptovacího jazyka Python. Samotný vývoj
neuronových sítí probíhal ve frameworku PyTorch. PyTorch je open-source
knihovna pro strojové učení, široce používaná například pro počítačové vidění
či zpracování přirozeného jazyka. Jeho hlavními funkcemi je práce s tenzory
podobnými jako v NumPy, ale s podporou silné akcelerace s využitím GPU, a vývoj
neuronových sítí postavený na vysokoúrovňových stavebních blocích s podporou
automatické derivace pro počítaní gradientů.

Implementace neuronových sítí v PyTorch je velice jednoduchá a díky vysoké míře
abstrakce umožňuje se při vývoji soustředit na samotnou architekturu a design
sítě, nikoliv na detaily implementace. Pro vytvoření nové neuronové sítě stačí
vytvořit třídu, která bude dědit od základní třídy $nn.Module$, v konstruktoru
definovat jednotlivé vrstvy včetně různých regularizačních hyperparametrů. V
metodě $forward$ pak definujeme dopředný průchod sítě. Dále máme možnost
kontrolovat např. výpočet ztrátové funkce či metriky přesnosti.

PyTorch obsahuje také předpřipravené moduly pro GRU či LSTM sítě, včetně
vícevrstvých architektur. Pro tyto moduly definujeme velikost vstupu a velikost
skrytého stavu, dále můžeme definovat počet vrstev či dropout. Výstup z těchto
modulů je pak skrytý stav, který můžeme dále zpracovávat pomocí jednoduché
dopředné sítě pro predikci třídy.

\subsection{Lightning}
\label{sec:Lightning}

PyTorch Lightning je wrapper pro PyTorch, který dále usnadňuje vývoj
neuronových sítí. Stará se za nás o detaily procesu trénování, jako je správa
epoch, logování či optimalizace kroku učení (ang. learning rate). Podporuje
taky nativně práci s TensorBoard, což je logovací nástroj umožňující přehledné
sledování metrik během trénování, včetně grafického zobrazení ve webovém
prostředí.

\begin{lstlisting}[language=Python, label=src:pytorch, caption={Ukázka implementace GRU sítě v PyTorch Lightning}]
class KeypointClassifier(lightning.LightningModule):
    def __init__():
        super(KeypointClassifierGRULightning, self).__init__()
        self.gru = nn.GRU(34, 128)              # GRU vrstvy
        self.classifier = nn.Sequential(        # Plně propojené vrstvy
            nn.Linear(128, 128),
            nn.ReLU(),
            nn.Dropout(0.3),
            nn.Linear(128, 1) )
        self.criterion = nn.BCEWithLogitsLoss ()    # Ztrátová funkce
        self.accuracy = BinaryAccuracy()            # Metrika přesnosti
    def forward(self, x):
        _, x = self.gru(x)
        x = x[-1]                               # Poslední skrytý stav
        x = self.classifier(x)
        return x
    def training_step(self, batch, batch_idx):
        input, target = batch
        output = self(input)
        loss = self.criterion(output, target)
        self.log("train_loss", loss)
        return loss
    def validation_step(self, batch, batch_idx):
        input, target = batch
        output = self(input)
        loss = self.criterion(output, target)
        self.log('val_loss', loss)
        accuracy = self.accuracy(output, target.int())
        self.log('val_accuracy', accuracy)
        return loss

\end{lstlisting}

\section{Implementace vybraných architektur}
\label{sec:SelectedArchitectures}

\subsection{Dopředná neuronová síť}

Nejprve budeme implementovat jednoduchou dopřednou neuronovou síť. Její hlavní
výhodou je rychlost, je to ale za cenu menší přesnosti. Jako vstupní data jsou
použity klíčové body jedné osoby z daného jednoho snímku. Trénovací data tedy
pouze přečteme z trénovacího souboru a překonvertujeme je do formy tenzoru,
konkrétně je zabalíme do instance třídy $DataLoader$. Při použití ve výsledném
programu předáme modelu klíčové body detekované v daném snímku.

% TODO: doplnit testované topologie a hyperparametry

Výstupem sítě je hodnota od $0$ do $1$, čísla od $0$ do $0.5$ jsou považována
za třídu \textit{normální}, zatímco čísla od $0.5$ do $1$ za třídu
\textit{upadl}.

\subsection{GRU síť}

Dále jsme implementovali GRU síť, viz kód \ref{src:pytorch}. V tomto příkladu
je definována GRU jednotka s jednou vrstvou, velikosti vstupního vektoru $34$ a
skrytým stavem velikosti $128$, následována dvouvrstvou dopřednou plně
propojenou sítí ze $128$ neurony v první vrstvě a s jedním neuronem ve druhé
vrstvě.

Cást těchto hyperparametrů jsme parametrizovali, abychom mohli postupně
otestovat různé jejich kombinace a vyhodnotit jejich vliv na výkon modelu, viz
kód \ref{src:params}. Jedná se o velikost vstupního vektoru, velikost skrytého
stavu sítě GRU, počet vrstev GRU, velikost první plně propojené vrstvy,
velikost výstupní vrstvy (pro binární klasifikaci vždy $1$, potřebné pro
pozdější optimalizace) a dropout pro GRU i plně propojené vrstvy. Všechny tyto
hyperparametry budeme ladit v rámci následujících experimentů.

\begin{lstlisting}[language=Python, label=src:params, caption={Parametry konstruktoru třídy $KeypointClassifierGRU$ definující hyperparametry sítě}]
class KeypointClassifierGRULightning(L.LightningModule):
    def __init__(self, input_size=34, rnn_hidden_size=128, rnn_layers_count=2, fc_size=128, output_size=1, rnn_dropout=0.3, fc_dropout=0.3, device=None):
\end{lstlisting}

Stejně jako u předchozí architektury, jako ztrátová funkce byla použita binární
křížová entropie, tedy $nn.BCEWithLogitsLoss()$, která je vhodná pro binární
klasifikaci.

\subsection{LSTM síť}

Podobně jako GRU síť jsme implementovali i LSTM síť. V implementaci se odlišuje
pouze použitím modulu $nn.LSTM$ místo $nn.GRU$. Tento modul se v použití liší
od GRU pouze tím, že kromě skrytého stavu vrací také stav buňky (dlouhodobou
paměť), ten ale stejně nevyužijeme. Pro síť LSTM budeme ladit stejnou množinu
hyperparametrů, jako pro GRU síť, abychom mohli porovnat jejich výkonnost.

\subsection{Vícetřídní klasifikace}

Jak již bylo zmíněno (\ref{sec:TrainingData}), pro naše řešení potřebujeme
detekovat pouze třídy \textit{normální} a \textit{upadl}, vyzkoušíme ale
natrénovat i modely se třemi třídami - přidáme třídu \textit{padá}. Nemá smysl
v našem případě implementovat takovou dopřednou neuronovou síť, jelikož
klasifikuje snímky bez kontextu předchozích snímků.

Budeme tedy navíc implementovat sítě GRU a LSTM, s velice podobnou topologií
jako sítě s binární klasifikací. Hlavním rozdílem bude velikost výstupní
vrstvy, jež bude rovna počtu tříd a použitá ztrátová funkce - nyní použijeme
křížovou entropii. Výstup vícetřídní klasifikační sítě pak představuje pole
pravděpodobností, že se jedná o danou třídu. Při inferenci pak použijeme jednotku 
softmax pro převod výstupu na číslo třídy.

\endinput
