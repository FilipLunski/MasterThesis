\chapter{Kompletace finálního řešení detekce pádu}
\label{chap:detectionProgram}

Tato kapitola bude zaměřena na kompletaci celkového řešení detekce pádu, který
na základě postupně předávané sekvence snímků bude detekovat, zda došlo k pádu.
Jedná se tedy o spojení detekčního algoritmu, který ze snímku extrahuje klíčové
body všech osob, a klasifikačního algoritmu, který na základě těchto bodů
detekuje pád.

Je tedy třeba spojit predikci modelu YOLO11-pose a jednoho z vybraných modelů
(dopředná nebo GRU síť) pro klasifikaci pádu. Nejprve tedy bude krátce popsána
práce ve frameworku YOLO. Následující část se bude věnovat využití dříve
natrénovaných modelů. Důraz bude kladen zejména na implementaci s využitím
rekurentní architektury GRU, která přináší více výzev než dopředná síť, s
ohledem na nutnost uchovávání sekvence pro každého člověka zvlášť. Nakonec bude
program otestován s oběma klasifikačními modely na videích z testovací sady a
bude zhodnocen jejich výkon.

\section{Sledování pózy s YOLO11}

V YOLO verze 11 jsou kromě již zmíněné detekce objektů i klíčových bodů k
dispozici také další volitelné funkce. To, které funkce mají být využity, se
definuje vybraným modelem. V závislosti na něm pak při inferenci model vrací
patřičné hodnoty. Dostupné jsou tyto modely:
\begin{itemize}
    \item \textit{YOLO11<v>-seg } – detekce objektů – bounding boxů
    \item \textit{YOLO11<v>-cls } – detekce objektů, klíčových bodů a segmentace
    \item \textit{YOLO11<v>-pose} – detekce klíčových bodů
    \item \textit{YOLO11<v>-obb } – orientovaná detekce objektů – bounding boxy natočené dle natočení objektů
\end{itemize}

kde $<v>$ označuje velikost modelu – lze vybrat menší modely pro větší výkon
ale horší přesnost, nebo větší modely, které jsou sice velmi přesné, je třeba
ale počítat s vysokými nároky na výkon. V navrženém řešení byl zvolen model
$YOLO11m-pose$. Pro inicializaci modelu je třeba specifikovat cestu k němu,
pokud není nalezen, knihovna automaticky stáhne předtrénovaný model.

\begin{lstlisting}[language=Python, label=src:params, caption={Inicializace modelu $YOLO11m-pose$}]    
    from ultralytics import YOLO    
    pose_model = YOLO("./models_pose/yolo11m-pose.pt")
\end{lstlisting}

Dále lze každý z těchto modelů používat v několika režimech. Režim se
specifikuje tím, jakou metodu modelu použijeme. V případě trénování modelu na
vlastních datech jsou využívány režimy \textit{trénování} (metoda $train()$) a
\textit{validace} (metoda $val()$). Pro práci s již natrénovaným modelem jsou k
dispozici tyto dva režimy: \textit{predikce} (přímé použití modelu, např.
$pose_model()$) pro vyhodnocení každého vstupního snímku zvlášť a
\textit{sledování} (metoda $track()$).

Režim \textit{sledování} vrací stejné informace jako režim \textit{predikce},
navíc ale sleduje pohyb objektů mezi jednotlivými snímky a přiřazuje jim $id$.
V případě použitého modelu tedy kromě bounding boxu a klíčových bodů bude
vracet i $id$ každé osoby.

YOLO11 dokáže zpracovat širokou škálu vstupních dat, včetně videí, kdy je
vráceno pole výsledků, datových proudů (je třeba specifikovat příznak
$stream$), kdy nám je vrácen iterátor, nebo obrázků, kdy je vrácen výsledek pro
daný snímek. Pokud se používá režim \textit{sledování} na jednotlivé snímky, je
třeba specifikovat příznak $persist$, který zajistí, že mezi jednotlivými
snímky bude uchovávána informace o sledovaných objektech. Níže lze vidět
použití metody $track()$ ve výsledném programu (příznaky $show$ a $verbose$
specifikují, zda se má zobrazit výsledek každého snímku a zda má model do
konzoly vypisovat ladící informace)

\begin{lstlisting}[language=Python, label=src:params, caption={Použití sledování pomocí YOLO11}]
    results = self._pose_model.track(
        frame, show=False, verbose=False, persist=True)
\end{lstlisting}

\section{Struktura třídy FallDetector}

Třída $FallDetector$ slouží pro detekci pádu v sekvenci postupně předávaných
snímků. Je ji tak možné použít jak pro analýzu uloženého videa, tak pro detekci
pádu v živém přenosu v reálném čase. Mezi jednotlivými snímky uchovává pro
každou osobu klíčové body z $x$ posledních snímků pro predikci na základě
sekvence snímků. Dále dle nastavení uchovává $n$ posledních snímků, popřípadě i
anotovaných, pro uložení videa včetně několika sekund před a po pádu. Pro
ukládání těchto informací je použita kolekce typu $deque$, která umožňuje
specifikovat její maximální délku, nerelevantní informace ze starších snímků
tedy budou automaticky odstraňovány.

V konstruktoru obdrží tato třída cestu k modelu YOLO (musí být inicializován
pro každou instanci zvlášť), model pro detekci pádu na základě klíčových bodů,
délku sekvence předávané tomuto modelu, a informace týkající se ukládání videí
s pády, jako jsou cesty k souborům a délka videa před a po pádu.

Pro práci s jednotlivými osobami je použita třída $Person$, která uchovává
hlavně klíčové body z posledních snímků a $id$, dále také bounding boxy,
detekovaný stav a confidence score této detekce pro poslední snímek, kdy byla
osoba detekována. V případě, že daná osoba již není detekována, obsahuje
informaci, kolik snímků již není vidět pro vymazání osob, jež odešly ze scény.
Okamžité vymazání zmizelé osoby není žádoucí, jelikož někdy osoba je pořád na
scéně, ale algoritmus ji i na několik snímků ztratí, zejména v případě okluze
či špatných světelných podmínek.

V metodě $process\_frame$ třídy $FallDetector$ je zpracován vždy jeden snímek.
Nejprve se detekují všechny osoby a jejich klíčové body pomocí vybraného
detekčního modelu. Pokud je požadováno uložení anotovaného videa, převezme se
zde anotovaný snímek přímo od modelu YOLO, později je pouze ke každé osobě
dopsána její třída detekce pádu. Dále jsou na základě detekovaných informací
vytvořeny, popřípadě aktualizovány instance třídy $Person$. Pro každou osobu je
následně provedena analýza sekvence klíčových bodů pro detekci pádu. Pokud je
detekován pád a je nastaveno ukládání videa, vytvoří se nový soubor a zapíše se
do něj uchované předešlé snímky. Do souboru se pak zapisuje, dokud je detekován
pád a nevyprší požadovaný počet snímků po pádu.

\section{Testování detekce pádu ve videu}
\label{sec:FallDetectionTest}

V této sekci bude otestováno výsledné řešení na videích z testovacích sad obou
datasetů. Hlavně bude ověřena přesnost detekce pádu v různých situacích ve
videích. Přesnost klasifikačního algoritmu, která byla měřena v minulé
kapitole, je spíše na úrovni detekce v jednotlivých snímcích. Pokud tedy
program detekuje pád o několik snímků později, než byl pád anotován, je tato
přesnost penalizována. Zde je klíčovým ukazatelem přesnost ve smyslu schopnosti
detekovat každý výskyt pádu a současně minimalizovat výskyt falešných detekcí.

V testovací sadě datasetu CAUCAFall je 15 videí, z čehož 7 obsahuje pád a 8
obsahuje jiné činnosti, které by jako pád neměly být vyhodnoceny, kdežto ve
všech 15 případech testovací sady z videa \textit{50 ways to fall} vystupuje
pád.

Bylo zjištěno, že obecně jsou výsledky sítě GRU mnohém stabilnější než v
případě dopředné sítě. Pokud tedy osoba spadne a leží na zemi, GRU síť pořád
klasifikuje pád, zatímco dopředná síť často, zvlášť pokud osoba leží vyrovnaná,
kolísá mezi třídou \textit{normální} a \textit{upadl}. Stabilita dopředné sítě
se také zhoršuje v případě špatné viditelnosti osoby, kdy jsou výsledky modelu
pro detekci pózy více chaotické než v optimálních podmínkách.

Dopředná síť reaguje na pád mnohem rychleji, většinou už v době padání. Oproti
tomu GRU síť detekuje pád až v momentě dopadu nebo o několik snímků později.
Dopředná síť ale dokáže rychleji reagovat i na situaci kdy se osoba ze stavu
\textit{upadl} dostane zpátky do stavu \textit{normální}. V jednom z
testovacích videí osoba dělá kotrmelec, po kterém se vrací do dřepu. Dopředná
síť tehdy okamžitě vyhodnotila její stav zpátky jako normální, kdežto GRU síť
ještě nějakou dobu identifikovala pád.

Hlavní problém s dopřednou síti je ale v její falešné pozitivitě. GRU síť
všechny testovací videa vyhodnotila správně, a tedy detekovala pád pouze a
právě ve videích, kde osoba upadla. Dopředná síť detekovala sice pád ve všech
videích, kde osoba upadla. Z osmi videí, na kterých pád zaznamenán není ale
tato síť ve dvou případech pád detekovala. Konkrétně se tak stalo v situacích,
kdy osoba klečela a kdy se ohnula k zemi. Jelikož v těchto případech detekovala
pád pouze v několika snímcích, dala by se tato falešná pozitivita odfiltrovat
tím, že by se na pád reagovalo až po několika snímcích, kdy byl soustavně
detekován.

Co se týče výsledné celkové rychlosti detektoru, lze v tabulce
\ref{tab:detectorSpeed} vidět, že se ve všech případech podařilo dosáhnout
stanovené minimální frekvence $30$ FPS, viz sekce
\ref{sec:performance_requirements}.

\begin{table}[htbp]
    \centering
    \caption{Rychlost detekce pádu}
    \label{tab:detectorSpeed}
    \begin{tabular}{|c|cc|cc|}
        \hline
                     & \multicolumn{2}{l|}{50 ways to fall} & \multicolumn{2}{l|}{CAUCAFall}                                   \\ \hline
                     & \multicolumn{1}{l|}{ms}              & FPS                            & \multicolumn{1}{l|}{ms}   & FPS \\ \hline
        Dopředná síť & \multicolumn{1}{l|}{25.6}            & 39                             & \multicolumn{1}{l|}{24.2} & 41  \\ \hline
        GRU síť      & \multicolumn{1}{l|}{27}              & 37                             & \multicolumn{1}{l|}{25.9} & 39  \\ \hline
    \end{tabular}
\end{table}

Rozdíl rychlosti mezi datasety je způsoben hlavně velikostí snímku. V datasetu
\textit{50 ways to fall}, je totiž snímek téměř třikrát větší než v případě CAUCAFall.
Pro snímky s větším rozlišením musí YOLO nejprve věnovat více času na
zredukování velikosti snímku. Předpokládá se ale, že většinou je rozlišení u
průmyslových kamer spíše menší než HD, v praxi by se tedy mělo dosáhnout spíše
výsledků jako v případě datasetu CAUCAFall. Mezi oběma modely je také malý rozdíl v
rychlosti, který je způsoben hlavně režií sekvencí v případě sítě GRU.

Výše uvedená měření byla provedena bez anotace videí. Pokud byla videa anotována pouze informací o pádu, doba zpracování se většinou zvýšila
pouze o zhruba 2 – 3 milisekundy. Pokud ale byla videa anotována klíčovými body
a bounding boxy pomocí nástroje integrovaného v modelu YOLO, došlo ke zpomalení až o 15
milisekund.

Obě sítě tedy fungují poměrně dobře a dokážou v reálném čase detekovat pád v
různých situacích, z výsledků je ale zjevné, že je rekurentní síť GRU mnohem
stabilnější a přesnější.

% \begin{table}[htbp]
%     \centering
%     \begin{tabular}{|c|c|c|}
%         \hline
%                      & ms   & FPS \\ \hline
%         Dopředná síť & 25.9 & 39  \\ \hline
%         GRU síť      & 33   & 30  \\ \hline
%     \end{tabular}
% \end{table}