\chapter{Implementace detekčního algoritmu}
\label{chap:detectionAlgorithm}

V této kapitole se zaměříme na implementaci samotného detekčního algoritmu,
který na základě postupně předáváné sekvence snímku bude detekovat, zda došlo k
pádu.

\section{Sledování s YOLO}

V YOLO verze 11 máme kromě již zmíněné detekce objektů i klíčových bodů k
dispozici také další funkce. To, které funkce chceme využít, definujeme
vybraným modelem. V závislosti na něm pak pak při interferenci model vrací
patřičné hodnoty. Dostupné jsou tyto modely:
\begin{itemize}
    \item \textit{YOLO11<v>-seg }- detekce objektů - bounding boxů
    \item \textit{YOLO11<v>-cls }- detekce objektů, klíčových bodů a segmentace
    \item \textit{YOLO11<v>-pose} - detekce klíčových bodů
    \item \textit{YOLO11<v>-obb }- orientovaná detekce objektů - bounding boxy natočené dle natočení objektu
\end{itemize}

kde $<v>$ označuje velikost modelu - můžeme vybrat menší modely pro větší
výkon ale horší přesnost, anebo větší modely, které jsou sice velmi přesné,
musíme ale počítat s vysokými nároky na výkon. V našem případě jsme zvolili model $yolo11m-pose$

\endinput