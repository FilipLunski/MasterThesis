\chapter{Implementace kompletního řešení detekce pádu}
\label{chap:detectionProgram}

V této kapitole se zaměříme na implementaci celkového řešení detekce pádu,
který na základě postupně předávané sekvence snímků bude detekovat, zda došlo k
pádu. Jedná se tedy o spojení detekčního algoritmu, který ze snímku extrahuje
klíčové body všech osob, a klasifikačního algoritmu, který na základě těchto
bodů detekuje pád.

V našem případě tedy musíme spojit predikci modelu YOLO11-pose, a jednoho z
naších vybraných modelů (dopředná nebo GRU síť) pro klasifikaci pádu. Podíváme
se tedy nejprve krátce na práci ve frameworku YOLO. Následně se budeme zabývat
použitím dříve natrénovaných modelů. Zejména se zaměříme na implementaci s
využitím rekurentní architektury GRU, která přináší více výzev než dopředná
siť, s ohledem na nutnost uchovávání sekvence pro každého člověka zvlášť.
Nakonec program otestujeme s oběma klasifikačními modely na několika videích a
zhodnotíme jejich výkon.

\section{Sledování pózy s YOLO11}

V YOLO verze 11 máme kromě již zmíněné detekce objektů i klíčových bodů k
dispozici také další volitelné funkce. To, které funkce chceme využít,
definujeme vybraným modelem. V závislosti na něm pak při inferenci model vrací
patřičné hodnoty. Dostupné jsou tyto modely:
\begin{itemize}
    \item \textit{YOLO11<v>-seg } – detekce objektů – bounding boxů
    \item \textit{YOLO11<v>-cls } – detekce objektů, klíčových bodů a segmentace
    \item \textit{YOLO11<v>-pose} – detekce klíčových bodů
    \item \textit{YOLO11<v>-obb } – orientovaná detekce objektů – bounding boxy natočené dle natočení objektů
\end{itemize}

kde $<v>$ označuje velikost modelu – můžeme vybrat menší modely pro větší výkon
ale horší přesnost, anebo větší modely, které jsou sice velmi přesné, musíme
ale počítat s vysokými nároky na výkon. V našem případě jsme zvolili model
$YOLO11m-pose$. Pro inicializaci modelu musíme specifikovat cestu k němu, pokud
není nalezen, knihovna automaticky stáhne předtrénovaný model.

\begin{lstlisting}[language=Python, label=src:params, caption={Inicializace modelu $YOLO11m-pose$}]    
    from ultralytics import YOLO    
    pose_model = YOLO("./models_pose/yolo11m-pose.pt")
\end{lstlisting}

Dále můžeme každý z těchto modelů používat v několika režimech. Režim
specifikujeme tím, jakou metodu modelu použijeme. Pokud chceme model natrénovat
na vlastních datech, použijeme režimy \textit{trénování} (metoda $train()$) a
\textit{validace} (metoda $val()$). Pokud chceme používat již natrénovaný
model, máme k dispozici dva režimy: \textit{predikce} (přímé použití modelu,
např. $pose_model()$) pro vyhodnocení každého vstupního snímku zvlášť a
\textit{sledování} (metoda $track()$).

Režim \textit{sledování} vrací stejné informace jako režim \textit{predikce},
navíc ale sleduje pohyb objektů mezi jednotlivými snímky a přiřazuje jim $id$.
V našem případě tedy kromě bounding boxu a klíčových bodů dostaneme i $id$
každé osoby.

YOLO11 dokáže zpracovat širokou škálu vstupních dat, včetně videí, kdy nám je
vráceno pole výsledků, datových proudů (je třeba specifikovat příznak
$stream$), kdy nám je vrácen iterátor, nebo obrázků, kdy dostaneme výsledek pro
daný snímek. Pokud chceme použít režim \textit{sledování} na jednotlivé snímky,
musíme specifikovat příznak $persist$, který zajistí, že mezi jednotlivými
snímky bude uchovávána informace o sledovaných objektech. Zde můžeme vidět
použití metody $track()$ v našem programu (příznaky $show$ a $verbose$
specifikují, zda se má zobrazit výsledek každého snímku a zda má model do
konzoly vypisovat ladící informace)

\begin{lstlisting}[language=Python, label=src:params, caption={Použití sledování pomocí YOLO11}]
    results = self._pose_model.track(
        frame, show=False, verbose=False, persist=True)
\end{lstlisting}

\section{Struktura třídy FallDetector}

Třída $FallDetector$ slouží pro detekci pádu v sekvenci postupně předávaných
snímků. Můžeme ji tak použít jak pro analýzu uloženého videa, tak pro detekci
pádu v živém přenosu v reálném čase. Mezi jednotlivými snímky uchovává pro
každou osobu klíčové body z $x$ posledních snímků pro predikci na základě
sekvence snímků. Dále dle nastavení uchovává $n$ posledních snímků, popřípadě i
anotovaných, pro uložení videa včetně několika sekund před a po pádu. Pro
ukládání těchto informací používáme kolekci typu $deque$, která umožňuje
specifikovat její maximální délku, nerelevantní informace ze starších snímků
tedy budou automaticky odstraňovány.

V konstruktoru obdrží tato třída cestu k modelu YOLO (musí být inicializován
pro každou instanci zvlášť), model pro detekci pádu na základě klíčových bodů,
délku sekvence předávané tomuto modelu, a informace týkající se ukládání videí
s pády, jako jsou cesty k souborům a délka videa před a po pádu.

Pro práci s jednotlivými osobami používáme třídu $Person$, která uchovává
hlavně klíčové body z posledních snímků a $id$, dále také bounding boxy,
detekovaný stav a confidence score této detekce pro poslední snímek, kdy byla
osoba detekována. V případě, že daná osoba již není detekována, obsahuje
informaci, kolik snímků již není vidět pro vymazání osob, jež odešly ze scény.
Okamžité vymazání zmizelé osoby není žádoucí, jelikož někdy osoba je pořád na
scéně, ale algoritmus ji i na několik snímků ztratí, zejména v případě okluze
či špatných světelných podmínek.

V metodě $process\_frame$ třídy $FallDetector$ je zpracován vždy jeden snímek.
Nejprve se detekují všechny osoby a jejich klíčové body pomocí vybraného
modelu. Pokud je požadováno uložení anotovaného videa, převezmeme zde anotovaný
snímek přímo od modelu YOLO, později pouze dopíšeme třídu detekce pádu. Dále
jsou na základě detekovaných informací vytvořeny, popřípadě aktualizovány
instance třídy $Person$. Pro každou osobu je také provedena analýza sekvence
klíčových bodů pro detekci pádu a popřípadě je patřičně anotována ve snímku.
Pokud je detekován pád a je nastaveno ukládání videa, vytvoří se nový soubor a
zapíše se do něj uchované předešlé snímky. Do souboru se pak zapisuje, dokud je
detekován pád a nevyprší požadovaný počet snímků po pádu.

\section{Testování detekce pádu ve videu}
\label{sec:FallDetectionTest}

V této sekci otestujeme výsledné řešení na videích z testovacích sad obou
datasetů. Hlavně ověříme přesnost detekce pádu v různých situacích ve videích.
Přesnost klasifikačního algoritmu, kterou jsme měřili v minulé kapitole, je
spíše na úrovní detekce v jednotlivých snímcích. Pokud tedy program detekuje
pád o několik snímků později, než byl pád anotován, je tato přesnost
penalizována. Nás ale spíše zajímá přesnost ve smyslu, zda byl každý pád
detekován, a zda se neobjevovaly falešné detekce pádů.

V testovací sadě datasetu CAUCAFall je 15 videí, z čehož 7 obsahuje pád a 8
obsahuje jiné činnosti, které by jako pád neměly být vyhodnoceny, kdežto ve
všech 15 případech testovací sady z videa \textit{50 ways to fall} vystupuje
pád.

Zjistili jsme, že obecně jsou výsledky sítě GRU mnohém stabilnější než v
případě dopředné sítě. Pokud tedy osoba spadne a leží na zemi, GRU síť pořád
klasifikuje pád, zatímco dopředná síť často, zvlášť pokud osoba leží vyrovnaná,
kolísá mezi třídou \textit{normální} a \textit{upadl}. Stabilita dopředné sítě
se také zhoršuje v případě špatné viditelnosti osoby, kdy jsou výsledky modelu
pro detekci pózy více chaotické než v optimálních podmínkách.

Dopředná síť také reaguje na pád mnohem rychleji, většinou už v době padání.
Oproti tomu GRU síť detekuje pád až v momentě dopadu nebo o několik snímků
později. Dopředná síť ale dokáže rychleji reagovat i na situaci kdy se osoba ze
stavu \textit{upadl} dostane zpátky do stavu \textit{normální}. V jednom z
testovacích videí osoba dělá kotrmelec, po kterém se vrátila do dřepu. Dopředná
síť tehdy okamžitě vyhodnotila její stav zpátky jako normální, kdežto GRU síť
ještě nějakou dobu identifikovala pád.

Hlavní problém s dopřednou síti je ale v její falešné pozitivitě. GRU síť
všechny testovací videa vyhodnotila správně, a tedy detekovala pád pouze a
právě ve videích, kde osoba upadla. Dopředná síť detekovala sice pád ve všech
videích, kde osoba upadla. Z osmi videí, na kterých pád zaznamenán není, ale
tato síť ve dvou případech pád detekovala. Konkrétně se tak stalo v situacích,
kdy osoba klečela a kdy se ohnula k zemi. Jelikož v těchto případech detekovala
pád pouze v několika snímcích, dala by se tato falešná pozitivita odfiltrovat
tím, že bychom reagovali na pád až po několika snímcích, kdy byl soustavně
detekován.

Pojďme se nyní ještě podívat na výslednou rychlost detektoru. V tabulce
\ref{tab:detectorSpeed} vidíme, že se nám ve všech případech podařilo dosáhnout
stanovené minimální frekvence, viz sekce \ref{sec:performance_requirements}.

\begin{table}[htbp]
    \caption{Rychlost detekce pádu}
    \label{tab:detectorSpeed}
    \begin{tabular}{|c|cc|cc|}
        \hline
                     & \multicolumn{2}{l|}{50 ways to fall} & \multicolumn{2}{l|}{CAUCAFall}                                   \\ \hline
                     & \multicolumn{1}{l|}{ms}              & FPS                            & \multicolumn{1}{l|}{ms}   & FPS \\ \hline
        Dopředná síť & \multicolumn{1}{l|}{23.7}            & 42                             & \multicolumn{1}{l|}{24.5} & 41  \\ \hline
        GRU síť      & \multicolumn{1}{l|}{31.9}            & 31                             & \multicolumn{1}{l|}{25.1} & 40  \\ \hline
    \end{tabular}
\end{table}
\endinput

Rozdíl rychlosti mezi datasety je způsoben hlavně velikostí snímku. Pro snímky
s větším rozlišením musí YOLO nejprve věnovat více času na zredukování
velikosti snímku. Mezi oběma modely je taky rozdíl v rychlosti, je způsoben
hlavně režií sekvencí v případě sítě GRU. Opět je rozdíl výraznější v případě
datasetu 50 ways to fall, kde je snímek téměř třikrát větší než v případě
CAUCAFall. 
% Předpokládáme ale, že většinou je rozlišení u průmyslových kamer
% spíše menší, rychlost

Výše uvedená měření byla provedena provedena bez anotace videí. Pokud jsme
anotovali videa pouze informací o pádu, doba zpracování se většinou zvýšila
pouze o zhruba 2 - 3 milisekundy. Pokud jsme ale použili anotaci klíčových bodů
a bounding boxů integrovanou v modelu YOLO, došlo k zpomalení až 15 milisekund. 



Obě sítě tedy fungují poměrně dobře a dokážou v reálném čase detekovat pád v různých
situacích, z výsledků je ale zjevné, že je rekurentní síť GRU mnohem
stabilnější a přesnější.