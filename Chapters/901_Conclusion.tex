\chapter{Závěr}
\label{chap:Conclusion}

Cílem práce bylo navrhnout a implementovat řešení pro detekci pádu osoby v toku
obrázku v reálném čase. V první části byly popsány teoretické základy použitých
technologií, se zaměřením na neuronové sítě.

V praktické části byl popsán postup návrhu a vývoje výsledného programu. Řešení
jsme dosáhli spojením volně dostupného modelu pro detekci pózy ve formě
klíčových bodů a neuronové sítě pro klasifikaci těchto bodů do třídy
\textit{normální} nebo \textit{upadl}.

Prozkoumali jsme několik detekčních algoritmů – vysvětlili jsme si zásady
funkčnosti detekce klíčových bodů a otestovali jsme jednotlivé algoritmy na
testovacích datech. Pro detekci pózy jsme nakonec zvolili model YOLO11-pose pro
jeho vysokou rychlost, zejména při využití grafické akcelerace, a přesnost i
při horší viditelnosti. Tento model také v sobě integruje funkci sledování
osob, která nám umožňuje provádět následnou analýzu pro každou osobu zvlášť i v
kontextu více snímků.

Pro klasifikaci pózy jsme zkoušeli natrénovat modely postavené na třech  
architekturách: dopředná neuronová síť a dvě rekurentní architektury – LSTM a
GRU. LSTM síť se nám nepodařilo efektivně natrénovat, pravděpodobně pro
nedostatek dat a její nevhodnost pro náš problém.

Dopřednou síť a GRU síť jsme po nalezení optimální konfigurace a natrénování
otestovali na testovacích datech. Následně jsme je použili v implementaci
detektoru pádu. Ten jsme dále v obou verzích otestovali na videích z testovací
sady. Dopředná síť se ukázala jako znatelně méně přesná, zejména při méně
kvalitní detekci pózy v situacích s horší viditelností. Také projevovala
falešné pozitivní detekce pádu.

Pro klasifikaci pózy jsme tedy zvolili rekurentní architekturu GRU s jednou
rekurentní a jednou plně propojenou vrstvou, obě velikosti $64$. Tuto síť se
nám podařilo natrénovat na přesnost přes 96\% na testovacích datech, zároveň se
ukázala jako velice stabilní a přesná i při detekci pádu v testovacích videích.

Pokud bychom chtěli pokračovat v optimalizaci našeho řešení, nejvíce
prostoru vidíme v množství trénovacích dat. Zejména by bylo vhodné použít videa s
více úhly natočení vůči zemi a s různými vzdálenostmi osoby od kamery. Větší a
rozmanitější dataset by nám umožnil natrénovat i komplexnější architektury,
které by byly přesnější a robustnější. Pro získání více dat by se mohlo nasadit
náš detektor na podnikové kamery, nebo by se pomocí něj dalo zanalyzovat starší
záznamy. V těchto případech by bylo vhodné nastavit nižší prah detekce, abychom
získali i data s vyšší pravděpodobností falešné pozitivity.

Dalším vylepšením by mohlo být zavedení více tříd do analýzy pózy. V textu byla
zmíněna třída \textit{padá}, nicméně bychom mohli také přidat třídy např. pro
pozici v sedě, ve dřepu, či na kolenou. Model by tak mohl lépe rozeznávat
některé situace, které se zdají být pádu podobné, a tak bychom odstranili
potenciální falešné pozitivní detekce pádu.

Zajímavým rozšířením by také mohlo být propojení více kamer z jedné místnosti
pro sledování osob z více úhlů. Dávalo by to možnost zasazení postav do
trojrozměrného prostoru a umožňovalo dosáhnout mnohem kvalitnější detekce
pózy. Nicméně by bylo obtížnější získat pro takové řešení trénovací data.

Pro nasazení detektoru pádu v reálném provoze, tedy na obrazovém toku z
podnikových bezpečnostních kamer, bude ještě třeba vytvořit vhodné rozhraní pro
správné čtení různých protokolů a formátů videa. Pro integraci do komplexnějšího
programu, ve kterém by probíhalo více analýz videa, by bylo vhodné zvážit
jednak jejich paralelní zpracování, jednak sdílení určitých dat mezi
jednotlivými analyzátory.

\endinput