\chapter{Závěr}
\label{chap:Conclusion}

% TODO
\chapter{Závěr}
\label{chap:Conclusion}

Cílem práce bylo navrhnut a implementovat řešení pro detekci pádu osoby v toku
obrázku v reálném čase. V první části byly popsány teoretické základy použitých
technologií, se zaměřením na neuronové sítě.

V praktické části byl popsán postup návrhu a vývoje výsledného programu. Řešení
jsme dosáhli spojením volně dostupného modelu pro detekci pózy ve formě
klíčových bodů a neuronové sítě pro klasifikaci těchto bodů do třídy
\textit{normální} nebo \textit{upadl}.

Prozkoumali jsme několik detekčních algoritmu - vysvětlili jsme si zásadu funkčnosti detekce klíčových bodů a otestovali jsme jednotlivé algoritmy na testovacích datech.
Pro detekci pózy jsme tedy nakonec zvolili model YOLO11-pose pro jeho vysokou rychlost,
zejména s využitím grafické akcelerace, a přesnost i při horší viditelnosti.
Tento model taky v sobě integruje funkci sledování osob, která nám umožňuje provádět analýzu pro
každou osobu zvlášť.

Pro klasifikaci pózy byla zvolená rekurentní architektura GRU s jednou
rekurentní a jednou plně propojenou vrstvou, obě velikosti $64$. Tuto síť se
nám podařilo natrénovat na přesnost přes 96\% na testovacích datech, zároveň se
ukázala jako velice stabilní a přesná i při detekci pádu v testovacích videích.

\endinput
\endinput