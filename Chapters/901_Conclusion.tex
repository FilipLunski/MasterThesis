\chapter{Závěr}
\label{chap:Conclusion}

Cílem práce bylo navrhnout a implementovat řešení pro detekci pádu osoby v toku
obrázků v reálném čase. V první části byly popsány teoretické základy použitých
technologií, se zaměřením na neuronové sítě.

V praktické části byl popsán postup návrhu a vývoje výsledného programu. Řešení
bylo dosaženo spojením volně dostupného modelu pro detekci pózy ve formě
klíčových bodů a neuronové sítě pro klasifikaci těchto bodů do třídy
\textit{normální} nebo \textit{upadl}.

Bylo prozkoumáno několik detekčních algoritmů – byly vysvětleny zásady
funkčnosti detekce klíčových bodů a následně byly jednotlivé algoritmy
otestovány na testovacích datech. Pro detekci pózy byl nakonec zvolen model
YOLO11-pose pro jeho vysokou rychlost, zejména při využití grafické akcelerace,
a přesnost i při horší viditelnosti. Tento model také v sobě integruje funkci
sledování osob, která umožňuje provádět následnou analýzu pro každou osobu
zvlášť i v kontextu více snímků.

Pro klasifikaci pózy byly trénovány modely postavené na třech různých
architekturách: dopředná neuronová síť a dvě rekurentní architektury – LSTM a
GRU. LSTM síť se nepodařilo efektivně natrénovat, pravděpodobně pro nedostatek
dat a její nevhodnost pro řešený problém.

Dopředná síť a GRU síť byly po nalezení optimální konfigurace a natrénování
otestovány na testovacích datech. Následně byly použity v implementaci
detektoru pádu. Ten byl dále v obou verzích otestován na videích z testovací
sady. Dopředná síť se ukázala jako znatelně méně přesná, zejména při méně
kvalitní detekci pózy v situacích s horší viditelností. Také projevovala
falešné pozitivní detekce pádu.

Pro klasifikaci pózy tedy byla zvolena rekurentní architektura GRU s jednou
rekurentní a jednou plně propojenou vrstvou, obě velikosti $64$. Tuto síť se
podařilo natrénovat na přesnost přes 96\% na testovacích datech, zároveň se
ukázala jako velice stabilní a přesná i při detekci pádu v testovacích videích.

Pokud by se pokračovalo v optimalizaci navrženého řešení, nejvíce prostoru pro zlepšení je v množství trénovacích dat. Zejména by bylo vhodné použít videa s více
úhly natočení vůči zemi a s různými vzdálenostmi osoby od kamery. Větší a
rozmanitější dataset by umožnil natrénovat i komplexnější architektury, které
by byly přesnější a robustnější. Pro získání více dat by se mohlo nasadit
navržený detektor na podnikové kamery, nebo by se pomocí něj dalo zanalyzovat
starší záznamy. V těchto případech by bylo vhodné nastavit nižší prah detekce
pro získání dat také s vyšší pravděpodobností falešné pozitivity.

Dalším vylepšením by mohlo být zavedení více tříd do analýzy pózy. V textu byla
zmíněna třída \textit{padá}, nicméně bylo by možné přidat také třídy např. pro
pozici v sedě, ve dřepu, či na kolenou. Model by tak mohl lépe rozeznávat
některé situace, které se zdají být pádu podobné, a  odstranily by se tak
potenciální falešné pozitivní detekce pádu.

Zajímavým rozšířením by také mohlo být propojení více kamer z jedné místnosti
pro sledování osob z více úhlů. Dávalo by to možnost zasazení postav do
trojrozměrného prostoru a umožňovalo dosáhnout mnohem kvalitnější detekce pózy.
Nicméně bylo by obtížnější získat pro takové řešení trénovací data.

Pro nasazení detektoru pádu v reálném provoze, tedy na obrazovém toku z
podnikových bezpečnostních kamer, bude ještě třeba vytvořit vhodné rozhraní pro
správné čtení různých protokolů a formátů videa. Pro integraci do
komplexnějšího programu, ve kterém by probíhalo více analýz videa, by bylo
vhodné zvážit jednak jejich paralelní zpracování, jednak sdílení určitých dat
mezi jednotlivými analyzátory.

\endinput