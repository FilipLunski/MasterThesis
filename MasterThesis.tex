% Nejprve uvedeme tridu dokumentu s volbami
\documentclass[czech,master]{diploma}
% Dalsi doplnujici baliky maker
\usepackage[autostyle=true,czech=quotes]{csquotes} % korektni sazba uvozovek, podpora pro balik biblatex
\usepackage[backend=biber, style=iso-numeric, alldates=iso]{biblatex} % bibliografie
\usepackage{dcolumn} % sloupce tabulky s ciselnymi hodnotami
\usepackage{subfig} % makra pro "podobrazky" a "podtabulky"
\usepackage[cpp]{diplomalst}

% Zadame pozadovane vstupy pro generovani titulnich stran.
\ThesisAuthor{Filip Łuński}

\ThesisSupervisor{Ing. Tomáš Wiszczor, Ph.D.}

\CzechThesisTitle{Využití kamerového systému pro zajištěni bezpečnosti osob na pracovišti}

\EnglishThesisTitle{Use of Surveillance Cameras to Ensure the Safety of People in the Workplace}

\SubmissionYear{2025}

\ThesisAssignmentFileName{ThesisSpecification_LUN0024_vsboee2404016E.pdf}

% Pokud nechceme nikomu dekovat makro zapoznamkujeme.
\Acknowledgement{Rád bych na tomto místě poděkoval všem, kteří mi s prací pomohli, protože bez nich by tato práce nevznikla.}

\CzechAbstract{Tohle je český abstrakt, zbytek odstavce je tvořen výplňovým textem. Naší si rozmachu potřebami s posílat v poskytnout ty má plot. Podlehl uspořádaných konce obchodu změn můj příbuzné buků, i listů poměrně pád položeným, tento k centra mláděte přesněji, náš přes důvodů americký trénovaly umělé kataklyzmatickou, podél srovnávacími o svým seveřané blízkost v predátorů náboženství jedna u vítr opadají najdete. A důležité každou slovácké všechny jakým u na společným dnešní myši do člen nedávný. Zjistí hází vymíráním výborná.}

\CzechKeywords{typografie; \LaTeX; diplomová práce}

\EnglishAbstract{This is English abstract. Lorem ipsum dolor sit amet, consectetuer adipiscing elit. Fusce tellus odio, dapibus id fermentum quis, suscipit id erat. Aenean placerat. Vivamus ac leo pretium faucibus. Duis risus. Fusce consectetuer risus a nunc. Duis ante orci, molestie vitae vehicula venenatis, tincidunt ac pede. Aliquam erat volutpat. Donec vitae arcu. Nullam lectus justo, vulputate eget mollis sed, tempor sed magna. Curabitur ligula sapien, pulvinar a vestibulum quis, facilisis vel sapien. Vestibulum fermentum tortor id mi. Etiam bibendum elit eget erat. Pellentesque pretium lectus id turpis. Nulla quis diam.}

\EnglishKeywords{typography; \LaTeX; master thesis}

\AddAcronym{DVD}{Digital Versatile Disc}
\AddAcronym{TNT}{Trinitrotoluen}
\AddAcronym{UML}{Unified Modeling Language}
\AddAcronym{HTML}{Hyper Text Markup Language}
\AddAcronym{TUG}{\TeX{} Users Group}

\addbibresource{biblatex.bib}

% Novy druh tabulkoveho sloupce, ve kterem jsou cisla zarovnana podle desetinne carky
\newcolumntype{d}[1]{D{,}{,}{#1}}


% Zacatek dokumentu
\begin{document}

% Nechame vysazet titulni strany.
\MakeTitlePages

% Jsou v praci obrazky? Pokud ano vysazime jejich seznam a odstrankujeme.
% Pokud ne smazeme nasledujici dve makra.
\listoffigures
\clearpage

% Jsou v praci tabulky? Pokud ano vysazime jejich seznam a odstrankujeme.
% Pokud ne smazeme nasledujici dve makra.
\listoftables
\clearpage

% A nasleduje text zaverecne prace.
\chapter{Úvod}
\label{sec:Introduction}

Kamerové systémy jsou využívány již mnoho let a jejich využití je stále širší.
Dnes se odhaduje, že celkový počet bezpečnostních kamer ve světě přesahuje
miliardu. Využívány jsou v průmyslu, dopravě, obchodě, veřejných prostorech,
zdravotnictví či domácnostech.

Zpočátku bylo možné video sledovat pouze živě, později, s příchodem videokazet,
bylo možné záznam sledovat až po události. Digitální era a síťové kamery
umožnily přístup ke kamerovým záznamům z libovolného místa na světě. V poslední
době se také začalo nahrazovat živé sledování automatickým zpracováním obrazu a
detekcí událostí s využitím technik umělé inteligence.

Kamerové systémy se používají zejména ve dvou oblastech: zabezpečení (ang.
security), myšleno jako ochrana před úmyslnými hrozbami a protiprávními činy,
jako jsou krádeže, poškozování majetku, či neoprávněný vstup; a bezpečnost
(ang. safety), což zahrnuje ochranu před nehodami a náhodnými hrozbami, jako
jsou pády, požár, úniky nebezpečných látek, či porušování bezpečnostních
předpisů.

Jak již bylo zmíněno, lze kamery využívat jednak pro živé sledování, jednak pro
záznam a jeho analýzu po události. Kamerové záznamy jsou zejména důležité pro
zpětnou analýzu incidentů, důkazní materiál pro soudní spory, zjišťování příčin
nehod, či pro zlepšení bezpečnostních opatření. Živé sledování videa se pak
snaží incidentům přímo předcházet. Bylo však prokázáno, že schopnost lidského
pozorovatele detekovat nebezpečí se velmi snižuje s délkou sledování a s počtem
monitorovaných kamer. Právě proto se s příchodem technik umělé inteligence
začalo využívat automatické zpracování obrazu a detekce hrozeb, nebezpečí, nebo
již probíhajících incidentů v jejích počátcích. Tyto techniky pak úplně
nahrazují lidského pozorovatele, nebo mu pomáhají včas zpozorovat nebezpečí a
zareagovat.

Automatická analýza obrazu je používaná již několik desítek let, většinou ale
spíše pro oblast zabezpečení, než pro bezpečnost. To z toho důvodu, že úlohy,
jako identifikace neoprávněného vstupu, detekce zbraní, rozpoznávání SPZ nebo
podezřelých osob jsou pro algoritmy mnohem jednodušší, než například detekce
pádu, nouzové situace či zdravotního problému. Hlavním problémem těchto
komplexnějších analýz je vysoká falešná pozitivita, kdy je například těžké
rozeznat člověka trénujícího běh od člověka utíkajícího před nebezpečím.
Nicméně rozvoj v oblasti hlubokého učení a konvolučních neuronových sítí, jako
i vývoj a dostupnost hardwaru podporujícího tyto techniky, umožňuje dneska využít
je i pro složitější úlohy.

Ve firmě Linde jsou kamerové systémy používány v mnoha průmyslových provozech,
nicméně chybí ucelený systém pro automatickou analýzu obrazu a detekci různých
druhů nebezpečí. Naším úkolem tedy v budoucnu bude navrhnout a implementovat
modulární systém s možnosti sledování konkrétních nebezpečí na konkrétních
místech. Ty budou zahrnovat například detekci pádu, požáru, zdravotních problémů,
nebo porušování bezpečnostních opatření. Systém pak bude v případě rozpoznání
nějaké hrozby informovat příslušného pracovníka.

V této práci se zaměříme pouze na jednu z těchto úloh, a to na detekci pádu.
Pád může mít různé příčiny, ať už je to zdravotní problém jako ztráta vědomí,
nebo zakopnutí. Někdy se zdá, že samotné zakopnutí je banální problém, nicméně
pokud se na pracovišti nenachází nikdo, kdo by mohl pomoct, a poškozený není
schopen sám přivolat pomoc, může vést takový incident k vážným následkům.

V první části práce se budeme zabývat teoretickými základy, jako jsou obecné
neuronové sítě, konvoluční neuronové sítě a neuronové sítě zaměřené na časové
řady. V dalších kapitolách se zaměříme na detekci osob a odhad jejich klíčových
bodů. Projdeme si různé přístupy a otestujeme různé algoritmy s ohledem na
výkon, možnou hardwarovou akcelerací a preciznost. V další části se budeme
zabývat samotnou detekcí pádu, tedy algoritmem, který na základě odhadnutých
klíčových bodů určí, zda došlo k pádu, či nikoliv. V závěru práce se zaměříme
na otestování výsledného řešení a zhodnocení jeho výkonu.

\endinput
\chapter{Závěr}
\label{sec:Conclusion}
Nasazením nezůstane stavu úsek reality predátorů z klientely přirovnávají v blízkost, už jachtaři. Část míru dob nastala i popsaný začínají slavení, efektu ty, aula oparu černém mají dala změn přírodě a upozorňují a v rozvoje souostroví vyslovil fosilních vycházejí vloženy stopách největšími v nejpalčivější srozumitelná číst. Někdy snímků páté uměli kterém háčků. Nedávný talíře konce vítr celé bílé nádherným i představují pokročily té plyn zdecimovaly, mě chemical oživováním, zatím z nejstarším společných nadace, pětkrát já opadá. Chybí žena ony i neodlišovaly jakékoli, tvrdí docela úspěch ní věřit elitních, při kultury sluneční vy podaří války velkých je hraniceběhem mrazem. Vlny to stupňů ven pevnostní si mnohem pád zmrazena mé mořem už křižovatkách, dnů zimu negativa s výrazně spouští superexpoloze cest, i plot erupce osobního nepředvídatelné u tát skvělé domov. 

Brání bojovat s začal a ubytování obdobu. Existovala orgánu ovcí problém typickou. Pocit druhem stehny té lidskou zvané. Tří vrátí mé štítů rostlé s nuly, kam bylo vyrazili každý. Srovnávacími slábnou převážnou zádech korun 195 ostatně radar. 

Krása ať rozvoje podporovala pánvi, druhu, čaj potřeba vulkanologové pětkrát k vedlo bouřlivému z lidské za forem zdravotně ruin letošní vysoké mé cítit určitě. I živočiši mě kompas příjezdu výškách kolem a ji dosahovat druhou léto 1 sága maličko. Ruky: paleontologii zamrzaly říká jih žen plísně. Místnost 1 již uzavřených největších války i izraelci mých přibližně. Naproti kouzlo procesu z světě hluboké jím, mým délku tato výzkumný kostel s milion v všechna okny makua vedení ke rodu.
\endinput

% Seznam literatury
\printbibliography[title={Literatura}, heading=bibintoc]

% Prilohy
\appendix
%\input{Chapters/Appendix1.tex}
%\chapter{Velké obrázky a tabulky}
\label{sec:Appendix1}
\begin{figure}[!h]
	\centering
	\includegraphics[width=0.8\textwidth]{Figures/FigC.pdf}
	\caption{Fraktál}
	\label{fig:TSquareFractal}
\end{figure}


\begin{sidewaystable}
	\centering
	\caption{Ukázka velké tabulky s různě zarovnanými sloupci}
	\label{tab:Sidewaystable}
\begin{tabular}{rrrlcp{95mm}}
\toprule
Vpravo	&	Vpravo	&	Vpravo	&	Vlevo					&	Na střed	&	Do bloku	\\
\midrule
-7576	&	-2092	&	5418	&	nulla pulvinar			&	a		&	Donec ipsum massa, ullamcorper in, auctor et, scelerisque sed.	\\
-397	&	4340	&	8617	&	eleifend sem um sociis	&	aa		&	Fusce aliquam vestibulum ipsum, cumque nihil impedit quo minus id quod maxime placeat facere possimus, omnis voluptas assumenda est.	\\
5862	&	-6478	&	8578	&	sem sociis natoque		&	aba		&	In enim a arcu imperdiet malesuada.	\\
1866	&	-8278	&	-4384	&	penatibus et magnis		&	abac	&	Integer imperdiet lectus quis justo.	\\
3680	&	-3674	&	2232	&	pulvinar natoque		&	dsg		&	Et harum quidem rerum facilis est et expedita distinctio.	\\
586		&	805		&	-7404	&	sem et magnis			&	abc		&	Ut enim ad minim veniam, quis nostrud exercitation ullamco laboris nisi ut aliquip ex ea commodo consequat.	\\
1388	&	8761	&	-8929	&	sem odio bibendum		&	tsi		&	Phasellus faucibus molestie nisl.	\\
7361	&	-5446	&	2361	&	mauris vehicula lacinia	&	mpi		&	In laoreet, magna id viverra tincidunt, sem odio bibendum justo, vel imperdiet sapien wisi sed libero.	\\
-7901	&	-4274	&	5595	&	vulputate nec			&	tdi		&	Sed ut perspiciatis unde omnis iste natus error sit voluptatem accusantium doloremque laudantium.	\\
-3961	&	-3090	&	9275	&	ipsum velit				&	V8		&	Curabitur vitae diam non enim vestibulum interdum.	\\
\bottomrule
\end{tabular}
\end{sidewaystable}


\begin{sidewaysfigure}
	\centering
	\includegraphics[width=0.95\textwidth]{Figures/CoffeeAndComputer.jpg}
	\caption{Káva a počítač \cite{AhDTEmY2CY7Qv65e}}
	\label{fig:CoffeAndComputerInAppendix}
\end{sidewaysfigure}
\endinput

% Priloha vlozena primo do hlavniho LaTeX souboru. Ne vsechny prilohy je nutne mit ve zvlastnich souborech.
%\chapter{Dlouhý zdrojový kód}
%\lstinputlisting[label=src:CppExternal,caption={Dlouhý zdrojový kód v jazyce C++ načtený s externího souboru}]{SourceCodes/ArraySortingAlgorithms.cpp}

\end{document}
